\newcommand*{\includesDirectory}{includes}
\newcommand*{\settingsDirectory}{\includesDirectory/settings}
\newcommand*{\tikzDirectory}{\includesDirectory/tikz}
\newcommand*{\proovesDirectory}{\includesDirectory/prooves}
\newcommand*{\oneDirectory}{\proovesDirectory/one}
\newcommand*{\listingsDirectory}{\includesDirectory/listings}


\documentclass[a4paper,12pt,oneside]{book}
\usepackage{geometry}
\usepackage{indentfirst}

\geometry{
    left=35mm,
    right=25mm,
    top=25mm,
    bottom=25mm
}

\renewcommand{\baselinestretch}{1.5}
\usepackage{amssymb}
\usepackage{amsmath}



\renewcommand{\theequation}{\Roman{section}.\arabic{equation}}
\usepackage[T1]{fontenc}
\usepackage[polish]{babel}
\usepackage{lmodern}
\usepackage{fancyhdr}
\pagestyle{fancy}
\fancyhf{}
\renewcommand{\headrulewidth}{0pt}
\fancyhead[R]{\thepage}

\fancypagestyle{plain}{
    \fancyhf{}
    \renewcommand{\headrulewidth}{0pt}
    \fancyhead[R]{\thepage}
}
\usepackage{titletoc}
\usepackage{hyperref}
\setcounter{tocdepth}{1}

\renewcommand\thechapter{Rozdział \Roman{chapter}}
\renewcommand\thesection{\arabic{section}.}
\renewcommand\thesubsection{\arabic{section}.\arabic{subsection}}
\contentsmargin{0.5cm}
\newlength\chapterlength
\settowidth\chapterlength{\hspace{2.5cm}}

\titlecontents{chapter}
    [\chapterlength] %5.3
    {\vspace{0.2cm}}
    {\contentslabel[\thecontentslabel]{\chapterlength}}%\thecontentslabel
    {\hspace*{-\chapterlength}}% unnumbered chapters
    {\titlerule*[1cm]{.}\contentspage}[\vspace{0.2cm}]%

\titlecontents{section}
    [\chapterlength] %5.3
    {\small}
    {\contentslabel[\thecontentslabel]{0.5cm}}
    {}
    {\titlerule*[0.5cm]{.}\contentspage}[]




\usepackage{titlesec}


\titleformat{\chapter}[block]
  {\normalfont\huge}{\thechapter\vspace{-13pt}\\}{0pt}{\LARGE}
\titlespacing*{\chapter}{0pt}{0pt}{0pt}

\titleformat{\section}[block]
  {\normalfont}
  {\makebox[0.5cm][l]{\thesection}}{10pt}{}
\titlespacing*{\section}{0pt}{0pt}{0pt}

\titleformat{\subsection}[block]
  {\normalfont}
  {\hspace{1cm}\makebox[0.5cm][l]{\thesubsection}}{10pt}{}
\titlespacing*{\subsection}{0pt}{0pt}{0pt}
\usepackage{listings}

\usepackage{xcolor}

\definecolor{codegreen}{rgb}{0,0.6,0}
\definecolor{codegray}{rgb}{0.5,0.5,0.5}
\definecolor{codepurple}{rgb}{0.58,0,0.82}
\definecolor{backcolour}{rgb}{0.95,0.95,0.92}


\lstdefinestyle{mystyle}{
    backgroundcolor=\color{backcolour},   
    commentstyle=\color{codegreen},
    keywordstyle=\color{magenta},
    numberstyle=\tiny\color{codegray},
    stringstyle=\color{codepurple},
    basicstyle=\ttfamily\footnotesize,
    breakatwhitespace=false,         
    breaklines=true,                 
    captionpos=b,                    
    keepspaces=true,                 
    numbers=left,                    
    numbersep=5pt,                  
    showspaces=false,                
    showstringspaces=false,
    showtabs=false,                  
    tabsize=2
}

\lstset{style=mystyle}

\newcommand{\scala}[1]{\lstinline[language=scala]{#1}}
\usepackage{caption}
\newcommand{\srcpl}{\\Źródło}
\newcommand{\mysrc}{{\srcpl}: opracowanie własne}

\usepackage{
	tikz,
	tzplot
}
\usetikzlibrary{
	shapes,
	quotes,
	angles,
	calc
}

\tikzset {
    point/.style args = {reference: #1 x: #2 y: #3 label: #4 color: #5}{
        insert path = {
            \pgfextra{
                \node[
                	circle,
                	fill = #5,
                	minimum size = 1 cm,
                	label = {#4},
                	scale = 0.1
                	]
                	at (#2,#3)
                	(#1) 
                	{};
            }
        }
    }
}
\tikzset {
    ellipse_block/.style args = {reference: #1 text: #2}{
        insert path = {
            \pgfextra{
                \node
                	[
                		ellipse,
                		draw = black
                	]
                	at (#1)
                	(#1_block) 
                	{#2};
            }
        }
    }
}
\tikzset {
   rectangle_block/.style ={draw=black,rectangle}
}


\newcommand{\block}[2]{

  \node[rectangle_block,below] at (#1.north) (#1_block) {
  	\begin{minipage}{5.2cm}\begin{tiny}\begin{flushleft}\begin{spacing}{1.5}
			#2
	\end{spacing}\end{flushleft}\end{tiny}\vspace{-8mm}\end{minipage}
  };
  
  
  
  \node at (#1_block.south) (#1_south) {};
 
}

\newcommand{\centerblock}[2]{

  \node[rectangle_block] at (#1.center) (#1_block) {
  	\begin{minipage}{5.2cm}\begin{tiny}\begin{flushleft}\begin{spacing}{1.5}
			#2
	\end{spacing}\end{flushleft}\end{tiny}\vspace{-8mm}\end{minipage}
  };
  
  
  
  \node at (#1_block.south) (#1_south) {};
 
}

\newcommand{\smallblock}[2]{

  \node[rectangle_block,below] at (#1.north) (#1_block) {
  	\begin{minipage}{5cm}\begin{small}\begin{center}\begin{spacing}{1.5}
			#2
	\end{spacing}\end{center}\end{small}\vspace{-12mm}\end{minipage}
  };
  
  
  
  \node at (#1_block.south) (#1_south) {};
 
}

\newcommand{\chooseblock}[2]{

  \node[diamond,draw =black,below, aspect = 3] at (#1.north) (#1_block) {
  	\tiny{
  		#2
	}
  };
  
  \node[below right] at (#1_block.south) {T};
  
  
  \node at (#1_block.south) (#1_south) {};
 
}
%\newcommand{\newnode}[2]{
%	\node at (#1) (#2) {}

%}
\tikzset {
    pointAndLabel/.style args = {reference: #1 x: #2 y: #3}{
        insert path = {
            \pgfextra{
                \path[point = reference: #1 x: #2 y: #3 label: $#1$ color: black];
            }
        }
    }
}
\tikzset {
    randomVertices/.style ={
        insert path = {
            \pgfextra{
               \path[pointAndLabel = 
    				reference: 1 
    				x: 1.371503889373591
    				y: 0.09390715693458951
				];
				\path[pointAndLabel = 
    				reference: 2 
    				x: 8.47977671920242
    				y: 1.6031886319353594
				];
				\path[pointAndLabel = 
    				reference: 3 
    				x: 6.045551430098136
    				y: 0.8234401782232126
				];
				\path[pointAndLabel = 
    				reference: 4 
    				x: 1.0581168893921622
    				y: 7.196652534121291
				];
				\path[pointAndLabel = 
    				reference: 5 
    				x: 7.54893452550192
    				y: 6.229955423976583
				];
				\path[pointAndLabel = 
    				reference: 6 
    				x: 3.452073817126815
    				y: 0.42802148287650943
				];
				\path[pointAndLabel = 
    				reference: 7 
    				x: 1.9234285748582658
    				y: 3.8128351776520875
				];
				\path[pointAndLabel = 
    				reference: 8 
    				x: 1.4291333394587036
    				y: 2.298954766988656
				];
				\path[pointAndLabel = 
    				reference: 9 
    				x: 4.158998560109078
    				y: 1.1819955486088674
				];
				\path[pointAndLabel = 
    				reference: 10 
    				x: 9.607809179355511
    				y: 0.9208848746431597
				];
				\path[pointAndLabel = 
    				reference: 11 
    				x: 4.505748586095314
    				y: 4.275899288243908
				];
				\path[pointAndLabel = 
    				reference: 12 
    				x: 4.335033269031589
    				y: 6.636185965600352
				];
				\path[pointAndLabel = 
    				reference: 13 
    				x: 8.265341405959747
    				y: 8.370333801938221
				];
				\path[pointAndLabel = 
    				reference: 14 
    				x: 3.954303586943823
    				y: 3.5057031654014024
				];
				\path[pointAndLabel = 
    				reference: 15 
    				x: 1.9259642615328465
    				y: 4.637048240596112
				];
				\path[pointAndLabel = 
    				reference: 16 
    				x: 9.450732288466309
    				y: 5.239112331048348
				];
				\path[pointAndLabel = 
    				reference: 17 
    				x: 9.452180434267264
    				y: 9.474236735440195
				];
				\path[pointAndLabel = 
    				reference: 18 
    				x: 2.062657121585195
    				y: 7.013735646100131
				];
				\path[pointAndLabel = 
    				reference: 19 
    				x: 1.3007739087073178
    				y: 8.606241672641316
				];
				\path[pointAndLabel = 
    				reference: 20 
    				x: 0.8654915304555744
    				y: 3.3090677225981358
				];
            }
        }
    }
}
\pgfarrowsdeclare{arcs}{arcs}{...}{
    \pgfsetdash{}{0pt}
    \pgfsetroundjoin
    \pgfsetroundcap
    \pgfpathmoveto{\pgfpoint{-3pt}{3pt}}
    \pgfpatharc{180}{270}{3pt}
    \pgfpatharc{90}{180}{3pt}
    \pgfusepathqstroke
}

\begin{document}

\newcommand{\alignspace}{\vspace{- 3pt}}
\renewcommand{\thelstlisting}{\arabic{lstlisting}}
\counterwithout{figure}{chapter}
\counterwithout{lstlisting}{chapter}
\counterwithout{equation}{chapter}
\setlength{\abovedisplayskip}{0ex}
\setlength{\belowdisplayskip}{0ex}
\setlength{\abovedisplayshortskip}{0ex}
\setlength{\belowdisplayshortskip}{0ex}
\setlength{\jot}{1.5ex}


    \thispagestyle{empty}
    \begin{center}
        \Huge{Otoczki wypukłe}
    \end{center}
    \tableofcontents	

  
    \chapterwithout{Wstęp}        

    \chapter{Omówienie teoretyczne otoczki wypukłej na płaszczyźnie}
    Otoczka wypukła zbioru punktów w swojej najbardziej podstawowej postaci jest wielokątem wypukłym obejmującym wszystkie punkty ze zbioru punktów leżących na płaszczyźnie w taki sposób, aby wielokąt ten miał jak najmniejsze pole.    
    \begin{figure}[h!]
        	\begin{center}
    			\begin{tikzpicture}
    	
    \path[randomVertices];
    	
    \draw (1.center)--
          (10.center)--
    	  (17.center)--
    	  (19.center)--
    	  (4.center)--
    	  (20.center)--
    	  cycle;
    
\end{tikzpicture}  
    			\caption{Otoczka wypukła na płaszczyźnie\mysrc}\label{fig:example_hull}
    	\end{center}
    	\vspace{-0.8cm}
    \end{figure}
    
    Dobrą reprezentacją otoczki wypukłej w świecie fizycznym może być grupa gwoździ przybita do płaskiej powierzchni i następnie opleciona ciasno sznurkiem. Gwoździe stykające się ze sznurkiem stanowić będą wierzchołki otoczki wypukłej tej grupy gwoździ.
        \section{Otoczka wypukła zbioru punktów}
        W celu wyznaczenia otoczki wypukłej dla najbardziej ogólnego przypadku --- nieuporządkowanego zbioru punktów na płaszczyźnie, możemy wykorzystać dwa najbardziej popularne algorytmy algorytm Jarvisa oraz algorytm Grahama.
        
        W wykorzystywanych algorytmach istotnym elementem jest sortowanie punktów względem wartości, które zostały przedstawione na rysunku \ref{fig:axis}. W zależności od tego, w jakiej postaci określone będą dane punkty, należy dokonać odpowiednich obliczeń.
        \begin{figure}[h!]
        	\begin{center}
    			\begin{tikzpicture}
    	
   \draw [-arcs] 
   (-2.5,0) --
   (2.5,0)  
   node[above] {$X$};
   
   \draw [-arcs] 
   (0,-2.5) --
   (0,2.5)   
   node[left] {$Y$} ;
   
   
   \path[point = reference: p x: -1.5 y: 1.5 label: $p$ color: black];
   
   \draw [dashed] 
   (0,0) -- node [above right = 0 mm] {\small{$R$}}
   (p.center);
   
   \draw [dashed] 
   (p.center) --++
   (1.5,0) 
   node[right] {\footnotesize{$y\left(p\right) = \left|R\right|\cdot \sin \theta$}} ;
   
   \draw [dashed] 
   (p.center) --++
   (0,-1.5) 
   node[below] {\footnotesize{$x\left(p\right) = \left|R\right|\cdot \cos \theta$}};
   
   
   \tzanglemark[-arcs](2.5,0)(0,0)(p){\small{$\theta$}}[pos=1.4](.6)
  

    
\end{tikzpicture} 
    			\caption{Punkt na układzie współrzędnych\mysrc}\label{fig:axis}
    	\end{center}
    	\end{figure}\vspace{-8 mm}\\
    	Gdzie poszczególne symbole oznaczają:\\
    	\begin{tabular}{rcl}
    	$p$ & - & rozpatrywany punkt\\
    	$x\left( p \right)$ & - & odcięta punktu $p$\\
    	$y\left( p \right)$ & - & rzędna punktu $p$\\
    	$R$ & - & długość wektora wodzącego punktu $p$\\
    	$\theta$ & - & kąt nachylenia wektora wodzącego punktu $p$ do osi $OX$
    	\end{tabular}\\
    	
    	Istotną komplikację z punktu widzenia obliczeń  w algorytmach może stanowić wyznaczenie wartości kąta $\theta$ (ze względu na potrzebę wykorzystania funkcji trygonometrycznych), którego dokładną wartość można wyliczyć za pomocą wzoru \ref{eq:theta}.\\
        \begin{align}\label{eq:theta}
        	\theta\left(p\right) = \left\{
        		\begin{array}{lclcr}
        			\operatorname {arctg} \frac{y\left(p\right)}{x\left(p\right)}& \text{dla} & x\left(p\right) > 0; \\
        			\operatorname {arctg} \frac{y\left(p\right)}{x\left(p\right)} + \pi& \text{dla} & x\left(p\right) < 0 & \wedge & y\left(p\right) \geqslant 0; \\
        			\operatorname {arctg} \frac{y\left(p\right)}{x\left(p\right)} - \pi& \text{dla} & x\left(p\right) < 0 & \wedge & y\left(p\right) < 0; \\
        			\frac{\pi}{2}& \text{dla} & x\left(p\right) = 0 & \wedge & y\left(p\right) < 0; \\
        			-\frac{\pi}{2}& \text{dla} & x\left(p\right) = 0 & \wedge & y\left(p\right) > 0. \\
        		\end{array}
        	\right.
        \end{align}\\
         Należy jednak zauważyć, że do celów sortowania wystarczy zastosować funkcję $\alpha\left(p\right)$, taką, że dla dowolnej pary punktów $p_1, p_2$ spełnione będą warunki \ref{eq:condition1} oraz \ref{eq:condition2}. 
    	\begin{align}\label{eq:condition1}
    		\alpha\left(p_1\right) <  \alpha\left(p_2\right)
    		\Leftrightarrow \theta\left(p_1\right) <  \theta\left(p_2\right)\\
    		\label{eq:condition2}
    		\alpha\left(p_1\right) =  \alpha\left(p_2\right)
    		\Leftrightarrow \theta\left(p_1\right) = \theta\left(p_2\right)
    	\end{align}
		Przykładowa funkcja $\alpha\left(p\right)$ zachowująca spełniająca warunki \ref{eq:condition1} i \ref{eq:condition2} została przedstawiona za pomocą wzoru \ref{eq:alpha1}.\\
   		\begin{align}\label{eq:alpha1}
        	\alpha\left( p\right) = \left\{
        		\begin{array}{lclcr}
        			\frac{y\left(p\right)}{d\left(p\right)} & \text{dla} &
        			x\left(p\right) \geqslant 0 & \wedge & y\left(p\right) \geqslant 0; \\
        			2 - \frac{y\left(p\right)}{d\left(p\right)} & \text{dla} &
        			x\left(p\right) < 0 & \wedge & y\left(p\right) \geqslant 0; \\
        			2 + \frac{\left|y\left(p\right)\right|}{d\left(p\right)} & \text{dla} &
        			x\left(p\right) < 0 & \wedge & y\left(p\right) < 0; \\
        			4 - \frac{\left|y\left(p\right)\right|}{d\left(p\right)} & \text{dla} &
        			x\left(p\right) \geqslant 0 & \wedge & y\left(p\right) < 0.
        		\end{array}
        	\right.
        \end{align}\\
    Gdzie funkcja $d\left(p\right)$ określona jest zgodnie ze wzorem \ref{eq:dp}.
    \begin{align}\label{eq:dp}
        	d\left(p\right) = \left|x\left(p\right)\right| + \left|y\left(p\right)\right|
    \end{align}
    
     W celu udowodnienia, że funkcja $\alpha\left(p\right)$ spełnia warunki \ref{eq:condition1} i \ref{eq:condition2}, musimy wykazać, że funkcja $\gamma\left(\theta\right)$ (przedstawiona we wzorze \ref{eq:gamma}), która stanowi przekształcenie funkcji $\alpha\left(p\right)$ takie, że jest w pełni zależna od wartości $\theta$, a wartości $x\left(p\left(\theta\right)\right)$ oraz $y\left(p\left(\theta\right)\right)$ są określone zgodnie ze wzorem \ref{eq:ptheta}, jest rosnąca w każdym swoim przedziale.
     \begin{align}\label{eq:gamma}
        	\gamma\left(\theta\right) = \alpha\left(p\left(\theta\right)\right)
     \end{align}
     \begin{align}\label{eq:ptheta}
        	p\left(\theta\right) = \left( 
        		\left|R\right| \cdot \cos \theta , 
        		\left|R\right| \cdot \sin \theta
        	\right)
     \end{align}
     
    

     Należy zauważyć, że poszczególne przypadki opisane we wzorze \ref{eq:alpha1} dzielą układ współrzędnych na cztery ćwiartki. Zatem z rysunku \ref{fig:axis} wynikają zależności
     \ref{eq:xytotheta1}, 
     \ref{eq:xytotheta2}, 
     \ref{eq:xytotheta3} oraz
     \ref{eq:xytotheta4}.
     \begin{align}\label{eq:xytotheta1}
    		x\left(p\right) \geqslant 0 \ \wedge \  y\left(p\right) \geqslant 0  \ & \Leftrightarrow \ \theta \in \left\langle 0;\frac{\pi}{2}\right\rangle \\
    		\label{eq:xytotheta2}
    		x\left(p\right) < 0 \ \wedge \ y\left(p\right) \geqslant 0 \ & \Leftrightarrow \
  		\theta \in \left( \frac{\pi}{2};\pi\right\rangle\\
  		\label{eq:xytotheta3}
    		x\left(p\right) < 0 \ \wedge \ y\left(p\right) < 0 \ & \Leftrightarrow \
  		\theta \in \left( \pi;\frac{3\pi}{2}\right)\\
  		\label{eq:xytotheta4}
  		x\left(p\right) \geqslant 0 \ \wedge \ y\left(p\right) < 0 \ & \Leftrightarrow \
  	\theta \in \left\langle
  		\frac{
  			3\pi
  		}{
  			2
  		};
  		2\pi
  	\right)
    \end{align}
    Zatem w celu udowodnienia spełnienia warunków \ref{eq:condition1} i \ref{eq:condition2} przez funkcję $\alpha\left(p\right)$ (\ref{eq:alpha1}) należy udowodnić, że funkcja $\gamma\left(\theta\right)$ (\ref{eq:gamma}) jest rosnąca (jej pochodna jest większa od zera) we wszystkich przedziałach opisanych w zależnościach
     \ref{eq:xytotheta1}, 
     \ref{eq:xytotheta2}, 
     \ref{eq:xytotheta3} oraz
     \ref{eq:xytotheta4}.
    
    Obliczenia dla tych czterech przedziałów przedstawiają się w następujący sposób. Dla każdego z przedziałów możliwe jest uproszczenie funkcji $d\left(p\right)$, tak aby pozbyć się modułu, przez co różniczkowanie funkcji $\gamma\left(\theta\right)$ staje się łatwiejsze.
 
    \newcommand*{\includesDirectory}{includes}
\newcommand*{\settingsDirectory}{\includesDirectory/settings}
\newcommand*{\tikzDirectory}{\includesDirectory/tikz}
\newcommand*{\proovesDirectory}{\includesDirectory/prooves}
\newcommand*{\oneDirectory}{\proovesDirectory/one}
\newcommand*{\listingsDirectory}{\includesDirectory/listings}


\documentclass[a4paper,12pt,oneside]{book}
\usepackage{geometry}
\usepackage{indentfirst}

\geometry{
    left=35mm,
    right=25mm,
    top=25mm,
    bottom=25mm
}

\renewcommand{\baselinestretch}{1.5}
\usepackage{amssymb}
\usepackage{amsmath}



\renewcommand{\theequation}{\Roman{section}.\arabic{equation}}
\usepackage[T1]{fontenc}
\usepackage[polish]{babel}
\usepackage{lmodern}
\usepackage{fancyhdr}
\pagestyle{fancy}
\fancyhf{}
\renewcommand{\headrulewidth}{0pt}
\fancyhead[R]{\thepage}

\fancypagestyle{plain}{
    \fancyhf{}
    \renewcommand{\headrulewidth}{0pt}
    \fancyhead[R]{\thepage}
}
\usepackage{titletoc}
\usepackage{hyperref}
\setcounter{tocdepth}{1}

\renewcommand\thechapter{Rozdział \Roman{chapter}}
\renewcommand\thesection{\arabic{section}.}
\renewcommand\thesubsection{\arabic{section}.\arabic{subsection}}
\contentsmargin{0.5cm}
\newlength\chapterlength
\settowidth\chapterlength{\hspace{2.5cm}}

\titlecontents{chapter}
    [\chapterlength] %5.3
    {\vspace{0.2cm}}
    {\contentslabel[\thecontentslabel]{\chapterlength}}%\thecontentslabel
    {\hspace*{-\chapterlength}}% unnumbered chapters
    {\titlerule*[1cm]{.}\contentspage}[\vspace{0.2cm}]%

\titlecontents{section}
    [\chapterlength] %5.3
    {\small}
    {\contentslabel[\thecontentslabel]{0.5cm}}
    {}
    {\titlerule*[0.5cm]{.}\contentspage}[]




\usepackage{titlesec}


\titleformat{\chapter}[block]
  {\normalfont\huge}{\thechapter\vspace{-13pt}\\}{0pt}{\LARGE}
\titlespacing*{\chapter}{0pt}{0pt}{0pt}

\titleformat{\section}[block]
  {\normalfont}
  {\makebox[0.5cm][l]{\thesection}}{10pt}{}
\titlespacing*{\section}{0pt}{0pt}{0pt}

\titleformat{\subsection}[block]
  {\normalfont}
  {\hspace{1cm}\makebox[0.5cm][l]{\thesubsection}}{10pt}{}
\titlespacing*{\subsection}{0pt}{0pt}{0pt}
\usepackage{listings}

\usepackage{xcolor}

\definecolor{codegreen}{rgb}{0,0.6,0}
\definecolor{codegray}{rgb}{0.5,0.5,0.5}
\definecolor{codepurple}{rgb}{0.58,0,0.82}
\definecolor{backcolour}{rgb}{0.95,0.95,0.92}


\lstdefinestyle{mystyle}{
    backgroundcolor=\color{backcolour},   
    commentstyle=\color{codegreen},
    keywordstyle=\color{magenta},
    numberstyle=\tiny\color{codegray},
    stringstyle=\color{codepurple},
    basicstyle=\ttfamily\footnotesize,
    breakatwhitespace=false,         
    breaklines=true,                 
    captionpos=b,                    
    keepspaces=true,                 
    numbers=left,                    
    numbersep=5pt,                  
    showspaces=false,                
    showstringspaces=false,
    showtabs=false,                  
    tabsize=2
}

\lstset{style=mystyle}

\newcommand{\scala}[1]{\lstinline[language=scala]{#1}}
\usepackage{caption}
\newcommand{\srcpl}{\\Źródło}
\newcommand{\mysrc}{{\srcpl}: opracowanie własne}
\usepackage[style=ieee]{biblatex}
\addbibresource{refs.bib}

\usepackage{
	tikz,
	tzplot
}
\usetikzlibrary{
	shapes,
	quotes,
	angles,
	calc
}

\tikzset {
    point/.style args = {reference: #1 x: #2 y: #3 label: #4 color: #5}{
        insert path = {
            \pgfextra{
                \node[
                	circle,
                	fill = #5,
                	minimum size = 1 cm,
                	label = {#4},
                	scale = 0.1
                	]
                	at (#2,#3)
                	(#1) 
                	{};
            }
        }
    }
}
\tikzset {
    ellipse_block/.style args = {reference: #1 text: #2}{
        insert path = {
            \pgfextra{
                \node
                	[
                		ellipse,
                		draw = black
                	]
                	at (#1)
                	(#1_block) 
                	{#2};
            }
        }
    }
}
\tikzset {
   rectangle_block/.style ={draw=black,rectangle}
}


\newcommand{\block}[2]{

  \node[rectangle_block,below] at (#1.north) (#1_block) {
  	\begin{minipage}{5.2cm}\begin{tiny}\begin{flushleft}\begin{spacing}{1.5}
			#2
	\end{spacing}\end{flushleft}\end{tiny}\vspace{-8mm}\end{minipage}
  };
  
  
  
  \node at (#1_block.south) (#1_south) {};
 
}

\newcommand{\centerblock}[2]{

  \node[rectangle_block] at (#1.center) (#1_block) {
  	\begin{minipage}{5.2cm}\begin{tiny}\begin{flushleft}\begin{spacing}{1.5}
			#2
	\end{spacing}\end{flushleft}\end{tiny}\vspace{-8mm}\end{minipage}
  };
  
  
  
  \node at (#1_block.south) (#1_south) {};
 
}

\newcommand{\smallblock}[2]{

  \node[rectangle_block,below] at (#1.north) (#1_block) {
  	\begin{minipage}{5cm}\begin{small}\begin{center}\begin{spacing}{1.5}
			#2
	\end{spacing}\end{center}\end{small}\vspace{-12mm}\end{minipage}
  };
  
  
  
  \node at (#1_block.south) (#1_south) {};
 
}

\newcommand{\chooseblock}[2]{

  \node[diamond,draw =black,below, aspect = 3] at (#1.north) (#1_block) {
  	\tiny{
  		#2
	}
  };
  
  \node[below right] at (#1_block.south) {T};
  
  
  \node at (#1_block.south) (#1_south) {};
 
}
%\newcommand{\newnode}[2]{
%	\node at (#1) (#2) {}

%}
\tikzset {
    pointAndLabel/.style args = {reference: #1 x: #2 y: #3}{
        insert path = {
            \pgfextra{
                \path[point = reference: #1 x: #2 y: #3 label: $#1$ color: black];
            }
        }
    }
}
\tikzset {
    randomVertices/.style ={
        insert path = {
            \pgfextra{
               \path[pointAndLabel = 
    				reference: 1 
    				x: 1.371503889373591
    				y: 0.09390715693458951
				];
				\path[pointAndLabel = 
    				reference: 2 
    				x: 8.47977671920242
    				y: 1.6031886319353594
				];
				\path[pointAndLabel = 
    				reference: 3 
    				x: 6.045551430098136
    				y: 0.8234401782232126
				];
				\path[pointAndLabel = 
    				reference: 4 
    				x: 1.0581168893921622
    				y: 7.196652534121291
				];
				\path[pointAndLabel = 
    				reference: 5 
    				x: 7.54893452550192
    				y: 6.229955423976583
				];
				\path[pointAndLabel = 
    				reference: 6 
    				x: 3.452073817126815
    				y: 0.42802148287650943
				];
				\path[pointAndLabel = 
    				reference: 7 
    				x: 1.9234285748582658
    				y: 3.8128351776520875
				];
				\path[pointAndLabel = 
    				reference: 8 
    				x: 1.4291333394587036
    				y: 2.298954766988656
				];
				\path[pointAndLabel = 
    				reference: 9 
    				x: 4.158998560109078
    				y: 1.1819955486088674
				];
				\path[pointAndLabel = 
    				reference: 10 
    				x: 9.607809179355511
    				y: 0.9208848746431597
				];
				\path[pointAndLabel = 
    				reference: 11 
    				x: 4.505748586095314
    				y: 4.275899288243908
				];
				\path[pointAndLabel = 
    				reference: 12 
    				x: 4.335033269031589
    				y: 6.636185965600352
				];
				\path[pointAndLabel = 
    				reference: 13 
    				x: 8.265341405959747
    				y: 8.370333801938221
				];
				\path[pointAndLabel = 
    				reference: 14 
    				x: 3.954303586943823
    				y: 3.5057031654014024
				];
				\path[pointAndLabel = 
    				reference: 15 
    				x: 1.9259642615328465
    				y: 4.637048240596112
				];
				\path[pointAndLabel = 
    				reference: 16 
    				x: 9.450732288466309
    				y: 5.239112331048348
				];
				\path[pointAndLabel = 
    				reference: 17 
    				x: 9.452180434267264
    				y: 9.474236735440195
				];
				\path[pointAndLabel = 
    				reference: 18 
    				x: 2.062657121585195
    				y: 7.013735646100131
				];
				\path[pointAndLabel = 
    				reference: 19 
    				x: 1.3007739087073178
    				y: 8.606241672641316
				];
				\path[pointAndLabel = 
    				reference: 20 
    				x: 0.8654915304555744
    				y: 3.3090677225981358
				];
            }
        }
    }
}
\pgfarrowsdeclare{arcs}{arcs}{...}{
    \pgfsetdash{}{0pt}
    \pgfsetroundjoin
    \pgfsetroundcap
    \pgfpathmoveto{\pgfpoint{-3pt}{3pt}}
    \pgfpatharc{180}{270}{3pt}
    \pgfpatharc{90}{180}{3pt}
    \pgfusepathqstroke
}

\begin{document}

\newcommand{\alignspace}{\vspace{- 3pt}}
\renewcommand{\thelstlisting}{\arabic{lstlisting}}
\counterwithout{figure}{chapter}
\counterwithout{lstlisting}{chapter}
\counterwithout{equation}{chapter}
\setlength{\abovedisplayskip}{0ex}
\setlength{\belowdisplayskip}{0ex}
\setlength{\abovedisplayshortskip}{0ex}
\setlength{\belowdisplayshortskip}{0ex}
\setlength{\jot}{1.5ex}


    \thispagestyle{empty}
    \begin{adjustwidth}{-1cm}{0cm}
    \begin{center}
    		\begin{large}
\textbf{UNIWERSYTET GDAŃSKI}

\vspace{.2 cm}
\textbf{WYDZIAŁ MATEMATYKI, FIZYKI I INFORMATYKI}
\end{large}

\vspace{3 cm}
\textbf{Marcin Belicki}

\textbf{numer albumu: 273417}
    \end{center}
    
\vspace{3 cm}
\textit{Kierunek studiów: Informatyka}

\vspace{3 cm}
\begin{center}
	\begin{large}
		\textbf{OTOCZKI WYPUKŁE}
	\end{large}
\end{center}

\vspace{3 cm}
\begin{flushright}
Praca magisterska

wykonana

pod kierunkiem

dr inż. Arkadiusz Mirakowski
\end{flushright}
\vspace{2 cm}
\begin{center}
\textbf{Gdańsk 2023}
\end{center}

    \end{adjustwidth}
    
    \tableofcontents	

  
    \chapterwithout{Wstęp}        

    \chapter{Otoczka wypukła na płaszczyźnie}
    Otoczka wypukła zbioru punktów w swojej najbardziej podstawowej postaci jest wielokątem wypukłym obejmującym wszystkie punkty ze zbioru punktów leżących na płaszczyźnie w taki sposób, aby wielokąt ten miał jak najmniejsze pole.    
    \begin{figure}[h!]
        	\begin{center}
    			\begin{tikzpicture}
    	
    \path[randomVertices];
    	
    \draw (1.center)--
          (10.center)--
    	  (17.center)--
    	  (19.center)--
    	  (4.center)--
    	  (20.center)--
    	  cycle;
    
\end{tikzpicture}  
    			\caption{Otoczka wypukła na płaszczyźnie\mysrc}\label{fig:example_hull}
    	\end{center}
    	\vspace{-0.8cm}
    \end{figure}
    
    Dobrą reprezentacją otoczki wypukłej w świecie fizycznym może być grupa gwoździ przybita do płaskiej powierzchni i następnie opleciona ciasno sznurkiem. Gwoździe stykające się ze sznurkiem stanowić będą wierzchołki otoczki wypukłej tej grupy gwoździ.
        \section{Otoczka wypukła zbioru punktów}
        W celu wyznaczenia otoczki wypukłej dla najbardziej ogólnego przypadku --- nieuporządkowanego zbioru punktów na płaszczyźnie, możemy wykorzystać dwa najbardziej popularne algorytmy algorytm Jarvisa oraz algorytm Grahama.
        
        W wykorzystywanych algorytmach istotnym elementem jest sortowanie punktów względem wartości, które zostały przedstawione na rysunku \ref{fig:axis}. W zależności od tego, w jakiej postaci określone będą dane punkty, należy dokonać odpowiednich obliczeń.
        \begin{figure}[h!]
        	\begin{center}
    			\begin{tikzpicture}
    	
   \draw [-arcs] 
   (-2.5,0) --
   (2.5,0)  
   node[above] {$X$};
   
   \draw [-arcs] 
   (0,-2.5) --
   (0,2.5)   
   node[left] {$Y$} ;
   
   
   \path[point = reference: p x: -1.5 y: 1.5 label: $p$ color: black];
   
   \draw [dashed] 
   (0,0) -- node [above right = 0 mm] {\small{$R$}}
   (p.center);
   
   \draw [dashed] 
   (p.center) --++
   (1.5,0) 
   node[right] {\footnotesize{$y\left(p\right) = \left|R\right|\cdot \sin \theta$}} ;
   
   \draw [dashed] 
   (p.center) --++
   (0,-1.5) 
   node[below] {\footnotesize{$x\left(p\right) = \left|R\right|\cdot \cos \theta$}};
   
   
   \tzanglemark[-arcs](2.5,0)(0,0)(p){\small{$\theta$}}[pos=1.4](.6)
  

    
\end{tikzpicture} 
    			\caption{Punkt na układzie współrzędnych\mysrc}\label{fig:axis}
    	\end{center}
    	\end{figure}\vspace{-8 mm}\\
    	Gdzie poszczególne symbole oznaczają:\\
    	\begin{tabular}{rcl}
    	$p$ & -- & rozpatrywany punkt\\
    	$x\left( p \right)$ & -- & odcięta punktu $p$\\
    	$y\left( p \right)$ & -- & rzędna punktu $p$\\
    	$R$ & -- & długość wektora wodzącego punktu $p$\\
    	$\theta$ & -- & kąt nachylenia wektora wodzącego punktu $p$ do osi $OX$
    	\end{tabular}\\
    	
    	Istotną komplikację z punktu widzenia obliczeń  w algorytmach może stanowić wyznaczenie wartości kąta $\theta$ (ze względu na potrzebę wykorzystania funkcji trygonometrycznych), którego dokładną wartość można wyliczyć za pomocą wzoru \ref{eq:theta}.\\
        \begin{align}\label{eq:theta}
        	\theta\left(p\right) = \left\{
        		\begin{array}{lclcr}
        			\operatorname {arctg} \frac{y\left(p\right)}{x\left(p\right)}& \text{dla} & x\left(p\right) > 0; \\
        			\operatorname {arctg} \frac{y\left(p\right)}{x\left(p\right)} + \pi& \text{dla} & x\left(p\right) < 0 & \wedge & y\left(p\right) \geqslant 0; \\
        			\operatorname {arctg} \frac{y\left(p\right)}{x\left(p\right)} - \pi& \text{dla} & x\left(p\right) < 0 & \wedge & y\left(p\right) < 0; \\
        			\frac{\pi}{2}& \text{dla} & x\left(p\right) = 0 & \wedge & y\left(p\right) < 0; \\
        			-\frac{\pi}{2}& \text{dla} & x\left(p\right) = 0 & \wedge & y\left(p\right) > 0. \\
        		\end{array}
        	\right.
        \end{align}\\
         Należy jednak zauważyć, że do celów sortowania wystarczy zastosować funkcję $\alpha\left(p\right)$, taką, że dla dowolnej pary punktów $p_1, p_2$ spełnione będą warunki \ref{eq:condition1} oraz \ref{eq:condition2}. 
    	\begin{align}\label{eq:condition1}
    		\alpha\left(p_1\right) <  \alpha\left(p_2\right)
    		\Leftrightarrow \theta\left(p_1\right) <  \theta\left(p_2\right)\\
    		\label{eq:condition2}
    		\alpha\left(p_1\right) =  \alpha\left(p_2\right)
    		\Leftrightarrow \theta\left(p_1\right) = \theta\left(p_2\right)
    	\end{align}
		Przykładowa funkcja $\alpha\left(p\right)$ zachowująca spełniająca warunki \ref{eq:condition1} i \ref{eq:condition2} została przedstawiona za pomocą wzoru \ref{eq:alpha1}.\\
   		\begin{align}\label{eq:alpha1}
        	\alpha\left( p\right) = \left\{
        		\begin{array}{lclcr}
        			\frac{y\left(p\right)}{d\left(p\right)} & \text{dla} &
        			x\left(p\right) \geqslant 0 & \wedge & y\left(p\right) \geqslant 0; \\
        			2 - \frac{y\left(p\right)}{d\left(p\right)} & \text{dla} &
        			x\left(p\right) < 0 & \wedge & y\left(p\right) \geqslant 0; \\
        			2 + \frac{\left|y\left(p\right)\right|}{d\left(p\right)} & \text{dla} &
        			x\left(p\right) < 0 & \wedge & y\left(p\right) < 0; \\
        			4 - \frac{\left|y\left(p\right)\right|}{d\left(p\right)} & \text{dla} &
        			x\left(p\right) \geqslant 0 & \wedge & y\left(p\right) < 0.
        		\end{array}
        	\right.
        \end{align}\\
    Gdzie funkcja $d\left(p\right)$ określona jest zgodnie ze wzorem \ref{eq:dp}.
    \begin{align}\label{eq:dp}
        	d\left(p\right) = \left|x\left(p\right)\right| + \left|y\left(p\right)\right|
    \end{align}
    
     W celu udowodnienia, że funkcja $\alpha\left(p\right)$ spełnia warunki \ref{eq:condition1} i \ref{eq:condition2}, musimy wykazać, że funkcja $\gamma\left(\theta\right)$ (przedstawiona we wzorze \ref{eq:gamma}), która stanowi przekształcenie funkcji $\alpha\left(p\right)$ takie, że jest w pełni zależna od wartości $\theta$, a wartości $x\left(p\left(\theta\right)\right)$ oraz $y\left(p\left(\theta\right)\right)$ są określone zgodnie ze wzorem \ref{eq:ptheta}, jest rosnąca w każdym swoim przedziale.
     \begin{align}\label{eq:gamma}
        	\gamma\left(\theta\right) = \alpha\left(p\left(\theta\right)\right)
     \end{align}
     \begin{align}\label{eq:ptheta}
        	p\left(\theta\right) = \left( 
        		\left|R\right| \cdot \cos \theta , 
        		\left|R\right| \cdot \sin \theta
        	\right)
     \end{align}
     
    

     Należy zauważyć, że poszczególne przypadki opisane we wzorze \ref{eq:alpha1} dzielą układ współrzędnych na cztery ćwiartki. Zatem z rysunku \ref{fig:axis} wynikają zależności
     \ref{eq:xytotheta1}, 
     \ref{eq:xytotheta2}, 
     \ref{eq:xytotheta3} oraz
     \ref{eq:xytotheta4}.
     \begin{align}\label{eq:xytotheta1}
    		x\left(p\right) \geqslant 0 \ \wedge \  y\left(p\right) \geqslant 0  \ & \Leftrightarrow \ \theta \in \left\langle 0;\frac{\pi}{2}\right\rangle \\
    		\label{eq:xytotheta2}
    		x\left(p\right) < 0 \ \wedge \ y\left(p\right) \geqslant 0 \ & \Leftrightarrow \
  		\theta \in \left( \frac{\pi}{2};\pi\right\rangle\\
  		\label{eq:xytotheta3}
    		x\left(p\right) < 0 \ \wedge \ y\left(p\right) < 0 \ & \Leftrightarrow \
  		\theta \in \left( \pi;\frac{3\pi}{2}\right)\\
  		\label{eq:xytotheta4}
  		x\left(p\right) \geqslant 0 \ \wedge \ y\left(p\right) < 0 \ & \Leftrightarrow \
  	\theta \in \left\langle
  		\frac{
  			3\pi
  		}{
  			2
  		};
  		2\pi
  	\right)
    \end{align}
    Zatem w celu udowodnienia spełnienia warunków \ref{eq:condition1} i \ref{eq:condition2} przez funkcję $\alpha\left(p\right)$ (\ref{eq:alpha1}) należy udowodnić, że funkcja $\gamma\left(\theta\right)$ (\ref{eq:gamma}) jest rosnąca (jej pochodna jest większa od zera) we wszystkich przedziałach opisanych w zależnościach
     \ref{eq:xytotheta1}, 
     \ref{eq:xytotheta2}, 
     \ref{eq:xytotheta3} oraz
     \ref{eq:xytotheta4}.
    
    Obliczenia dla tych czterech przedziałów przedstawiają się w następujący sposób. Dla każdego z przedziałów możliwe jest uproszczenie funkcji $d\left(p\right)$, tak aby pozbyć się modułu, przez co różniczkowanie funkcji $\gamma\left(\theta\right)$ staje się łatwiejsze.
 
    \newcommand*{\includesDirectory}{includes}
\newcommand*{\settingsDirectory}{\includesDirectory/settings}
\newcommand*{\tikzDirectory}{\includesDirectory/tikz}
\newcommand*{\proovesDirectory}{\includesDirectory/prooves}
\newcommand*{\oneDirectory}{\proovesDirectory/one}
\newcommand*{\listingsDirectory}{\includesDirectory/listings}


\documentclass[a4paper,12pt,oneside]{book}
\usepackage{geometry}
\usepackage{indentfirst}

\geometry{
    left=35mm,
    right=25mm,
    top=25mm,
    bottom=25mm
}

\renewcommand{\baselinestretch}{1.5}
\usepackage{amssymb}
\usepackage{amsmath}



\renewcommand{\theequation}{\Roman{section}.\arabic{equation}}
\usepackage[T1]{fontenc}
\usepackage[polish]{babel}
\usepackage{lmodern}
\usepackage{fancyhdr}
\pagestyle{fancy}
\fancyhf{}
\renewcommand{\headrulewidth}{0pt}
\fancyhead[R]{\thepage}

\fancypagestyle{plain}{
    \fancyhf{}
    \renewcommand{\headrulewidth}{0pt}
    \fancyhead[R]{\thepage}
}
\usepackage{titletoc}
\usepackage{hyperref}
\setcounter{tocdepth}{1}

\renewcommand\thechapter{Rozdział \Roman{chapter}}
\renewcommand\thesection{\arabic{section}.}
\renewcommand\thesubsection{\arabic{section}.\arabic{subsection}}
\contentsmargin{0.5cm}
\newlength\chapterlength
\settowidth\chapterlength{\hspace{2.5cm}}

\titlecontents{chapter}
    [\chapterlength] %5.3
    {\vspace{0.2cm}}
    {\contentslabel[\thecontentslabel]{\chapterlength}}%\thecontentslabel
    {\hspace*{-\chapterlength}}% unnumbered chapters
    {\titlerule*[1cm]{.}\contentspage}[\vspace{0.2cm}]%

\titlecontents{section}
    [\chapterlength] %5.3
    {\small}
    {\contentslabel[\thecontentslabel]{0.5cm}}
    {}
    {\titlerule*[0.5cm]{.}\contentspage}[]




\usepackage{titlesec}


\titleformat{\chapter}[block]
  {\normalfont\huge}{\thechapter\vspace{-13pt}\\}{0pt}{\LARGE}
\titlespacing*{\chapter}{0pt}{0pt}{0pt}

\titleformat{\section}[block]
  {\normalfont}
  {\makebox[0.5cm][l]{\thesection}}{10pt}{}
\titlespacing*{\section}{0pt}{0pt}{0pt}

\titleformat{\subsection}[block]
  {\normalfont}
  {\hspace{1cm}\makebox[0.5cm][l]{\thesubsection}}{10pt}{}
\titlespacing*{\subsection}{0pt}{0pt}{0pt}
\usepackage{listings}

\usepackage{xcolor}

\definecolor{codegreen}{rgb}{0,0.6,0}
\definecolor{codegray}{rgb}{0.5,0.5,0.5}
\definecolor{codepurple}{rgb}{0.58,0,0.82}
\definecolor{backcolour}{rgb}{0.95,0.95,0.92}


\lstdefinestyle{mystyle}{
    backgroundcolor=\color{backcolour},   
    commentstyle=\color{codegreen},
    keywordstyle=\color{magenta},
    numberstyle=\tiny\color{codegray},
    stringstyle=\color{codepurple},
    basicstyle=\ttfamily\footnotesize,
    breakatwhitespace=false,         
    breaklines=true,                 
    captionpos=b,                    
    keepspaces=true,                 
    numbers=left,                    
    numbersep=5pt,                  
    showspaces=false,                
    showstringspaces=false,
    showtabs=false,                  
    tabsize=2
}

\lstset{style=mystyle}

\newcommand{\scala}[1]{\lstinline[language=scala]{#1}}
\usepackage{caption}
\newcommand{\srcpl}{\\Źródło}
\newcommand{\mysrc}{{\srcpl}: opracowanie własne}
\usepackage[style=ieee]{biblatex}
\addbibresource{refs.bib}

\usepackage{
	tikz,
	tzplot
}
\usetikzlibrary{
	shapes,
	quotes,
	angles,
	calc
}

\tikzset {
    point/.style args = {reference: #1 x: #2 y: #3 label: #4 color: #5}{
        insert path = {
            \pgfextra{
                \node[
                	circle,
                	fill = #5,
                	minimum size = 1 cm,
                	label = {#4},
                	scale = 0.1
                	]
                	at (#2,#3)
                	(#1) 
                	{};
            }
        }
    }
}
\tikzset {
    ellipse_block/.style args = {reference: #1 text: #2}{
        insert path = {
            \pgfextra{
                \node
                	[
                		ellipse,
                		draw = black
                	]
                	at (#1)
                	(#1_block) 
                	{#2};
            }
        }
    }
}
\tikzset {
   rectangle_block/.style ={draw=black,rectangle}
}


\newcommand{\block}[2]{

  \node[rectangle_block,below] at (#1.north) (#1_block) {
  	\begin{minipage}{5.2cm}\begin{tiny}\begin{flushleft}\begin{spacing}{1.5}
			#2
	\end{spacing}\end{flushleft}\end{tiny}\vspace{-8mm}\end{minipage}
  };
  
  
  
  \node at (#1_block.south) (#1_south) {};
 
}

\newcommand{\centerblock}[2]{

  \node[rectangle_block] at (#1.center) (#1_block) {
  	\begin{minipage}{5.2cm}\begin{tiny}\begin{flushleft}\begin{spacing}{1.5}
			#2
	\end{spacing}\end{flushleft}\end{tiny}\vspace{-8mm}\end{minipage}
  };
  
  
  
  \node at (#1_block.south) (#1_south) {};
 
}

\newcommand{\smallblock}[2]{

  \node[rectangle_block,below] at (#1.north) (#1_block) {
  	\begin{minipage}{5cm}\begin{small}\begin{center}\begin{spacing}{1.5}
			#2
	\end{spacing}\end{center}\end{small}\vspace{-12mm}\end{minipage}
  };
  
  
  
  \node at (#1_block.south) (#1_south) {};
 
}

\newcommand{\chooseblock}[2]{

  \node[diamond,draw =black,below, aspect = 3] at (#1.north) (#1_block) {
  	\tiny{
  		#2
	}
  };
  
  \node[below right] at (#1_block.south) {T};
  
  
  \node at (#1_block.south) (#1_south) {};
 
}
%\newcommand{\newnode}[2]{
%	\node at (#1) (#2) {}

%}
\tikzset {
    pointAndLabel/.style args = {reference: #1 x: #2 y: #3}{
        insert path = {
            \pgfextra{
                \path[point = reference: #1 x: #2 y: #3 label: $#1$ color: black];
            }
        }
    }
}
\tikzset {
    randomVertices/.style ={
        insert path = {
            \pgfextra{
               \path[pointAndLabel = 
    				reference: 1 
    				x: 1.371503889373591
    				y: 0.09390715693458951
				];
				\path[pointAndLabel = 
    				reference: 2 
    				x: 8.47977671920242
    				y: 1.6031886319353594
				];
				\path[pointAndLabel = 
    				reference: 3 
    				x: 6.045551430098136
    				y: 0.8234401782232126
				];
				\path[pointAndLabel = 
    				reference: 4 
    				x: 1.0581168893921622
    				y: 7.196652534121291
				];
				\path[pointAndLabel = 
    				reference: 5 
    				x: 7.54893452550192
    				y: 6.229955423976583
				];
				\path[pointAndLabel = 
    				reference: 6 
    				x: 3.452073817126815
    				y: 0.42802148287650943
				];
				\path[pointAndLabel = 
    				reference: 7 
    				x: 1.9234285748582658
    				y: 3.8128351776520875
				];
				\path[pointAndLabel = 
    				reference: 8 
    				x: 1.4291333394587036
    				y: 2.298954766988656
				];
				\path[pointAndLabel = 
    				reference: 9 
    				x: 4.158998560109078
    				y: 1.1819955486088674
				];
				\path[pointAndLabel = 
    				reference: 10 
    				x: 9.607809179355511
    				y: 0.9208848746431597
				];
				\path[pointAndLabel = 
    				reference: 11 
    				x: 4.505748586095314
    				y: 4.275899288243908
				];
				\path[pointAndLabel = 
    				reference: 12 
    				x: 4.335033269031589
    				y: 6.636185965600352
				];
				\path[pointAndLabel = 
    				reference: 13 
    				x: 8.265341405959747
    				y: 8.370333801938221
				];
				\path[pointAndLabel = 
    				reference: 14 
    				x: 3.954303586943823
    				y: 3.5057031654014024
				];
				\path[pointAndLabel = 
    				reference: 15 
    				x: 1.9259642615328465
    				y: 4.637048240596112
				];
				\path[pointAndLabel = 
    				reference: 16 
    				x: 9.450732288466309
    				y: 5.239112331048348
				];
				\path[pointAndLabel = 
    				reference: 17 
    				x: 9.452180434267264
    				y: 9.474236735440195
				];
				\path[pointAndLabel = 
    				reference: 18 
    				x: 2.062657121585195
    				y: 7.013735646100131
				];
				\path[pointAndLabel = 
    				reference: 19 
    				x: 1.3007739087073178
    				y: 8.606241672641316
				];
				\path[pointAndLabel = 
    				reference: 20 
    				x: 0.8654915304555744
    				y: 3.3090677225981358
				];
            }
        }
    }
}
\pgfarrowsdeclare{arcs}{arcs}{...}{
    \pgfsetdash{}{0pt}
    \pgfsetroundjoin
    \pgfsetroundcap
    \pgfpathmoveto{\pgfpoint{-3pt}{3pt}}
    \pgfpatharc{180}{270}{3pt}
    \pgfpatharc{90}{180}{3pt}
    \pgfusepathqstroke
}

\begin{document}

\newcommand{\alignspace}{\vspace{- 3pt}}
\renewcommand{\thelstlisting}{\arabic{lstlisting}}
\counterwithout{figure}{chapter}
\counterwithout{lstlisting}{chapter}
\counterwithout{equation}{chapter}
\setlength{\abovedisplayskip}{0ex}
\setlength{\belowdisplayskip}{0ex}
\setlength{\abovedisplayshortskip}{0ex}
\setlength{\belowdisplayshortskip}{0ex}
\setlength{\jot}{1.5ex}


    \thispagestyle{empty}
    \begin{adjustwidth}{-1cm}{0cm}
    \begin{center}
    		\begin{large}
\textbf{UNIWERSYTET GDAŃSKI}

\vspace{.2 cm}
\textbf{WYDZIAŁ MATEMATYKI, FIZYKI I INFORMATYKI}
\end{large}

\vspace{3 cm}
\textbf{Marcin Belicki}

\textbf{numer albumu: 273417}
    \end{center}
    
\vspace{3 cm}
\textit{Kierunek studiów: Informatyka}

\vspace{3 cm}
\begin{center}
	\begin{large}
		\textbf{OTOCZKI WYPUKŁE}
	\end{large}
\end{center}

\vspace{3 cm}
\begin{flushright}
Praca magisterska

wykonana

pod kierunkiem

dr inż. Arkadiusz Mirakowski
\end{flushright}
\vspace{2 cm}
\begin{center}
\textbf{Gdańsk 2023}
\end{center}

    \end{adjustwidth}
    
    \tableofcontents	

  
    \chapterwithout{Wstęp}        

    \chapter{Otoczka wypukła na płaszczyźnie}
    Otoczka wypukła zbioru punktów w swojej najbardziej podstawowej postaci jest wielokątem wypukłym obejmującym wszystkie punkty ze zbioru punktów leżących na płaszczyźnie w taki sposób, aby wielokąt ten miał jak najmniejsze pole.    
    \begin{figure}[h!]
        	\begin{center}
    			\begin{tikzpicture}
    	
    \path[randomVertices];
    	
    \draw (1.center)--
          (10.center)--
    	  (17.center)--
    	  (19.center)--
    	  (4.center)--
    	  (20.center)--
    	  cycle;
    
\end{tikzpicture}  
    			\caption{Otoczka wypukła na płaszczyźnie\mysrc}\label{fig:example_hull}
    	\end{center}
    	\vspace{-0.8cm}
    \end{figure}
    
    Dobrą reprezentacją otoczki wypukłej w świecie fizycznym może być grupa gwoździ przybita do płaskiej powierzchni i następnie opleciona ciasno sznurkiem. Gwoździe stykające się ze sznurkiem stanowić będą wierzchołki otoczki wypukłej tej grupy gwoździ.
        \section{Otoczka wypukła zbioru punktów}
        W celu wyznaczenia otoczki wypukłej dla najbardziej ogólnego przypadku --- nieuporządkowanego zbioru punktów na płaszczyźnie, możemy wykorzystać dwa najbardziej popularne algorytmy algorytm Jarvisa oraz algorytm Grahama.
        
        W wykorzystywanych algorytmach istotnym elementem jest sortowanie punktów względem wartości, które zostały przedstawione na rysunku \ref{fig:axis}. W zależności od tego, w jakiej postaci określone będą dane punkty, należy dokonać odpowiednich obliczeń.
        \begin{figure}[h!]
        	\begin{center}
    			\begin{tikzpicture}
    	
   \draw [-arcs] 
   (-2.5,0) --
   (2.5,0)  
   node[above] {$X$};
   
   \draw [-arcs] 
   (0,-2.5) --
   (0,2.5)   
   node[left] {$Y$} ;
   
   
   \path[point = reference: p x: -1.5 y: 1.5 label: $p$ color: black];
   
   \draw [dashed] 
   (0,0) -- node [above right = 0 mm] {\small{$R$}}
   (p.center);
   
   \draw [dashed] 
   (p.center) --++
   (1.5,0) 
   node[right] {\footnotesize{$y\left(p\right) = \left|R\right|\cdot \sin \theta$}} ;
   
   \draw [dashed] 
   (p.center) --++
   (0,-1.5) 
   node[below] {\footnotesize{$x\left(p\right) = \left|R\right|\cdot \cos \theta$}};
   
   
   \tzanglemark[-arcs](2.5,0)(0,0)(p){\small{$\theta$}}[pos=1.4](.6)
  

    
\end{tikzpicture} 
    			\caption{Punkt na układzie współrzędnych\mysrc}\label{fig:axis}
    	\end{center}
    	\end{figure}\vspace{-8 mm}\\
    	Gdzie poszczególne symbole oznaczają:\\
    	\begin{tabular}{rcl}
    	$p$ & -- & rozpatrywany punkt\\
    	$x\left( p \right)$ & -- & odcięta punktu $p$\\
    	$y\left( p \right)$ & -- & rzędna punktu $p$\\
    	$R$ & -- & długość wektora wodzącego punktu $p$\\
    	$\theta$ & -- & kąt nachylenia wektora wodzącego punktu $p$ do osi $OX$
    	\end{tabular}\\
    	
    	Istotną komplikację z punktu widzenia obliczeń  w algorytmach może stanowić wyznaczenie wartości kąta $\theta$ (ze względu na potrzebę wykorzystania funkcji trygonometrycznych), którego dokładną wartość można wyliczyć za pomocą wzoru \ref{eq:theta}.\\
        \begin{align}\label{eq:theta}
        	\theta\left(p\right) = \left\{
        		\begin{array}{lclcr}
        			\operatorname {arctg} \frac{y\left(p\right)}{x\left(p\right)}& \text{dla} & x\left(p\right) > 0; \\
        			\operatorname {arctg} \frac{y\left(p\right)}{x\left(p\right)} + \pi& \text{dla} & x\left(p\right) < 0 & \wedge & y\left(p\right) \geqslant 0; \\
        			\operatorname {arctg} \frac{y\left(p\right)}{x\left(p\right)} - \pi& \text{dla} & x\left(p\right) < 0 & \wedge & y\left(p\right) < 0; \\
        			\frac{\pi}{2}& \text{dla} & x\left(p\right) = 0 & \wedge & y\left(p\right) < 0; \\
        			-\frac{\pi}{2}& \text{dla} & x\left(p\right) = 0 & \wedge & y\left(p\right) > 0. \\
        		\end{array}
        	\right.
        \end{align}\\
         Należy jednak zauważyć, że do celów sortowania wystarczy zastosować funkcję $\alpha\left(p\right)$, taką, że dla dowolnej pary punktów $p_1, p_2$ spełnione będą warunki \ref{eq:condition1} oraz \ref{eq:condition2}. 
    	\begin{align}\label{eq:condition1}
    		\alpha\left(p_1\right) <  \alpha\left(p_2\right)
    		\Leftrightarrow \theta\left(p_1\right) <  \theta\left(p_2\right)\\
    		\label{eq:condition2}
    		\alpha\left(p_1\right) =  \alpha\left(p_2\right)
    		\Leftrightarrow \theta\left(p_1\right) = \theta\left(p_2\right)
    	\end{align}
		Przykładowa funkcja $\alpha\left(p\right)$ zachowująca spełniająca warunki \ref{eq:condition1} i \ref{eq:condition2} została przedstawiona za pomocą wzoru \ref{eq:alpha1}.\\
   		\begin{align}\label{eq:alpha1}
        	\alpha\left( p\right) = \left\{
        		\begin{array}{lclcr}
        			\frac{y\left(p\right)}{d\left(p\right)} & \text{dla} &
        			x\left(p\right) \geqslant 0 & \wedge & y\left(p\right) \geqslant 0; \\
        			2 - \frac{y\left(p\right)}{d\left(p\right)} & \text{dla} &
        			x\left(p\right) < 0 & \wedge & y\left(p\right) \geqslant 0; \\
        			2 + \frac{\left|y\left(p\right)\right|}{d\left(p\right)} & \text{dla} &
        			x\left(p\right) < 0 & \wedge & y\left(p\right) < 0; \\
        			4 - \frac{\left|y\left(p\right)\right|}{d\left(p\right)} & \text{dla} &
        			x\left(p\right) \geqslant 0 & \wedge & y\left(p\right) < 0.
        		\end{array}
        	\right.
        \end{align}\\
    Gdzie funkcja $d\left(p\right)$ określona jest zgodnie ze wzorem \ref{eq:dp}.
    \begin{align}\label{eq:dp}
        	d\left(p\right) = \left|x\left(p\right)\right| + \left|y\left(p\right)\right|
    \end{align}
    
     W celu udowodnienia, że funkcja $\alpha\left(p\right)$ spełnia warunki \ref{eq:condition1} i \ref{eq:condition2}, musimy wykazać, że funkcja $\gamma\left(\theta\right)$ (przedstawiona we wzorze \ref{eq:gamma}), która stanowi przekształcenie funkcji $\alpha\left(p\right)$ takie, że jest w pełni zależna od wartości $\theta$, a wartości $x\left(p\left(\theta\right)\right)$ oraz $y\left(p\left(\theta\right)\right)$ są określone zgodnie ze wzorem \ref{eq:ptheta}, jest rosnąca w każdym swoim przedziale.
     \begin{align}\label{eq:gamma}
        	\gamma\left(\theta\right) = \alpha\left(p\left(\theta\right)\right)
     \end{align}
     \begin{align}\label{eq:ptheta}
        	p\left(\theta\right) = \left( 
        		\left|R\right| \cdot \cos \theta , 
        		\left|R\right| \cdot \sin \theta
        	\right)
     \end{align}
     
    

     Należy zauważyć, że poszczególne przypadki opisane we wzorze \ref{eq:alpha1} dzielą układ współrzędnych na cztery ćwiartki. Zatem z rysunku \ref{fig:axis} wynikają zależności
     \ref{eq:xytotheta1}, 
     \ref{eq:xytotheta2}, 
     \ref{eq:xytotheta3} oraz
     \ref{eq:xytotheta4}.
     \begin{align}\label{eq:xytotheta1}
    		x\left(p\right) \geqslant 0 \ \wedge \  y\left(p\right) \geqslant 0  \ & \Leftrightarrow \ \theta \in \left\langle 0;\frac{\pi}{2}\right\rangle \\
    		\label{eq:xytotheta2}
    		x\left(p\right) < 0 \ \wedge \ y\left(p\right) \geqslant 0 \ & \Leftrightarrow \
  		\theta \in \left( \frac{\pi}{2};\pi\right\rangle\\
  		\label{eq:xytotheta3}
    		x\left(p\right) < 0 \ \wedge \ y\left(p\right) < 0 \ & \Leftrightarrow \
  		\theta \in \left( \pi;\frac{3\pi}{2}\right)\\
  		\label{eq:xytotheta4}
  		x\left(p\right) \geqslant 0 \ \wedge \ y\left(p\right) < 0 \ & \Leftrightarrow \
  	\theta \in \left\langle
  		\frac{
  			3\pi
  		}{
  			2
  		};
  		2\pi
  	\right)
    \end{align}
    Zatem w celu udowodnienia spełnienia warunków \ref{eq:condition1} i \ref{eq:condition2} przez funkcję $\alpha\left(p\right)$ (\ref{eq:alpha1}) należy udowodnić, że funkcja $\gamma\left(\theta\right)$ (\ref{eq:gamma}) jest rosnąca (jej pochodna jest większa od zera) we wszystkich przedziałach opisanych w zależnościach
     \ref{eq:xytotheta1}, 
     \ref{eq:xytotheta2}, 
     \ref{eq:xytotheta3} oraz
     \ref{eq:xytotheta4}.
    
    Obliczenia dla tych czterech przedziałów przedstawiają się w następujący sposób. Dla każdego z przedziałów możliwe jest uproszczenie funkcji $d\left(p\right)$, tak aby pozbyć się modułu, przez co różniczkowanie funkcji $\gamma\left(\theta\right)$ staje się łatwiejsze.
 
    \newcommand*{\includesDirectory}{includes}
\newcommand*{\settingsDirectory}{\includesDirectory/settings}
\newcommand*{\tikzDirectory}{\includesDirectory/tikz}
\newcommand*{\proovesDirectory}{\includesDirectory/prooves}
\newcommand*{\oneDirectory}{\proovesDirectory/one}
\newcommand*{\listingsDirectory}{\includesDirectory/listings}


\input{\settingsDirectory/geometry}
\input{\settingsDirectory/math}
\input{\settingsDirectory/language}
\input{\settingsDirectory/page_number}
\input{\settingsDirectory/table_of_contents}
\input{\settingsDirectory/chapters}
\input{\settingsDirectory/listings}
\input{\settingsDirectory/captions}
\input{\settingsDirectory/biblatex}

\input{\tikzDirectory/libraries}
\input{\tikzDirectory/point}
\input{\tikzDirectory/ellipse_block}
\input{\tikzDirectory/rectangle_block}
\input{\tikzDirectory/pointAndLabel}
\input{\tikzDirectory/randomVertices}
\input{\tikzDirectory/arcs}

\begin{document}

\newcommand{\alignspace}{\vspace{- 3pt}}
\renewcommand{\thelstlisting}{\arabic{lstlisting}}
\counterwithout{figure}{chapter}
\counterwithout{lstlisting}{chapter}
\counterwithout{equation}{chapter}
\setlength{\abovedisplayskip}{0ex}
\setlength{\belowdisplayskip}{0ex}
\setlength{\abovedisplayshortskip}{0ex}
\setlength{\belowdisplayshortskip}{0ex}
\setlength{\jot}{1.5ex}


    \thispagestyle{empty}
    \begin{adjustwidth}{-1cm}{0cm}
    \begin{center}
    		\begin{large}
\textbf{UNIWERSYTET GDAŃSKI}

\vspace{.2 cm}
\textbf{WYDZIAŁ MATEMATYKI, FIZYKI I INFORMATYKI}
\end{large}

\vspace{3 cm}
\textbf{Marcin Belicki}

\textbf{numer albumu: 273417}
    \end{center}
    
\vspace{3 cm}
\textit{Kierunek studiów: Informatyka}

\vspace{3 cm}
\begin{center}
	\begin{large}
		\textbf{OTOCZKI WYPUKŁE}
	\end{large}
\end{center}

\vspace{3 cm}
\begin{flushright}
Praca magisterska

wykonana

pod kierunkiem

dr inż. Arkadiusz Mirakowski
\end{flushright}
\vspace{2 cm}
\begin{center}
\textbf{Gdańsk 2023}
\end{center}

    \end{adjustwidth}
    
    \tableofcontents	

  
    \chapterwithout{Wstęp}        

    \chapter{Otoczka wypukła na płaszczyźnie}
    Otoczka wypukła zbioru punktów w swojej najbardziej podstawowej postaci jest wielokątem wypukłym obejmującym wszystkie punkty ze zbioru punktów leżących na płaszczyźnie w taki sposób, aby wielokąt ten miał jak najmniejsze pole.    
    \begin{figure}[h!]
        	\begin{center}
    			\input{\tikzDirectory/hull.tikz}
    			\caption{Otoczka wypukła na płaszczyźnie\mysrc}\label{fig:example_hull}
    	\end{center}
    	\vspace{-0.8cm}
    \end{figure}
    
    Dobrą reprezentacją otoczki wypukłej w świecie fizycznym może być grupa gwoździ przybita do płaskiej powierzchni i następnie opleciona ciasno sznurkiem. Gwoździe stykające się ze sznurkiem stanowić będą wierzchołki otoczki wypukłej tej grupy gwoździ.
        \section{Otoczka wypukła zbioru punktów}
        W celu wyznaczenia otoczki wypukłej dla najbardziej ogólnego przypadku --- nieuporządkowanego zbioru punktów na płaszczyźnie, możemy wykorzystać dwa najbardziej popularne algorytmy algorytm Jarvisa oraz algorytm Grahama.
        
        W wykorzystywanych algorytmach istotnym elementem jest sortowanie punktów względem wartości, które zostały przedstawione na rysunku \ref{fig:axis}. W zależności od tego, w jakiej postaci określone będą dane punkty, należy dokonać odpowiednich obliczeń.
        \begin{figure}[h!]
        	\begin{center}
    			\input{\tikzDirectory/axis.tikz}
    			\caption{Punkt na układzie współrzędnych\mysrc}\label{fig:axis}
    	\end{center}
    	\end{figure}\vspace{-8 mm}\\
    	Gdzie poszczególne symbole oznaczają:\\
    	\begin{tabular}{rcl}
    	$p$ & -- & rozpatrywany punkt\\
    	$x\left( p \right)$ & -- & odcięta punktu $p$\\
    	$y\left( p \right)$ & -- & rzędna punktu $p$\\
    	$R$ & -- & długość wektora wodzącego punktu $p$\\
    	$\theta$ & -- & kąt nachylenia wektora wodzącego punktu $p$ do osi $OX$
    	\end{tabular}\\
    	
    	Istotną komplikację z punktu widzenia obliczeń  w algorytmach może stanowić wyznaczenie wartości kąta $\theta$ (ze względu na potrzebę wykorzystania funkcji trygonometrycznych), którego dokładną wartość można wyliczyć za pomocą wzoru \ref{eq:theta}.\\
        \begin{align}\label{eq:theta}
        	\theta\left(p\right) = \left\{
        		\begin{array}{lclcr}
        			\operatorname {arctg} \frac{y\left(p\right)}{x\left(p\right)}& \text{dla} & x\left(p\right) > 0; \\
        			\operatorname {arctg} \frac{y\left(p\right)}{x\left(p\right)} + \pi& \text{dla} & x\left(p\right) < 0 & \wedge & y\left(p\right) \geqslant 0; \\
        			\operatorname {arctg} \frac{y\left(p\right)}{x\left(p\right)} - \pi& \text{dla} & x\left(p\right) < 0 & \wedge & y\left(p\right) < 0; \\
        			\frac{\pi}{2}& \text{dla} & x\left(p\right) = 0 & \wedge & y\left(p\right) < 0; \\
        			-\frac{\pi}{2}& \text{dla} & x\left(p\right) = 0 & \wedge & y\left(p\right) > 0. \\
        		\end{array}
        	\right.
        \end{align}\\
         Należy jednak zauważyć, że do celów sortowania wystarczy zastosować funkcję $\alpha\left(p\right)$, taką, że dla dowolnej pary punktów $p_1, p_2$ spełnione będą warunki \ref{eq:condition1} oraz \ref{eq:condition2}. 
    	\begin{align}\label{eq:condition1}
    		\alpha\left(p_1\right) <  \alpha\left(p_2\right)
    		\Leftrightarrow \theta\left(p_1\right) <  \theta\left(p_2\right)\\
    		\label{eq:condition2}
    		\alpha\left(p_1\right) =  \alpha\left(p_2\right)
    		\Leftrightarrow \theta\left(p_1\right) = \theta\left(p_2\right)
    	\end{align}
		Przykładowa funkcja $\alpha\left(p\right)$ zachowująca spełniająca warunki \ref{eq:condition1} i \ref{eq:condition2} została przedstawiona za pomocą wzoru \ref{eq:alpha1}.\\
   		\begin{align}\label{eq:alpha1}
        	\alpha\left( p\right) = \left\{
        		\begin{array}{lclcr}
        			\frac{y\left(p\right)}{d\left(p\right)} & \text{dla} &
        			x\left(p\right) \geqslant 0 & \wedge & y\left(p\right) \geqslant 0; \\
        			2 - \frac{y\left(p\right)}{d\left(p\right)} & \text{dla} &
        			x\left(p\right) < 0 & \wedge & y\left(p\right) \geqslant 0; \\
        			2 + \frac{\left|y\left(p\right)\right|}{d\left(p\right)} & \text{dla} &
        			x\left(p\right) < 0 & \wedge & y\left(p\right) < 0; \\
        			4 - \frac{\left|y\left(p\right)\right|}{d\left(p\right)} & \text{dla} &
        			x\left(p\right) \geqslant 0 & \wedge & y\left(p\right) < 0.
        		\end{array}
        	\right.
        \end{align}\\
    Gdzie funkcja $d\left(p\right)$ określona jest zgodnie ze wzorem \ref{eq:dp}.
    \begin{align}\label{eq:dp}
        	d\left(p\right) = \left|x\left(p\right)\right| + \left|y\left(p\right)\right|
    \end{align}
    
     W celu udowodnienia, że funkcja $\alpha\left(p\right)$ spełnia warunki \ref{eq:condition1} i \ref{eq:condition2}, musimy wykazać, że funkcja $\gamma\left(\theta\right)$ (przedstawiona we wzorze \ref{eq:gamma}), która stanowi przekształcenie funkcji $\alpha\left(p\right)$ takie, że jest w pełni zależna od wartości $\theta$, a wartości $x\left(p\left(\theta\right)\right)$ oraz $y\left(p\left(\theta\right)\right)$ są określone zgodnie ze wzorem \ref{eq:ptheta}, jest rosnąca w każdym swoim przedziale.
     \begin{align}\label{eq:gamma}
        	\gamma\left(\theta\right) = \alpha\left(p\left(\theta\right)\right)
     \end{align}
     \begin{align}\label{eq:ptheta}
        	p\left(\theta\right) = \left( 
        		\left|R\right| \cdot \cos \theta , 
        		\left|R\right| \cdot \sin \theta
        	\right)
     \end{align}
     
    

     Należy zauważyć, że poszczególne przypadki opisane we wzorze \ref{eq:alpha1} dzielą układ współrzędnych na cztery ćwiartki. Zatem z rysunku \ref{fig:axis} wynikają zależności
     \ref{eq:xytotheta1}, 
     \ref{eq:xytotheta2}, 
     \ref{eq:xytotheta3} oraz
     \ref{eq:xytotheta4}.
     \begin{align}\label{eq:xytotheta1}
    		x\left(p\right) \geqslant 0 \ \wedge \  y\left(p\right) \geqslant 0  \ & \Leftrightarrow \ \theta \in \left\langle 0;\frac{\pi}{2}\right\rangle \\
    		\label{eq:xytotheta2}
    		x\left(p\right) < 0 \ \wedge \ y\left(p\right) \geqslant 0 \ & \Leftrightarrow \
  		\theta \in \left( \frac{\pi}{2};\pi\right\rangle\\
  		\label{eq:xytotheta3}
    		x\left(p\right) < 0 \ \wedge \ y\left(p\right) < 0 \ & \Leftrightarrow \
  		\theta \in \left( \pi;\frac{3\pi}{2}\right)\\
  		\label{eq:xytotheta4}
  		x\left(p\right) \geqslant 0 \ \wedge \ y\left(p\right) < 0 \ & \Leftrightarrow \
  	\theta \in \left\langle
  		\frac{
  			3\pi
  		}{
  			2
  		};
  		2\pi
  	\right)
    \end{align}
    Zatem w celu udowodnienia spełnienia warunków \ref{eq:condition1} i \ref{eq:condition2} przez funkcję $\alpha\left(p\right)$ (\ref{eq:alpha1}) należy udowodnić, że funkcja $\gamma\left(\theta\right)$ (\ref{eq:gamma}) jest rosnąca (jej pochodna jest większa od zera) we wszystkich przedziałach opisanych w zależnościach
     \ref{eq:xytotheta1}, 
     \ref{eq:xytotheta2}, 
     \ref{eq:xytotheta3} oraz
     \ref{eq:xytotheta4}.
    
    Obliczenia dla tych czterech przedziałów przedstawiają się w następujący sposób. Dla każdego z przedziałów możliwe jest uproszczenie funkcji $d\left(p\right)$, tak aby pozbyć się modułu, przez co różniczkowanie funkcji $\gamma\left(\theta\right)$ staje się łatwiejsze.
 
    \input{\oneDirectory/main}
    
    Z powyższych obliczeń wynika, że we wszystkich przedziałach z zależności
    \ref{eq:xytotheta1}, 
     \ref{eq:xytotheta2}, 
     \ref{eq:xytotheta3} oraz
     \ref{eq:xytotheta4} funkcja $\gamma\left(\theta\right)$ ma dodatnią pochodną, a tym samym jest rosnąca we wszystkich przedziałach, oraz każda z wartości osiąganych w przedziale jest większa od każdej wartości z poprzedniego przedziału, a tym samym funkcja $\alpha\left(p\right)$ spełnia warunki \ref{eq:condition1} i \ref{eq:condition2}.
     
     Postać funkcji $\gamma\left(\theta\right)$ może być opisana wzorem \ref{eq:gammaeq} a wykres jej przebiegu, przedstawiony na rysunku \ref{fig:gamma} wyraźnie potwierdza jej rosnącą monotoniczność we wszystkich przedziałach.
        \begin{align}\label{eq:gammaeq}
        	\gamma\left(
        		\theta
        	\right) = 
        		\left\{
        			\begin{array}{lcl}
        				\frac{
        					\sin\theta
        				}{
							\sin\theta + \cos\theta        			
        				} & 
        					\text{dla} &
        					\theta \in \left\langle
        						0;
        						\frac{
        							\pi
        						}{
									2        						
        						}
        					\right\rangle\\
        				2
        				- \frac{
        					\sin\theta
        				}{
							\sin\theta - \cos\theta        			
        				} & 
        					\text{dla} &
        					\theta \in \left(
        						\frac{
        							\pi
        						}{
									2        						
        						};
        						\pi
        					\right\rangle\\
        				2
        				+ \frac{
        					\sin\theta
        				}{
							\sin\theta + \cos\theta        			
        				} & 
        					\text{dla} &
        					\theta \in \left(
        						\pi;
        						\frac{
        							3\pi
        						}{
									2        						
        						}
        					\right)\\
        				4
        				- \frac{
        					\sin\theta
        				}{
							\sin\theta + \cos\theta        			
        				} & 
        					\text{dla} &
        					\theta \in \left\langle
        						\frac{
        							3\pi
        						}{
									2        						
        						};
        						2\pi
        					\right)\\
        			\end{array}
        		\right.      		
        \end{align}
        \begin{figure}[h!]
        	\begin{center}
    			\input{\tikzDirectory/gamma.tikz}
    			\caption{Wykres funkcji $\gamma\left(\theta\right)$\mysrc}\label{fig:gamma}
    	\end{center}
    	\end{figure}\vspace{-8 mm}\\
       
              
        \subsection{Algorytm Grahama}
        \null
        \begin{figure}[h!]
    			\input{\tikzDirectory/graham.tikz}
    			\caption{Algorytm Grahama}\label{fig:graham}
    	\end{figure}
       \newpage
        \subsection{Algorytm Jarvisa}
        
        
        \section{Otoczka wypukła wielokąta prostego}
        \cite{convexhullsimplepolygon}
        \section{Redukcja zbioru punktów do wielokąta prostego}
    \chapter{Zastosowania} 
%    as first chapter
    Istotnym, jeśli nie najistotniejszym zagadnieniem dotyczącym otoczek wypukłych są ich zastosowania. Już od kilkudziesięciu lat problem ten temat znajduje swoje użycie w wielu dziedzinach kombinatoryki i informatyki. Dla przykładu problem sortowania elementów liczbowych listy można sprowadzić do problemu znalezienia otoczki wypukłej. W związku z tym rozwiązanie tego problemu może pomóc w rozwiązaniu innych problemów w bardziej symboliczny i graficzny sposób.
        \section{Generalizacja kartograficzna}
        W celu jak najbardziej informatywnego i niezłożonego przedstawienia danych geograficznych w sposób graficzny potrzebne jest zastosowanie algorytmu generalizującego informacje. Taka generalizacja w kartografii zaczęła być coraz bardziej potrzebna, kiedy w dobie cyfrowego przetwarzania danych mapy zaczęły być przechowywane cyfrowo. Przed tym generalizacją zajmowali się manualnie ludzie, którzy wyposażeni w ludzkie zmysły wykrywania wzorów i kształtów byli w stanie z łatwością narysować obrys zbioru drzew na płaszczyźnie i opisać go jako teren leśny. W przypadku przechowywania danych geograficznych w postaci cyfrowej zaistniała potrzeba wykonywania tej samej operacji w sposób zalgorytmizowany. \cite{cartography}
        
        Do celu generalizacji można by wykorzystać otoczkę wypukłą w sposób bezpośredni \cite{LAURINI2017139}, jednak taki obrys terenu może zawierać w sobie przestrzenie pozbawione obiektów, co byłoby niepomocne dla ludzkiej interpretacji przedstawianego mu obszaru. 
        
        Należy też zwrócić uwagę, że otoczka wypukła wielokąta prostego zawiera w sobie wszystkie punkty należące do wewnętrznej powierzchni tego wielokąta. \cite{convexhullsimplepolygon} W związku z tym jeśli udowodni się, że dany punkt nie należy do wnętrza otoczki wypukłej będzie to implikowało, że punkt ten nie należy do wnętrza wielokąta prostego. Taka wiedza okazuje się być przydatna jeśli uwzględnimy dodatkowo fakt, że określanie czy punkt leży wewnątrz wielokąta wypukłego jest dużo łatwiejsze niż dla pozostałych wielokątów. \cite{heckbert1994graphics}
        
        Dla przykładu rozpatrując obrys granic Polski (Rysunek \ref{fig:poland}; źródło współrzędnych obrysu: \cite{datahubCountryPolygons}; współrzędne przedstawione zgodnie z projekcją EPSG:2180 \cite{epsg2180} przy użyciu biblioteki GeoTools \cite{geotools}) można zauważyć, że jego kształt jest bardzo skomplikowany. W istocie zawiera on 1336 krawędzi.
        \begin{figure}[h!]
        	\begin{center}
    			\input{\tikzDirectory/poland.tikz}
    			\caption{Obrys granic Polski\mysrc}\label{fig:poland}
    	    \end{center}
    	\end{figure}
    	
    	Natomiast otoczka wypukła obrysu granic Polski zawiera tylko 28 krawędzi i została przedstawiona na rysunku \ref{fig:poland_convex_hull}.
    	  \begin{figure}[h!]
        	\begin{center}
    			\input{\tikzDirectory/poland_convex_hull.tikz}
    			\caption{Otoczka wypukła obrysu granic Polski\mysrc}\label{fig:poland_convex_hull}
    	    \end{center}
    	\end{figure}
    	
    	W związku z tym, że złożoność obliczeniowa algorytmu sprawdzającego czy punkt należy do wnętrza danego wielokąta jest liniowa \cite{complex}, możliwe jest dużo szybsze wykluczenie punktów znajdujących się poza otoczką. 
		\section{Grafika komputerowa}
		Obiekty wykorzystywane w grafice komputerowej często mogą charakteryzować się skomplikowanymi kształtami. Im bardziej skomplikowany kształt, tym więcej mocy obliczeniowej potrzeba w celu wykonania danej operacji na tym kształcie. W niektórych przypadkach do uproszczenia graficznego danego obiektu używa się otoczki wypukłej jego kształtu. Ze względu na fakt, iż otoczka wypukła wielokąta zawsze będzie miała liczbę wierzchołków mniejszą lub równą liczbie wierzchołków samego wielokąta, powstała otoczka może posłużyć do wykonania mniejszej liczby obliczeń przy wykonywaniu operacji na danym obiekcie.
		\section{Detekcja obiektów}
		Proces analizowania obrazu przez ludzkie organy jest wyjątkowo złożony złożony \cite{BIEDERMAN198529}. W celu imitacji tego procesu w sposób algorytmiczny, zostały przeprowadzone liczne badania naukowe \cite{detection}. Dzięki temu zaproponowano wiele podejść rozwiązania tego problemu. Istotnym aspektem ze względów praktycznych tych pomysłów jest złożoność obliczeniowa ich implementacji.
		
		Jednym z działań wykonywanych przez ludzkie zdolności poznawcze po zarejestrowaniu obrazu jest wybranie miejsca w obrazie o największym znaczeniu (z ludzkiego punktu widzenia) \cite{s23146450}. Niektóre z rozwiązań tego problemu w informatyce opierały się nie tylko na samych danych pochodzących z obrazu, ale również tworząc modele z uwzględnieniem danych statycznych z badań dotyczących ludzkich umiejętności poznawczych \cite{Borji_Cheng_Hou_Jiang_Li_2019}. Jednak spora część z nich opiera się na przetwarzaniu samych danych pozyskanych z obrazów, bez potrzeby dostarczania danych statystycznych. Aby komputer był w stanie wykonać tego typu operację potrzebny jest szereg złożonych obliczeniowo operacji \cite{detection, Borji_Cheng_Hou_Jiang_Li_2019}. 
		
		W celu zmniejszenia kosztu obliczeniowego tego typu procesu można wykonać preprocesowanie danych wejściowych. Jednym z nich może być wykorzystanie algorytmu wyznaczającego otoczkę wypukłą tak zwanych punktów kluczowych \cite{detection}. W celu znalezienia lokalizacji punktów kluczowych można skorzystać z wyznaczania lokalnej energii piksela za pomocą wzoru (\ref{eq:harris}) na wielopoziomową energię Harrisa \cite{detection, harris}. 
		\begin{align}\label{eq:harris}  
			\textbf{E}_{WPH}\left(p\right) = \sum_{l=0}^{L}\textbf{R}\left(l\right)\left(p\right)	 		
        \end{align}\\[-5 pt]
        Gdzie poszczególne symbole oznaczają:\\[10 pt]
       \begin{tabular}{rcl}
    	$p$ & -- & rozpatrywany piksel obrazu\\
		$\textbf{R}\left(l\right)\left(p\right)$ & -- &  \begin{minipage}[t]{123 mm} energia piksela $p$ wyznaczona ze wzoru Harrisa (\ref{eq:harris_single}) dla $l$ - tego poziomu obrazu \end{minipage} \\[20 pt] 
		$L$ & -- & \begin{minipage}[t]{123 mm}
		liczba poziomów dla których ma być wyznaczona suma energii, przyjęte jako $L=5$ \cite{detection}
		\end{minipage}
		
    	\end{tabular}\\[15 pt]
    	W celu wyznaczenia wielopoziomowej Harrisa dla danego piksela $p$ musimy znać też wzór (\ref{eq:harris_single}) na energię piksela $p$ dla pojedynczego poziomu $l$.
    	\begin{align}\label{eq:harris_single}  
		\textbf{R}\left(l\right)\left(p\right) =
		\left|\textbf{M}\left(l\right)\left(p\right)\right| + k \cdot \mathrm{tr}\left(\textbf{M}\left(l\right)\left(p\right)\right)
        \end{align}
Gdzie poszczególne symbole oznaczają:\\[10 pt]
        \begin{tabular}{rcl}
    	$k$ & -- & współczynnik skalarny\\
    	$\textbf{M}\left(l\right)\left(p\right)$ & -- & 
    	\begin{minipage}[t]{121 mm}
    	macierz autokorelacji dla danego piksela $p$ i poziomu obrazu $l$ wyznaczana ze wzoru \ref{eq:autocorelate} \cite{harrispptx, harris, detection}
    	\end{minipage}\\[20 pt]
    	$\mathrm{tr}\left(\textbf{A}\right)$ & -- & 
    	\begin{minipage}[t]{121 mm}
    	ślad macierzy $\textbf{A}$, czyli suma elementów należących do głównej przekątnej macierzy $\textbf{A}$, obliczany według wzoru \cite{matrixtrace}
    	\end{minipage}
    	\end{tabular}\\[10 pt]
    	\begin{align}\label{eq:autocorelate}  
		\textbf{M}\left(l\right)\left(p\right) =
		\textbf{G}_{\sigma}
		\cdot\left[
		\begin{matrix}
			\textbf{I}_x^2\left(l\right)\left(p\right) & \textbf{I}_x\left(l\right)\left(p\right) \cdot \textbf{I}_y\left(l\right)\left(p\right)\\
			\textbf{I}_y\left(l\right)\left(p\right) \cdot \textbf{I}_x\left(l\right)\left(p\right) & \textbf{I}_y^2\left(l\right)\left(p\right)
		\end{matrix}
		\right]
        \end{align}\\[10 pt]
        Gdzie poszczególne symbole oznaczają:\\[10 pt]
    	\begin{tabular}{rcl}
    	$G_{\sigma}$ & -- & 
    	\begin{minipage}[t]{123 mm}
    	prostokątna macierz o liczbie kolumn 2 stanowiąca okno czasowe Gaussa o odchyleniu standardowym $\sigma$
    	\end{minipage}\\[5 pt]
    	$\textbf{M}\left(l\right)\left(p\right)$ & -- & 
    	\begin{minipage}[t]{123 mm}
    	macierz autokorelacji dla danego piksela $p$ i poziomu obrazu $l$ wyznaczana ze wzoru \ref{eq:autocorelate} \cite{harrispptx, harris, detection}
    	\end{minipage}
    	\end{tabular}\\[10 pt]
		
        \section{Wyznaczanie obwiedni sygnału}
        Sygnał to zapis informacji zmieniającej się w zależności od czasu. Taką informacją może być natężenie/napięcie prądu, natężenie pola elektromagnetycznego, dźwięk utworu muzycznego itd. W przypadku, jeśli zmienną informację da się  reprezentować w sposób liczbowy, możliwe jest graficzne przedstawienie sygnału. Zapisanie informacji w ten sposób może być łatwiejsze do interpretacji przez człowieka. Jednak w przypadku, gdy mamy do czynienia ze skomplikowanym przebiegiem, na
        
        Algorytm wyznaczający otoczkę wypukłą wielokąta prostego może posłużyć znajdowaniu obwiedni sygnału.
        
   \chapter{Implementacja w języku Scala}
   		\section{Pojęcia ogólne}
   		W tej sekcji przedstawione zostaną listingi zawierające kod w języku Scala definiujący pojęcia potrzebne do implementacji poszczególnych algorytmów wyznaczających otoczkę wypukłą.
   		
   		Klasa \texttt{Point} (Listing \ref{lst:point}: linia 4) posłuży jako podstawowa klasa reprezentująca punkt w przestrzeni dwuwymiarowej. W celu łatwiejszego manipulowania danymi. Zostały w jej ramach zaimplementowane 3 metody pomocnicze.
   		
   		Odejmowanie (Listing \ref{lst:point}: linia 7) - zaimplementowane jako odejmowanie od siebie odpowiadających sobie współrzędnych dwu punktów.
   		
   		Dodawanie (Listing \ref{lst:point}: linia 8) - dodawanie do siebie odpowiadających sobie współrzędnych dwu punktów.
   		
   		Dzielenie przez skalar (Listing \ref{lst:point}: linia 9) - zaimplementowane jako dzielenie obu współrzędnych przez wskazaną liczbę. Zaimplementowane dzięki użyciu domniemanemu argumentowi \scala{numeric: Numeric[T]}. W języku Scala tego typu konstrukcja umożliwia korzystanie z metody \scala{/} w tedy i tylko wtedy jeśli zaimportowany zostanie do aktualnego kontekstu obiekt typu \scala{Numeric[T]}. Domyślnie typ ten jest zaimplementowany dla \scala{T} będącym jednym z podstawowych typów liczbowych: \scala{Float}, \scala{Int}, \scala{Double} etc. Więc w domyśle tego typu konstrukcja służy niejako ,,udowodnieniu'', że wskazywany argument jest w istocie liczbą.
   		
   		\lstinputlisting[language=scala, caption={Point.scala\mysrc}, label={lst:point}]{\listingsDirectory/Point.scala}
   		Kolejnym istotnym elementem wykorzystanym w implementacji algorytmów wyznaczających otoczkę wypukłą zbioru punktów na płaszczyźnie są metody wyliczające określone właściwości punktów.
   		
   		Metoda \scala{calculateCentroid(points: Points): Points} (Listing \ref{lst:pointsutils}: linia 11) wyznacza centroid zbioru punktów \scala{points}.
   		
   		Metoda \scala{distanceFromCenterSquared(p: Point): Double} (Listing \ref{lst:pointsutils}: linia 17) wyznacza kwadrat odległości wskazanego punktu od środka układu współrzędnych.
   		
   		Na podstawie wyżej wymienionej metody zostały zaimplementowane trzy metody związane z odległością między punktami.
   		
   		Metoda \scala{distanceSquared(a: Point, b: Point): Double} (Listing \ref{lst:pointsutils}: linia 13) wyznacza kwadrat odległości pomiędzy dwoma punktami.
   		
   		Metoda \scala{distanceFromCenter(p: Point): Double} (Listing \ref{lst:pointsutils}: linia 19) wyznacza odległość od środka układu współrzędnych.
   		
   		Metoda \scala{distance(a: Point, b: Point): Double} (Listing \ref{lst:pointsutils}: linia 15) wyznacza odległość pomiędzy dwoma punktami.

		W ramach obiektu \scala{PointsUtils} (Listing \ref{lst:pointsutils}: linia 5) zostały również zaimplementowane metody służące do wyznaczania wartości umożliwiających sortowanie punktów względem kąta występującego pomiędzy horyzontalną osią układu współrzędnych, a wektorem łączącym początek układu współrzędnych z punktem (kąt $\theta$ zaznaczony na rysunku \ref{fig:axis}).
		
		Metoda \scala{phase(p: Point): Double} (Listing \ref{lst:pointsutils}: linia 21) wyznacza dokładną wartość szukanego kąta. Matematycznie ta funkcja została zapisana we wzorze \ref{eq:theta}.
		
		Metoda \scala{alpha(p: Point): Double} (Listing \ref{lst:pointsutils}: linia 32) wyznacza wartość umożliwiającą porównanie kąta dla wskazanego punktu z innymi punktami. Matematycznie ta funkcja została zapisana we wzorze \ref{eq:alpha1}.
   		
   		\lstinputlisting[language=scala, caption={PointsUtils.scala}, label={lst:pointsutils}]{\listingsDirectory/PointsUtils.scala}
   		\newpage
   		
   		W celu zapewnienia odpowiedniego sortowania punktów został utworzony \scala{trait OrientationOrdering}. Język Scala umożliwia wykorzystywanie zaimplementowanie własnego obiektu \scala{Ordering[T]}, który będzie następnie używany w metodzie \scala{sorted} dla kolekcji \scala{C[T]}.
   		
   	 	Trait \scala{OrientationOrdering} (Listing \ref{lst:order}: linia 6) ma zaimplementowaną metodę \scala{compare(x: Point, y: Point)} (Listing \ref{lst:order}: linia 7) w taki sposób, aby w pierwszej kolejności porównywane były wartości funkcji reprezentujących fazę tych punktów, a następnie porównywane wartości funkcji reprezentujących odległość od środka układu współrzędnych.
   	 	
   	 	W ramach \scala{OrientationOrdering} zostały zaimplementowane dwa obiekty.
   	 	
   	 	Obiekt \scala{Exact} (Listing \ref{lst:order}: linia 19) służy do dokładnego porównania fazy punktów, wyznaczanej ze wzoru \ref{eq:theta} oraz odległości od środka układu współrzędnych. 
   	 	
   	 	Obiekt \scala{Indicator} (Listing \ref{lst:order}: linia 24) służy do porównywania fazy punktów za pomocą funkcji wyznaczanej ze wzoru \ref{eq:alpha1}.
   		
   		\lstinputlisting[language=scala,caption={OrientationOrdering.scala},captionpos=b, label={lst:order}]{\listingsDirectory/OrientationOrdering.scala}
   		\newpage
   		\lstinputlisting[language=scala,caption={ConvexHullAlgorithm.scala},captionpos=b]{\listingsDirectory/ConvexHullAlgorithm.scala}
   		\newpage
   		\lstinputlisting[language=scala,caption={AlgorithmTest.scala},captionpos=b]{\listingsDirectory/AlgorithmTest.scala}
   		\newpage
   		\section{Algorytm Grahama}
   		\lstinputlisting[language=scala,caption={Graham.scala},captionpos=b]{\listingsDirectory/Graham.scala}
   		\newpage
   		\lstinputlisting[language=scala,caption={PointsCycle.scala},captionpos=b]{\listingsDirectory/PointsCycle.scala}
   		\newpage
   		\lstinputlisting[language=scala,caption={GrahamAlgorithmTest.scala},captionpos=b]{\listingsDirectory/GrahamAlgorithmTest.scala}
   		
   		\section{Algorytm Jarvisa}
   		Dzięki wykorzystaniu pattern matchingu dostępnego w Scali, możliwe jest przejrzyste zaimplementowanie algorytmu Jarvisa, który, jak widać jest dużo mniej skomplikowany od algorytmu Grahama. Mniejsze skomplikowanie implementacji wiąże się jednak z dużo większą złożonością obliczeniową.
        \lstinputlisting[language=scala,caption={Jarvis.scala},captionpos=b]{\listingsDirectory/Jarvis.scala}
        \lstinputlisting[language=scala,caption={JarvisAlgorithmTest.scala},captionpos=b]{\listingsDirectory/JarvisAlgorithmTest.scala}
    \chapter{Dynamiczna otoczka wypukła}
    	\section{Algorytm}
    	\section{Implementacja w języku Scala} 
	\chapter{Podsumowanie} 
    \printbibliography[heading=bibintoc]
\end{document}

    
    Z powyższych obliczeń wynika, że we wszystkich przedziałach z zależności
    \ref{eq:xytotheta1}, 
     \ref{eq:xytotheta2}, 
     \ref{eq:xytotheta3} oraz
     \ref{eq:xytotheta4} funkcja $\gamma\left(\theta\right)$ ma dodatnią pochodną, a tym samym jest rosnąca we wszystkich przedziałach, oraz każda z wartości osiąganych w przedziale jest większa od każdej wartości z poprzedniego przedziału, a tym samym funkcja $\alpha\left(p\right)$ spełnia warunki \ref{eq:condition1} i \ref{eq:condition2}.
     
     Postać funkcji $\gamma\left(\theta\right)$ może być opisana wzorem \ref{eq:gammaeq} a wykres jej przebiegu, przedstawiony na rysunku \ref{fig:gamma} wyraźnie potwierdza jej rosnącą monotoniczność we wszystkich przedziałach.
        \begin{align}\label{eq:gammaeq}
        	\gamma\left(
        		\theta
        	\right) = 
        		\left\{
        			\begin{array}{lcl}
        				\frac{
        					\sin\theta
        				}{
							\sin\theta + \cos\theta        			
        				} & 
        					\text{dla} &
        					\theta \in \left\langle
        						0;
        						\frac{
        							\pi
        						}{
									2        						
        						}
        					\right\rangle\\
        				2
        				- \frac{
        					\sin\theta
        				}{
							\sin\theta - \cos\theta        			
        				} & 
        					\text{dla} &
        					\theta \in \left(
        						\frac{
        							\pi
        						}{
									2        						
        						};
        						\pi
        					\right\rangle\\
        				2
        				+ \frac{
        					\sin\theta
        				}{
							\sin\theta + \cos\theta        			
        				} & 
        					\text{dla} &
        					\theta \in \left(
        						\pi;
        						\frac{
        							3\pi
        						}{
									2        						
        						}
        					\right)\\
        				4
        				- \frac{
        					\sin\theta
        				}{
							\sin\theta + \cos\theta        			
        				} & 
        					\text{dla} &
        					\theta \in \left\langle
        						\frac{
        							3\pi
        						}{
									2        						
        						};
        						2\pi
        					\right)\\
        			\end{array}
        		\right.      		
        \end{align}
        \begin{figure}[h!]
        	\begin{center}
    			\begin{tikzpicture}
	
	\draw 
		[help lines, very thin] 
		(0,0) 
		grid [xstep = pi/2]
		(2*pi - 0.001,4- 0.001);
		
	\tzaxes 
		[
			-arcs,
			line width=0.2mm
		]
		(2*pi,4)
		{$\theta$}
		{$\gamma\left(\theta\right)$}
		
	\tzticksx
		(-5pt:5pt)
		{
			0,
			pi*.5/$\frac{\pi}{2}$,
			pi/$\pi$,
			pi*1.5/$\frac{3\pi}{2}$
		}
		[below=5pt]
		
	\tzticksy
		(-5pt:5pt)
		{	
			0,
			1,
			2,
			3
		}
		[left=5pt]
	\tzfn
		[
			black,
			line width=0.25mm
		]
		{sin(\x r)/(sin(\x r) + cos(\x r))}
		[0:.5*pi] 
	\tzfn
		[
			black,
			line width=0.25mm
		]
		{2 - sin(\x r)/(sin(\x r) - cos(\x r))}
		[.5*pi:pi] 
	\tzfn
		[
			black,
			line width=0.25mm
		]
		{2 + sin(\x r)/(sin(\x r) + cos(\x r))}
		[pi:1.5*pi] 
	\tzfn
		[
			black,
			line width=0.25mm
		]
		{4 + sin(\x r)/(cos(\x r) - sin(\x r))}
		[1.5*pi:2*pi] 
\end{tikzpicture}
    			\caption{Wykres funkcji $\gamma\left(\theta\right)$\mysrc}\label{fig:gamma}
    	\end{center}
    	\end{figure}\vspace{-8 mm}\\
       
              
        \subsection{Algorytm Grahama}
        \null
        \begin{figure}[h!]
    			\begin{tikzpicture}[node distance=1.4cm] 
	\node (0,0) (start) {};
	\path[ellipse_block = 
    		reference: start 
    		text: START
	];
	\node[below of = start_block] (input) {};
	\block
		{input}
		{Dane wejściowe:\\
		zbiór punktów 
			$P = \left\{
				p_1,
				p_2,
				...,
				p_n
			\right\} \subset \mathbb{R}^2$, dla każdego z punktów w zbiorze określone są funkcje:\\
			 $x\left(p\right)$ (oznaczająca wartość odciętej punktu $p$)\\
			 $y\left(p\right)$ (oznaczająca wartość rzędnej punktu $p$)\\
		\vspace{.8cm}
		}
	\node[below of = input_south] (centroid) {};
	\block
		{centroid}
		{
		Wyznaczyć punkt $O$, stanowiący centroid zbioru $P$.
		Następnie dla każdego punktu $p$ ze zbioru $P$ wykonać przekształcenia:
		 $x\left(p\right) := x\left(p\right) -  x\left(O\right)$\\
		 $y\left(p\right) := y\left(p\right) -  y\left(O\right)$\\
		 Następnie przypisać współrzędne punktu $O$:\\
		 $x\left(O\right) := 0$\\
		 $y\left(O\right) := 0$\\
		(Tak aby punkt $O$ znajdował się na początku układu współrzędnych.)
		}



	\node[below of = centroid_south] (sort) {};
	\block
		{sort}
		{
			Posortować leksykograficznie wszystkie punkty zbioru $P$ według porządku $\left(\theta\left(p\right), \left|R\right|\left(p\right)\right)$, gdzie $\theta\left(p\right)$ to wartość kąta wektora wiodącego punktu $p$, natomiast $\left|R\right|\left(p\right)$ to długość tego wektora.
		}
	\node[below of = sort_south] (cykl) {};
	\block
		{cykl}
		{
			Z otrzymanego, uporządkowanego zbioru $P$ utworzyć listę pozwalającą na wyznaczenie następnika punktu $p$ w liście - poprzez zastosowanie funkcji $next\left(p\right)$ oraz znalezienie poprzednika punktu $p$ w liście poprzez zastosowanie funkcji $prev\left(p\right)$.\\
	Funkcja $next$ dla ostatniego punktu w uporządkowanym zbiorze będzie wskazywała na pierwszy punkt tego zbioru, natomiast funkcja $prev$ dla pierwszego punktu uporządkowanego zbioru będzie wskazywała na ostatni punkt tego zbioru.
		}
	\node[below of =cykl_south] (punkts) {};
	\block
		{punkts}
		{
			Wyznacz punkt $s$ będący punktem o najmniejszej wartości odciętej ze zbioru punktów o najmniejszej wartości rzędnej ze zbioru $P$.
		}
		
	\node[right of =start, node distance = 8 cm] (q) {};
	\smallblock
		{q}
		{
			$q:=s$
		}
	\node[below of =q] (isequal) {};
	\chooseblock
		{isequal}
		{
			\begin{small}
			$next\left(q\right) \neq s$
			\end{small}
		}
	\node[below of =isequal_south] (inside) {};
	\chooseblock
		{inside}
		{
			$next\left(q\right) \in \Delta\left(
			 	O,
			 	q,
			 	next\left(
			 		next\left(
			 			q
			 		\right)
			 	\right)
			 \right)$
		}
	\node[below of =inside_south] (remove) {};
	\smallblock
		{remove}
		{
			$next\left(q\right) := next\left(
				next\left(
					q
				\right)
			\right)$\\
			$prev\left(next\left(q\right)\right) := q$
			
			
		}
	
	\node[below of =remove_south] (qequalss) {};
	\chooseblock
		{qequalss}
		{	
		\begin{small}
			$q\neq s$
		\end{small}	
		}
	
	\node[below of =qequalss_south] (prev) {};
	\smallblock
		{prev}
		{	
		$q:=prev\left(q\right)$
		}
	
	\node[below of =prev_south] (next) {};
	\smallblock
		{next}
		{	
		$q:=next\left(q\right)$
		}
	
	\node[below of =next_south] (output) {};
	\block
		{output}
		{	
			Wynikiem działania algorytmu jest lista zaczynająca się od punktu $s$ - zawiera ona uporządkowane wierzchołki otoczki wypukłej zbioru punktów $P$.
		}
	
	\node[below of =output_south] (stop) {};
	\path[ellipse_block = 
    		reference: stop
    		text: STOP
	];
	
	\draw
		[-arcs]
		(start_block)--
		(input_block);
		
	\draw
		[-arcs]
		(input_block)--
		(centroid_block);
	
	\draw
		[-arcs]
		(centroid_block)--
		(sort_block);
	
	\draw
		[-arcs]
		(sort_block)--
		(cykl_block);
	
	\draw
		[-arcs]
		(cykl_block)--
		(punkts_block);
	
	
	\draw [-arcs] 
		(punkts_block.east) --++
		(.5,0) |-
		(q_block.west);
	
	\draw [-arcs] 
		(q_block) -- node (centerq) {}
		(isequal_block);
		
	\draw [-arcs] 
		(isequal_block) --
		(inside_block);
	
	\draw [-arcs] 
		(inside_block) --
		(remove_block);
	
	\draw [-arcs] 
		(remove_block) --
		(qequalss_block);
		
	\draw [-arcs] 
		(qequalss_block) --
		(prev_block);
	
	\draw [-arcs] 
		(isequal_block.west) --++
		(-2,0) |-
		(output_block.west);
	
	\draw [-arcs] 
		(inside_block.west) --++
		(-1.07,0) |-
		(next_block.west);
		
	\draw [-arcs] 
		(inside_block.west) --++
		(-1.07,0) |-
		(next_block.west);
	
	
		
	
	
	\draw [-arcs]
		(prev_block.south) --++
		(0,-1) node (prev_below) {};
		
	\draw [-arcs]
		(qequalss_block.west) --++
		(-1.95,0)  |-
		(prev_below.center) --++
		(3.3,0)
		node (prev_right) {};
	
	\draw [-arcs]
		(next_block.east) -|
		(prev_right.center) |-
		(centerq.center);
	
	\draw [-arcs]
		(output_block) --
		(stop_block);
	
	
	
\end{tikzpicture}
    			\caption{Algorytm Grahama}\label{fig:graham}
    	\end{figure}
       \newpage
        \subsection{Algorytm Jarvisa}
        
        
        \section{Otoczka wypukła wielokąta prostego}
        \cite{convexhullsimplepolygon}
        \section{Redukcja zbioru punktów do wielokąta prostego}
    \chapter{Zastosowania} 
%    as first chapter
    Istotnym, jeśli nie najistotniejszym zagadnieniem dotyczącym otoczek wypukłych są ich zastosowania. Już od kilkudziesięciu lat problem ten temat znajduje swoje użycie w wielu dziedzinach kombinatoryki i informatyki. Dla przykładu problem sortowania elementów liczbowych listy można sprowadzić do problemu znalezienia otoczki wypukłej. W związku z tym rozwiązanie tego problemu może pomóc w rozwiązaniu innych problemów w bardziej symboliczny i graficzny sposób.
        \section{Generalizacja kartograficzna}
        W celu jak najbardziej informatywnego i niezłożonego przedstawienia danych geograficznych w sposób graficzny potrzebne jest zastosowanie algorytmu generalizującego informacje. Taka generalizacja w kartografii zaczęła być coraz bardziej potrzebna, kiedy w dobie cyfrowego przetwarzania danych mapy zaczęły być przechowywane cyfrowo. Przed tym generalizacją zajmowali się manualnie ludzie, którzy wyposażeni w ludzkie zmysły wykrywania wzorów i kształtów byli w stanie z łatwością narysować obrys zbioru drzew na płaszczyźnie i opisać go jako teren leśny. W przypadku przechowywania danych geograficznych w postaci cyfrowej zaistniała potrzeba wykonywania tej samej operacji w sposób zalgorytmizowany. \cite{cartography}
        
        Do celu generalizacji można by wykorzystać otoczkę wypukłą w sposób bezpośredni \cite{LAURINI2017139}, jednak taki obrys terenu może zawierać w sobie przestrzenie pozbawione obiektów, co byłoby niepomocne dla ludzkiej interpretacji przedstawianego mu obszaru. 
        
        Należy też zwrócić uwagę, że otoczka wypukła wielokąta prostego zawiera w sobie wszystkie punkty należące do wewnętrznej powierzchni tego wielokąta. \cite{convexhullsimplepolygon} W związku z tym jeśli udowodni się, że dany punkt nie należy do wnętrza otoczki wypukłej będzie to implikowało, że punkt ten nie należy do wnętrza wielokąta prostego. Taka wiedza okazuje się być przydatna jeśli uwzględnimy dodatkowo fakt, że określanie czy punkt leży wewnątrz wielokąta wypukłego jest dużo łatwiejsze niż dla pozostałych wielokątów. \cite{heckbert1994graphics}
        
        Dla przykładu rozpatrując obrys granic Polski (Rysunek \ref{fig:poland}; źródło współrzędnych obrysu: \cite{datahubCountryPolygons}; współrzędne przedstawione zgodnie z projekcją EPSG:2180 \cite{epsg2180} przy użyciu biblioteki GeoTools \cite{geotools}) można zauważyć, że jego kształt jest bardzo skomplikowany. W istocie zawiera on 1336 krawędzi.
        \begin{figure}[h!]
        	\begin{center}
    			\begin{tikzpicture}[scale=5.5] \draw (0.47582,0.85189)--(0.51112,0.85726)--(0.51889,0.86108)--(0.52258,0.86197)--(0.52555,0.86337)--(0.53129,0.86945)--(0.53288,0.87017)--(0.53489,0.86856)--(0.53955,0.86699)--(0.54689,0.86668)--(0.56630,0.86588)--(0.60275,0.86453)--(0.62993,0.86366)--(0.65713,0.86291)--(0.69084,0.86213)--(0.73168,0.86141)--(0.75791,0.86108)--(0.78416,0.86086)--(0.80617,0.86075)--(0.82391,0.86071)--(0.83028,0.86320)--(0.83197,0.86474)--(0.83411,0.86928)--(0.83422,0.86953)--(0.83651,0.87075)--(0.83852,0.87059)--(0.84432,0.86799)--(0.84535,0.86791)--(0.84746,0.86848)--(0.84844,0.86831)--(0.84949,0.86735)--(0.85010,0.86612)--(0.85059,0.86489)--(0.85124,0.86389)--(0.85234,0.86323)--(0.85559,0.86303)--(0.85624,0.86217)--(0.85598,0.86062)--(0.85554,0.85876)--(0.85563,0.85706)--(0.85746,0.85478)--(0.85962,0.85492)--(0.86182,0.85600)--(0.86369,0.85654)--(0.86545,0.85581)--(0.87159,0.85125)--(0.87532,0.84926)--(0.88310,0.84684)--(0.88508,0.84535)--(0.88692,0.84397)--(0.88791,0.84261)--(0.88983,0.83943)--(0.89078,0.83823)--(0.89180,0.83763)--(0.89386,0.83720)--(0.89476,0.83673)--(0.89647,0.83452)--(0.89800,0.83138)--(0.89926,0.82784)--(0.90020,0.82445)--(0.90206,0.81702)--(0.90232,0.81339)--(0.90107,0.81083)--(0.89998,0.80980)--(0.89951,0.80913)--(0.89909,0.80839)--(0.89914,0.80666)--(0.89964,0.80522)--(0.90048,0.80411)--(0.90154,0.80334)--(0.90215,0.80265)--(0.90247,0.80186)--(0.90247,0.80097)--(0.90214,0.80001)--(0.90255,0.79530)--(0.90379,0.79142)--(0.90528,0.78779)--(0.90652,0.78380)--(0.90794,0.77529)--(0.90916,0.77204)--(0.91169,0.76879)--(0.91169,0.76878)--(0.91170,0.76878)--(0.91256,0.75891)--(0.91557,0.74782)--(0.92534,0.72330)--(0.93050,0.71416)--(0.93277,0.70920)--(0.93764,0.69423)--(0.93965,0.68976)--(0.94155,0.68754)--(0.94265,0.68533)--(0.94356,0.68304)--(0.94488,0.68063)--(0.94674,0.67884)--(0.94849,0.67767)--(0.94964,0.67580)--(0.94969,0.67188)--(0.94897,0.66950)--(0.94793,0.66748)--(0.94715,0.66520)--(0.94724,0.66205)--(0.94825,0.65938)--(0.95145,0.65580)--(0.95273,0.65349)--(0.95297,0.65226)--(0.95288,0.65031)--(0.95296,0.64929)--(0.95409,0.64486)--(0.95422,0.64357)--(0.95399,0.64256)--(0.95307,0.64031)--(0.95288,0.63902)--(0.95309,0.63730)--(0.95349,0.63636)--(0.95394,0.63558)--(0.95427,0.63436)--(0.95643,0.61490)--(0.95696,0.61008)--(0.95609,0.60310)--(0.95252,0.59804)--(0.94011,0.58829)--(0.92402,0.58256)--(0.91565,0.57694)--(0.90743,0.56920)--(0.89656,0.55004)--(0.89290,0.54497)--(0.89125,0.54203)--(0.88721,0.53241)--(0.88752,0.53235)--(0.88828,0.53178)--(0.88905,0.53121)--(0.89022,0.52968)--(0.89066,0.52758)--(0.89004,0.52467)--(0.89235,0.52338)--(0.89958,0.52187)--(0.90107,0.52256)--(0.90245,0.52132)--(0.90873,0.51938)--(0.91082,0.51929)--(0.91088,0.51829)--(0.91024,0.51676)--(0.91055,0.51651)--(0.91193,0.51715)--(0.91321,0.51667)--(0.91415,0.51581)--(0.91511,0.51569)--(0.91636,0.51742)--(0.91680,0.51673)--(0.91783,0.51596)--(0.91853,0.51524)--(0.92009,0.51695)--(0.92054,0.51577)--(0.92039,0.51382)--(0.92002,0.51323)--(0.92312,0.50790)--(0.92508,0.50741)--(0.92978,0.50785)--(0.93171,0.50684)--(0.93245,0.50563)--(0.93303,0.50435)--(0.93369,0.50337)--(0.93465,0.50301)--(0.93592,0.50224)--(0.93644,0.50041)--(0.93666,0.49856)--(0.93761,0.49661)--(0.93940,0.49062)--(0.94072,0.48794)--(0.93923,0.48668)--(0.93806,0.48454)--(0.93646,0.47995)--(0.93601,0.47694)--(0.93626,0.47172)--(0.93573,0.46986)--(0.93613,0.46973)--(0.93631,0.46973)--(0.93639,0.46954)--(0.93649,0.46881)--(0.93534,0.46806)--(0.93515,0.46454)--(0.93412,0.46314)--(0.93587,0.46032)--(0.93787,0.45795)--(0.93699,0.45414)--(0.93519,0.45211)--(0.93317,0.45069)--(0.93160,0.44863)--(0.93127,0.44745)--(0.93098,0.44570)--(0.93064,0.44195)--(0.93106,0.44114)--(0.93180,0.44005)--(0.93209,0.43906)--(0.93120,0.43856)--(0.93083,0.43809)--(0.93122,0.43707)--(0.93190,0.43600)--(0.93244,0.43533)--(0.93125,0.43366)--(0.93066,0.43325)--(0.92977,0.43307)--(0.92985,0.43186)--(0.93106,0.43069)--(0.93129,0.42914)--(0.93104,0.42476)--(0.93153,0.42247)--(0.93254,0.42028)--(0.93498,0.41663)--(0.93430,0.41652)--(0.93361,0.41655)--(0.93501,0.41411)--(0.93665,0.41360)--(0.93807,0.41286)--(0.93860,0.41073)--(0.93885,0.40975)--(0.93967,0.40754)--(0.94132,0.40654)--(0.94318,0.40598)--(0.94466,0.40510)--(0.94330,0.40178)--(0.94352,0.40079)--(0.94800,0.39501)--(0.94895,0.39313)--(0.94782,0.39251)--(0.94717,0.39134)--(0.94715,0.38968)--(0.94792,0.38754)--(0.94612,0.38670)--(0.94445,0.38469)--(0.94321,0.38223)--(0.94281,0.38005)--(0.94368,0.37667)--(0.94532,0.37595)--(0.94727,0.37604)--(0.94906,0.37502)--(0.95545,0.36312)--(0.95791,0.36050)--(0.96327,0.35757)--(0.96840,0.35298)--(0.96966,0.35110)--(0.96805,0.35012)--(0.96777,0.34861)--(0.96948,0.34552)--(0.97241,0.34157)--(0.97344,0.33948)--(0.97508,0.33282)--(0.97515,0.33182)--(0.97480,0.33053)--(0.97559,0.32992)--(0.97692,0.32948)--(0.97822,0.32872)--(0.97945,0.32708)--(0.97983,0.32577)--(0.98003,0.32433)--(0.98073,0.32226)--(0.98242,0.31983)--(0.98974,0.31394)--(0.99854,0.30985)--(1.00000,0.30792)--(0.99894,0.30504)--(0.99589,0.30417)--(0.99251,0.30391)--(0.99050,0.30284)--(0.98769,0.30418)--(0.98491,0.30362)--(0.98269,0.30168)--(0.98162,0.29883)--(0.98202,0.29577)--(0.98366,0.29382)--(0.98615,0.29286)--(0.98903,0.29271)--(0.98910,0.29171)--(0.98796,0.28880)--(0.98956,0.28643)--(0.99247,0.28488)--(0.99526,0.28439)--(0.99489,0.28068)--(0.99593,0.27735)--(0.99895,0.27127)--(0.99899,0.27121)--(0.99690,0.26694)--(0.99846,0.25936)--(0.99934,0.25728)--(0.99990,0.25687)--(0.99980,0.25650)--(0.99864,0.25456)--(0.99684,0.25236)--(0.99047,0.24845)--(0.99031,0.24649)--(0.99068,0.24340)--(0.99065,0.24130)--(0.99030,0.23953)--(0.98841,0.23408)--(0.98314,0.23147)--(0.96451,0.23002)--(0.96103,0.22867)--(0.95931,0.22765)--(0.95800,0.22619)--(0.95702,0.22376)--(0.95599,0.21940)--(0.95594,0.21917)--(0.95464,0.21700)--(0.94706,0.20766)--(0.94422,0.20506)--(0.94420,0.20506)--(0.94419,0.20505)--(0.94418,0.20505)--(0.93882,0.20039)--(0.93437,0.19649)--(0.91220,0.16943)--(0.90935,0.16443)--(0.90572,0.16124)--(0.90184,0.15645)--(0.89146,0.13930)--(0.88733,0.13459)--(0.88686,0.13323)--(0.88618,0.12994)--(0.88575,0.12877)--(0.88489,0.12781)--(0.88309,0.12704)--(0.88220,0.12638)--(0.88125,0.12509)--(0.87542,0.11311)--(0.87375,0.11124)--(0.87263,0.11069)--(0.87041,0.11013)--(0.86930,0.10906)--(0.86861,0.10763)--(0.86706,0.10240)--(0.85965,0.09133)--(0.85734,0.08499)--(0.85963,0.07933)--(0.86045,0.07701)--(0.86144,0.07019)--(0.86192,0.06838)--(0.86226,0.06750)--(0.86373,0.06540)--(0.86612,0.06312)--(0.86677,0.06192)--(0.86736,0.05938)--(0.86951,0.04469)--(0.86923,0.04233)--(0.86769,0.03734)--(0.86648,0.03156)--(0.86498,0.02800)--(0.86451,0.02601)--(0.86519,0.02519)--(0.86567,0.02554)--(0.86698,0.02736)--(0.86870,0.02628)--(0.86857,0.02620)--(0.86817,0.02608)--(0.87172,0.02384)--(0.87501,0.01992)--(0.87703,0.01858)--(0.88190,0.01617)--(0.88326,0.01460)--(0.88334,0.01324)--(0.88254,0.01003)--(0.88252,0.00784)--(0.88306,0.00635)--(0.88493,0.00348)--(0.88508,0.00315)--(0.88531,0.00263)--(0.88420,0.00000)--(0.88201,0.00081)--(0.87956,0.00281)--(0.87762,0.00370)--(0.87643,0.00402)--(0.87433,0.00667)--(0.87321,0.00763)--(0.87205,0.00771)--(0.86971,0.00687)--(0.86586,0.00696)--(0.86361,0.00664)--(0.86132,0.00680)--(0.85863,0.00846)--(0.85445,0.01265)--(0.85230,0.01321)--(0.85017,0.01096)--(0.84642,0.01259)--(0.83813,0.01257)--(0.83420,0.01359)--(0.82860,0.01874)--(0.82632,0.01953)--(0.82048,0.01905)--(0.81546,0.02035)--(0.81464,0.02099)--(0.81436,0.02200)--(0.81461,0.02305)--(0.81455,0.02418)--(0.81334,0.02542)--(0.81246,0.02555)--(0.80994,0.02514)--(0.80888,0.02516)--(0.80764,0.02565)--(0.80411,0.02772)--(0.79658,0.02884)--(0.79350,0.03091)--(0.79260,0.03601)--(0.79105,0.04164)--(0.78783,0.04641)--(0.78389,0.04984)--(0.77811,0.05245)--(0.77412,0.05586)--(0.77215,0.05689)--(0.77005,0.05642)--(0.76830,0.05469)--(0.76687,0.05286)--(0.76580,0.05209)--(0.76413,0.05334)--(0.76040,0.05866)--(0.75857,0.06041)--(0.75751,0.06068)--(0.75590,0.06026)--(0.75510,0.06034)--(0.75401,0.06104)--(0.75199,0.06288)--(0.75090,0.06356)--(0.74892,0.06400)--(0.74142,0.06289)--(0.73985,0.06219)--(0.73796,0.06135)--(0.73640,0.06176)--(0.73253,0.06077)--(0.73079,0.06070)--(0.72046,0.06327)--(0.71448,0.06623)--(0.71302,0.06654)--(0.71116,0.06516)--(0.70802,0.06021)--(0.70627,0.05847)--(0.70402,0.05802)--(0.70238,0.05909)--(0.70083,0.06072)--(0.69890,0.06197)--(0.69726,0.06207)--(0.69302,0.06109)--(0.69143,0.06028)--(0.68937,0.05784)--(0.69069,0.05641)--(0.69315,0.05500)--(0.69450,0.05261)--(0.69368,0.05111)--(0.69175,0.05067)--(0.68953,0.05058)--(0.68789,0.05019)--(0.68578,0.04803)--(0.68247,0.04290)--(0.68025,0.04087)--(0.67780,0.03991)--(0.67585,0.04023)--(0.67411,0.04143)--(0.67235,0.04318)--(0.67034,0.04459)--(0.66868,0.04480)--(0.66689,0.04465)--(0.66452,0.04496)--(0.66283,0.04614)--(0.65323,0.05708)--(0.65158,0.05734)--(0.64762,0.05661)--(0.64534,0.05690)--(0.64438,0.05681)--(0.64331,0.05618)--(0.64168,0.05408)--(0.64050,0.05292)--(0.63803,0.05198)--(0.63571,0.05250)--(0.62674,0.05686)--(0.62510,0.05645)--(0.62518,0.05368)--(0.61975,0.05337)--(0.61541,0.05492)--(0.61416,0.05487)--(0.61310,0.05405)--(0.61273,0.05304)--(0.61254,0.05156)--(0.61203,0.04925)--(0.61132,0.04700)--(0.61081,0.04628)--(0.60270,0.04543)--(0.60111,0.04451)--(0.59886,0.04173)--(0.59784,0.04075)--(0.59552,0.04099)--(0.59525,0.04123)--(0.59466,0.04043)--(0.59272,0.03588)--(0.59214,0.03395)--(0.59144,0.03215)--(0.59016,0.03055)--(0.58964,0.02492)--(0.58860,0.02087)--(0.58656,0.01925)--(0.58304,0.02092)--(0.57753,0.02597)--(0.57457,0.02746)--(0.57120,0.02704)--(0.56931,0.02560)--(0.56727,0.02345)--(0.56580,0.02190)--(0.56347,0.02101)--(0.55860,0.02132)--(0.55592,0.02227)--(0.55454,0.02414)--(0.55454,0.02415)--(0.55494,0.02626)--(0.55887,0.03281)--(0.55933,0.03320)--(0.56001,0.03340)--(0.56061,0.03376)--(0.56085,0.03467)--(0.56065,0.03537)--(0.56061,0.03541)--(0.55980,0.03624)--(0.55977,0.03630)--(0.55960,0.03667)--(0.55959,0.03667)--(0.55814,0.03773)--(0.55777,0.03826)--(0.55777,0.03828)--(0.55782,0.03895)--(0.55862,0.04016)--(0.55877,0.04082)--(0.55807,0.04860)--(0.55751,0.05128)--(0.55650,0.05432)--(0.55648,0.05433)--(0.55553,0.05504)--(0.55202,0.05359)--(0.54989,0.05335)--(0.54760,0.05358)--(0.54154,0.05559)--(0.54154,0.05564)--(0.54175,0.05753)--(0.54166,0.06006)--(0.54182,0.06093)--(0.54230,0.06195)--(0.53810,0.06196)--(0.53590,0.06253)--(0.53409,0.06391)--(0.53354,0.06505)--(0.53181,0.07017)--(0.52984,0.07828)--(0.52855,0.08151)--(0.52609,0.08313)--(0.52529,0.08396)--(0.52351,0.08706)--(0.52262,0.08742)--(0.52204,0.08765)--(0.52144,0.08738)--(0.52104,0.08658)--(0.51201,0.07652)--(0.51122,0.07584)--(0.50974,0.07523)--(0.50866,0.07505)--(0.50543,0.07507)--(0.50341,0.07463)--(0.50170,0.07378)--(0.50018,0.07232)--(0.49875,0.07003)--(0.49709,0.06480)--(0.49727,0.06461)--(0.49761,0.06413)--(0.49782,0.06337)--(0.49757,0.06229)--(0.49711,0.06194)--(0.49581,0.06190)--(0.49551,0.06184)--(0.49478,0.05870)--(0.49451,0.05669)--(0.49376,0.05540)--(0.49052,0.05409)--(0.48788,0.05356)--(0.48679,0.05364)--(0.48578,0.05415)--(0.48369,0.05567)--(0.48273,0.05600)--(0.48043,0.05546)--(0.47637,0.05310)--(0.47371,0.05289)--(0.47166,0.05327)--(0.47107,0.05509)--(0.47151,0.06038)--(0.47124,0.06277)--(0.47073,0.06495)--(0.47062,0.06725)--(0.47154,0.06996)--(0.46851,0.07182)--(0.45812,0.07279)--(0.45857,0.07548)--(0.45828,0.07881)--(0.45679,0.08568)--(0.45633,0.08714)--(0.45510,0.08996)--(0.45462,0.09151)--(0.45393,0.09705)--(0.45350,0.09831)--(0.45194,0.09948)--(0.45031,0.09920)--(0.44867,0.09850)--(0.44715,0.09838)--(0.44583,0.09918)--(0.44379,0.10161)--(0.44243,0.10273)--(0.44095,0.10319)--(0.43773,0.10348)--(0.43634,0.10446)--(0.43569,0.10568)--(0.43467,0.11037)--(0.43223,0.11544)--(0.43045,0.12035)--(0.43043,0.12467)--(0.43331,0.12797)--(0.43056,0.12948)--(0.43012,0.13145)--(0.43039,0.13393)--(0.42976,0.13685)--(0.42823,0.13779)--(0.42412,0.13520)--(0.42170,0.13520)--(0.41400,0.13953)--(0.40999,0.14100)--(0.40579,0.14052)--(0.40602,0.14052)--(0.40620,0.14042)--(0.40634,0.14024)--(0.40645,0.13996)--(0.40352,0.13745)--(0.40201,0.13716)--(0.39996,0.13887)--(0.39939,0.13997)--(0.39870,0.14247)--(0.39826,0.14336)--(0.39704,0.14429)--(0.39460,0.14482)--(0.39339,0.14538)--(0.39014,0.14824)--(0.38857,0.14913)--(0.38195,0.15047)--(0.38062,0.15205)--(0.38133,0.15501)--(0.37532,0.15905)--(0.37385,0.15944)--(0.37214,0.15989)--(0.37255,0.15624)--(0.37309,0.15600)--(0.37392,0.15625)--(0.37481,0.15621)--(0.37553,0.15509)--(0.37569,0.15348)--(0.37515,0.15278)--(0.37176,0.15181)--(0.36759,0.15174)--(0.36570,0.15128)--(0.36260,0.14859)--(0.35880,0.14750)--(0.35503,0.14836)--(0.34841,0.15516)--(0.34727,0.15679)--(0.34649,0.15873)--(0.34594,0.16073)--(0.34526,0.16252)--(0.34414,0.16391)--(0.34318,0.16490)--(0.34287,0.16583)--(0.34322,0.16671)--(0.34420,0.16752)--(0.34424,0.16763)--(0.34427,0.16776)--(0.34425,0.16791)--(0.34421,0.16807)--(0.34276,0.16840)--(0.33997,0.16985)--(0.33861,0.17024)--(0.33750,0.17000)--(0.33662,0.16951)--(0.33556,0.16934)--(0.33396,0.17010)--(0.32905,0.17656)--(0.32886,0.17774)--(0.32965,0.17935)--(0.33079,0.17990)--(0.33579,0.18000)--(0.33857,0.18103)--(0.34302,0.18359)--(0.34514,0.18537)--(0.34616,0.18780)--(0.34460,0.19080)--(0.34352,0.19190)--(0.34332,0.19353)--(0.34355,0.19500)--(0.34472,0.19800)--(0.34496,0.19850)--(0.34504,0.19898)--(0.34500,0.19942)--(0.34475,0.19979)--(0.34289,0.20238)--(0.34233,0.20293)--(0.34059,0.20361)--(0.33991,0.20319)--(0.33963,0.20210)--(0.33903,0.20079)--(0.33589,0.19661)--(0.33409,0.19520)--(0.33169,0.19462)--(0.32249,0.19642)--(0.31756,0.19608)--(0.31445,0.19246)--(0.31391,0.19211)--(0.31336,0.19201)--(0.31281,0.19214)--(0.31229,0.19250)--(0.31140,0.19552)--(0.30918,0.19739)--(0.30690,0.19727)--(0.30571,0.19435)--(0.30436,0.19542)--(0.30368,0.19709)--(0.30352,0.19913)--(0.30370,0.20123)--(0.30324,0.20165)--(0.30282,0.20212)--(0.30247,0.20267)--(0.30220,0.20329)--(0.30411,0.20401)--(0.30200,0.20496)--(0.29716,0.20493)--(0.29472,0.20570)--(0.29114,0.20808)--(0.29016,0.20921)--(0.28957,0.21062)--(0.28898,0.21272)--(0.28867,0.21447)--(0.28891,0.21492)--(0.28771,0.21531)--(0.28456,0.21474)--(0.28314,0.21479)--(0.28184,0.21553)--(0.27950,0.21760)--(0.27839,0.21832)--(0.26789,0.22157)--(0.26550,0.22135)--(0.26400,0.22230)--(0.26115,0.22413)--(0.25880,0.22449)--(0.25596,0.22287)--(0.25590,0.22063)--(0.25906,0.21566)--(0.26015,0.21268)--(0.26082,0.20923)--(0.26175,0.20608)--(0.26362,0.20404)--(0.26610,0.20249)--(0.26792,0.20038)--(0.26893,0.19746)--(0.26898,0.19354)--(0.27027,0.19173)--(0.27042,0.18960)--(0.26972,0.18824)--(0.26848,0.18875)--(0.26743,0.19043)--(0.26697,0.19142)--(0.26637,0.19186)--(0.26485,0.19192)--(0.26455,0.19149)--(0.26452,0.18954)--(0.26424,0.18896)--(0.26331,0.18892)--(0.26142,0.18965)--(0.26051,0.18964)--(0.25925,0.18882)--(0.25746,0.18648)--(0.25646,0.18564)--(0.25530,0.18538)--(0.25403,0.18510)--(0.25196,0.18530)--(0.24990,0.18506)--(0.24752,0.18314)--(0.24558,0.18038)--(0.24439,0.17823)--(0.24307,0.17648)--(0.24072,0.17488)--(0.23740,0.17063)--(0.23318,0.17045)--(0.22890,0.17284)--(0.22544,0.17625)--(0.22370,0.17869)--(0.22279,0.18073)--(0.22180,0.18626)--(0.22078,0.18935)--(0.21776,0.19272)--(0.21652,0.19513)--(0.21510,0.19721)--(0.21263,0.20208)--(0.21122,0.20422)--(0.20899,0.20591)--(0.20445,0.20775)--(0.20270,0.21025)--(0.20249,0.21167)--(0.20262,0.21313)--(0.20260,0.21463)--(0.20193,0.21616)--(0.20096,0.21650)--(0.19627,0.21667)--(0.19505,0.21632)--(0.19429,0.21571)--(0.19353,0.21557)--(0.19225,0.21668)--(0.19180,0.21763)--(0.19062,0.22184)--(0.18730,0.22255)--(0.18644,0.22547)--(0.18735,0.22799)--(0.18938,0.22745)--(0.18921,0.22795)--(0.18877,0.22976)--(0.18940,0.22979)--(0.19037,0.23036)--(0.19094,0.23056)--(0.19258,0.23182)--(0.19442,0.23283)--(0.19601,0.23410)--(0.19688,0.23615)--(0.19852,0.23662)--(0.20178,0.23558)--(0.20355,0.23591)--(0.20457,0.23723)--(0.20676,0.23925)--(0.20768,0.24045)--(0.20739,0.24037)--(0.20741,0.24132)--(0.20759,0.24255)--(0.20778,0.24336)--(0.20826,0.24428)--(0.20856,0.24472)--(0.20858,0.24525)--(0.20823,0.24648)--(0.21150,0.24769)--(0.21068,0.25079)--(0.20228,0.26035)--(0.19920,0.26225)--(0.19588,0.26323)--(0.19217,0.26344)--(0.19084,0.26321)--(0.19003,0.26162)--(0.18921,0.25959)--(0.18790,0.25807)--(0.18616,0.25773)--(0.18463,0.25852)--(0.18148,0.26113)--(0.17718,0.26174)--(0.17059,0.25584)--(0.16620,0.25522)--(0.16601,0.25571)--(0.16595,0.25622)--(0.16606,0.25673)--(0.16628,0.25725)--(0.16780,0.25839)--(0.16811,0.26029)--(0.16764,0.26254)--(0.16684,0.26475)--(0.16561,0.26734)--(0.16429,0.26873)--(0.16261,0.26904)--(0.15645,0.26701)--(0.15362,0.26705)--(0.15297,0.26731)--(0.14818,0.27633)--(0.14789,0.27733)--(0.14772,0.27843)--(0.14519,0.27879)--(0.13661,0.27711)--(0.13555,0.27750)--(0.13483,0.27826)--(0.13357,0.28040)--(0.13245,0.28122)--(0.11235,0.28934)--(0.10808,0.28918)--(0.10707,0.28805)--(0.10602,0.28638)--(0.10483,0.28525)--(0.10466,0.28531)--(0.10341,0.28578)--(0.10282,0.28714)--(0.10298,0.28867)--(0.10334,0.29027)--(0.10339,0.29183)--(0.10231,0.29492)--(0.10083,0.29627)--(0.09903,0.29718)--(0.09707,0.29892)--(0.09608,0.30072)--(0.09501,0.30344)--(0.09421,0.30633)--(0.09402,0.30863)--(0.09488,0.31156)--(0.09585,0.31320)--(0.09604,0.31475)--(0.09454,0.31748)--(0.09190,0.31896)--(0.08611,0.31935)--(0.08528,0.32177)--(0.08494,0.32424)--(0.08378,0.32490)--(0.08242,0.32404)--(0.08139,0.32199)--(0.08139,0.32197)--(0.08097,0.32035)--(0.07981,0.32018)--(0.07842,0.32096)--(0.07729,0.32220)--(0.07651,0.32384)--(0.07143,0.32525)--(0.06959,0.32713)--(0.06923,0.32582)--(0.06872,0.32510)--(0.06719,0.32411)--(0.06661,0.32402)--(0.06602,0.32377)--(0.06548,0.32338)--(0.06499,0.32285)--(0.06498,0.32285)--(0.06531,0.32101)--(0.06688,0.31961)--(0.06831,0.31727)--(0.06817,0.31259)--(0.06781,0.31179)--(0.06675,0.31071)--(0.06643,0.31018)--(0.06629,0.30928)--(0.06632,0.30866)--(0.06641,0.30820)--(0.06623,0.30260)--(0.06594,0.30122)--(0.05435,0.30272)--(0.05061,0.30190)--(0.04844,0.30210)--(0.04862,0.30509)--(0.05020,0.30739)--(0.05405,0.31121)--(0.05986,0.32313)--(0.06503,0.33438)--(0.06544,0.33621)--(0.06563,0.33931)--(0.06589,0.34088)--(0.06650,0.34197)--(0.06732,0.34267)--(0.06805,0.34371)--(0.06838,0.34576)--(0.06871,0.34983)--(0.06979,0.35430)--(0.07139,0.35852)--(0.07331,0.36183)--(0.07292,0.36230)--(0.07258,0.36282)--(0.07336,0.36747)--(0.07206,0.37070)--(0.06992,0.37352)--(0.06823,0.37691)--(0.06801,0.37803)--(0.06773,0.38095)--(0.06774,0.38110)--(0.06751,0.38795)--(0.06737,0.39222)--(0.06834,0.39419)--(0.06742,0.39646)--(0.06567,0.39852)--(0.06417,0.39984)--(0.06195,0.40106)--(0.05731,0.40273)--(0.05509,0.40417)--(0.05509,0.40419)--(0.05306,0.40582)--(0.04662,0.40842)--(0.04454,0.41087)--(0.04431,0.41259)--(0.04416,0.41376)--(0.04516,0.41684)--(0.04729,0.41982)--(0.04866,0.42297)--(0.04874,0.42722)--(0.04800,0.43137)--(0.04687,0.43422)--(0.04224,0.43856)--(0.04015,0.44134)--(0.03936,0.44550)--(0.03922,0.44977)--(0.03795,0.45171)--(0.03622,0.45307)--(0.03473,0.45568)--(0.03464,0.45911)--(0.03611,0.46182)--(0.04218,0.46731)--(0.04416,0.46966)--(0.04587,0.47246)--(0.04690,0.47556)--(0.04783,0.48538)--(0.04846,0.48691)--(0.05084,0.49068)--(0.05135,0.49218)--(0.05162,0.49365)--(0.05202,0.49507)--(0.05286,0.49640)--(0.05482,0.49857)--(0.04866,0.50378)--(0.04780,0.50615)--(0.04865,0.51029)--(0.04989,0.51316)--(0.05041,0.51544)--(0.04911,0.51776)--(0.04917,0.51876)--(0.05125,0.52120)--(0.05148,0.52450)--(0.05011,0.52742)--(0.04742,0.52879)--(0.04465,0.52961)--(0.04140,0.53154)--(0.03934,0.53417)--(0.04011,0.53706)--(0.03833,0.54187)--(0.03642,0.54905)--(0.03626,0.55547)--(0.04164,0.55966)--(0.04424,0.56330)--(0.04610,0.56698)--(0.04575,0.56872)--(0.04402,0.57051)--(0.04501,0.57420)--(0.04811,0.57981)--(0.04521,0.58251)--(0.04469,0.58318)--(0.03939,0.58993)--(0.03367,0.59262)--(0.03110,0.59665)--(0.02930,0.59764)--(0.02834,0.59869)--(0.02541,0.60551)--(0.02401,0.60738)--(0.01413,0.61572)--(0.01156,0.61732)--(0.00863,0.62130)--(0.00673,0.62230)--(0.00457,0.62260)--(0.00257,0.62338)--(0.00096,0.62483)--(0.00000,0.62717)--(0.00274,0.62924)--(0.00388,0.63112)--(0.00449,0.63414)--(0.00440,0.63617)--(0.00360,0.64196)--(0.00313,0.64362)--(0.00320,0.64463)--(0.00814,0.64781)--(0.01426,0.65051)--(0.02028,0.65470)--(0.02346,0.65762)--(0.02495,0.66045)--(0.02531,0.66178)--(0.02881,0.67071)--(0.02876,0.67305)--(0.02832,0.67537)--(0.02814,0.67801)--(0.02855,0.68013)--(0.02937,0.68206)--(0.03042,0.68343)--(0.03148,0.68391)--(0.03225,0.68493)--(0.03509,0.68974)--(0.03390,0.69098)--(0.03358,0.69191)--(0.03219,0.69325)--(0.03166,0.69431)--(0.03147,0.69592)--(0.03160,0.69884)--(0.03153,0.70022)--(0.03108,0.70328)--(0.02967,0.71280)--(0.02921,0.72120)--(0.02741,0.72474)--(0.02663,0.72817)--(0.02454,0.73193)--(0.02406,0.73530)--(0.02423,0.73887)--(0.02488,0.74475)--(0.02480,0.74635)--(0.02321,0.75274)--(0.02198,0.75599)--(0.02254,0.75847)--(0.02272,0.76300)--(0.02273,0.76302)--(0.02273,0.76302)--(0.02367,0.76306)--(0.02544,0.76370)--(0.02678,0.76479)--(0.02699,0.76609)--(0.02539,0.76737)--(0.02636,0.76787)--(0.02670,0.76840)--(0.02643,0.76897)--(0.02553,0.76957)--(0.02615,0.77074)--(0.02703,0.76984)--(0.02832,0.76908)--(0.02945,0.76811)--(0.03035,0.76493)--(0.03164,0.76380)--(0.03417,0.76237)--(0.03595,0.76018)--(0.03694,0.75924)--(0.04111,0.75825)--(0.04613,0.75608)--(0.04783,0.75457)--(0.04915,0.75238)--(0.05192,0.74614)--(0.05293,0.74464)--(0.05347,0.74508)--(0.05374,0.74766)--(0.05427,0.74913)--(0.05514,0.75007)--(0.05652,0.75103)--(0.05649,0.75164)--(0.05650,0.75228)--(0.05665,0.75322)--(0.05564,0.75358)--(0.05320,0.75396)--(0.05214,0.75460)--(0.04963,0.76194)--(0.05047,0.76546)--(0.05325,0.77230)--(0.05428,0.77126)--(0.05525,0.77147)--(0.05632,0.77202)--(0.05781,0.77201)--(0.05739,0.77612)--(0.05794,0.77933)--(0.05877,0.78217)--(0.05928,0.78518)--(0.05863,0.78522)--(0.05828,0.78331)--(0.05745,0.78207)--(0.05503,0.77991)--(0.05556,0.77878)--(0.05474,0.77918)--(0.05410,0.77965)--(0.05360,0.78029)--(0.05314,0.78114)--(0.05366,0.78158)--(0.05459,0.78267)--(0.05524,0.78322)--(0.05410,0.78603)--(0.05191,0.78746)--(0.04143,0.78923)--(0.04004,0.79097)--(0.04145,0.79403)--(0.04116,0.79437)--(0.04108,0.79442)--(0.04106,0.79456)--(0.04092,0.79517)--(0.03988,0.79362)--(0.03912,0.79344)--(0.03835,0.79392)--(0.03723,0.79431)--(0.03638,0.79403)--(0.03568,0.79328)--(0.03497,0.79228)--(0.03409,0.79129)--(0.03690,0.79055)--(0.03840,0.78974)--(0.03914,0.78865)--(0.03883,0.78671)--(0.03774,0.78536)--(0.03650,0.78488)--(0.03568,0.78557)--(0.03297,0.78409)--(0.03165,0.78376)--(0.03027,0.78371)--(0.03020,0.78261)--(0.03148,0.78201)--(0.03138,0.78168)--(0.03013,0.78151)--(0.02924,0.78171)--(0.02778,0.78259)--(0.02668,0.78284)--(0.02516,0.78402)--(0.02103,0.79096)--(0.01887,0.79213)--(0.01865,0.79212)--(0.01864,0.79212)--(0.01805,0.79468)--(0.01652,0.79684)--(0.01745,0.79705)--(0.01827,0.79754)--(0.02018,0.80177)--(0.02170,0.80095)--(0.02268,0.80002)--(0.02551,0.79872)--(0.03487,0.79732)--(0.03869,0.79785)--(0.04251,0.79958)--(0.04965,0.80417)--(0.07544,0.81380)--(0.09740,0.81821)--(0.12435,0.82790)--(0.14162,0.83158)--(0.16051,0.83571)--(0.17920,0.84318)--(0.19449,0.84442)--(0.20507,0.84749)--(0.20383,0.84633)--(0.19907,0.84443)--(0.20106,0.84289)--(0.20944,0.84401)--(0.20948,0.84500)--(0.20883,0.84503)--(0.20888,0.84622)--(0.21338,0.84604)--(0.21336,0.84683)--(0.21327,0.84698)--(0.21309,0.84693)--(0.21278,0.84718)--(0.21306,0.84977)--(0.21090,0.84996)--(0.20570,0.84856)--(0.20901,0.85003)--(0.21301,0.85243)--(0.21911,0.85905)--(0.21783,0.85670)--(0.21564,0.85458)--(0.21477,0.85341)--(0.21564,0.85369)--(0.21737,0.85398)--(0.21889,0.85452)--(0.22026,0.85605)--(0.22362,0.85748)--(0.22488,0.85883)--(0.22385,0.86122)--(0.22199,0.86135)--(0.22112,0.86037)--(0.21969,0.85903)--(0.22394,0.86576)--(0.23534,0.87937)--(0.23530,0.87838)--(0.23474,0.87683)--(0.23568,0.87639)--(0.23721,0.87692)--(0.23846,0.87827)--(0.23913,0.88041)--(0.23861,0.88109)--(0.23737,0.88063)--(0.23593,0.87935)--(0.23799,0.88313)--(0.24137,0.88628)--(0.24790,0.89005)--(0.26757,0.89230)--(0.27526,0.89509)--(0.27926,0.89564)--(0.28283,0.89681)--(0.29208,0.90639)--(0.32057,0.91888)--(0.35596,0.92481)--(0.37187,0.92998)--(0.39674,0.93192)--(0.41205,0.93175)--(0.41415,0.93095)--(0.41750,0.92796)--(0.41874,0.92730)--(0.42069,0.92683)--(0.45243,0.90758)--(0.45823,0.90001)--(0.45917,0.89823)--(0.46012,0.89572)--(0.46020,0.89353)--(0.45855,0.89274)--(0.45814,0.89329)--(0.45769,0.89446)--(0.45712,0.89563)--(0.45542,0.89684)--(0.45428,0.89987)--(0.45275,0.90184)--(0.44834,0.90940)--(0.44747,0.90981)--(0.44420,0.91050)--(0.44337,0.91085)--(0.44249,0.91237)--(0.43985,0.91331)--(0.43430,0.91717)--(0.42884,0.91984)--(0.42797,0.92111)--(0.42496,0.92347)--(0.42417,0.92385)--(0.42303,0.92335)--(0.42243,0.92252)--(0.42107,0.91793)--(0.42088,0.91687)--(0.42086,0.91451)--(0.42136,0.91391)--(0.42358,0.91289)--(0.42408,0.91228)--(0.42519,0.91031)--(0.42653,0.90732)--(0.42647,0.90608)--(0.42611,0.90502)--(0.42613,0.90396)--(0.42720,0.90281)--(0.42661,0.89966)--(0.42767,0.89831)--(0.42945,0.89868)--(0.43102,0.90067)--(0.43045,0.89800)--(0.43119,0.89586)--(0.43234,0.89404)--(0.43466,0.88669)--(0.43538,0.88514)--(0.43499,0.88144)--(0.43503,0.87506)--(0.43561,0.86902)--(0.43690,0.86631)--(0.43838,0.86581)--(0.44124,0.86359)--(0.44270,0.86308)--(0.44396,0.86294)--(0.44553,0.86245)--(0.44687,0.86156)--(0.44744,0.86016)--(0.44902,0.85793)--(0.46488,0.85275)--(0.46839,0.85248)--(0.47132,0.85410)--(0.47213,0.85328)--(0.47323,0.85267)--(0.47582,0.85189)--cycle; \end{tikzpicture}
    			\caption{Obrys granic Polski\mysrc}\label{fig:poland}
    	    \end{center}
    	\end{figure}
    	
    	Natomiast otoczka wypukła obrysu granic Polski zawiera tylko 28 krawędzi i została przedstawiona na rysunku \ref{fig:poland_convex_hull}.
    	  \begin{figure}[h!]
        	\begin{center}
    			\begin{tikzpicture}[scale=5.5] \draw (0.39674,0.93192)--(0.41205,0.93175)--(0.83852,0.87059)--(0.84844,0.86831)--(0.88310,0.84684)--(0.88692,0.84397)--(0.89476,0.83673)--(0.89647,0.83452)--(0.89800,0.83138)--(0.89926,0.82784)--(0.94964,0.67580)--(0.95409,0.64486)--(1.00000,0.30792)--(0.99990,0.25687)--(0.99980,0.25650)--(0.88420,0.00000)--(0.58656,0.01925)--(0.55860,0.02132)--(0.55592,0.02227)--(0.47166,0.05327)--(0.23318,0.17045)--(0.22890,0.17284)--(0.04844,0.30210)--(0.00000,0.62717)--(0.01652,0.79684)--(0.02018,0.80177)--(0.32057,0.91888)--(0.37187,0.92998)--cycle; \end{tikzpicture}
    			\caption{Otoczka wypukła obrysu granic Polski\mysrc}\label{fig:poland_convex_hull}
    	    \end{center}
    	\end{figure}
    	
    	W związku z tym, że złożoność obliczeniowa algorytmu sprawdzającego czy punkt należy do wnętrza danego wielokąta jest liniowa \cite{complex}, możliwe jest dużo szybsze wykluczenie punktów znajdujących się poza otoczką. 
		\section{Grafika komputerowa}
		Obiekty wykorzystywane w grafice komputerowej często mogą charakteryzować się skomplikowanymi kształtami. Im bardziej skomplikowany kształt, tym więcej mocy obliczeniowej potrzeba w celu wykonania danej operacji na tym kształcie. W niektórych przypadkach do uproszczenia graficznego danego obiektu używa się otoczki wypukłej jego kształtu. Ze względu na fakt, iż otoczka wypukła wielokąta zawsze będzie miała liczbę wierzchołków mniejszą lub równą liczbie wierzchołków samego wielokąta, powstała otoczka może posłużyć do wykonania mniejszej liczby obliczeń przy wykonywaniu operacji na danym obiekcie.
		\section{Detekcja obiektów}
		Proces analizowania obrazu przez ludzkie organy jest wyjątkowo złożony złożony \cite{BIEDERMAN198529}. W celu imitacji tego procesu w sposób algorytmiczny, zostały przeprowadzone liczne badania naukowe \cite{detection}. Dzięki temu zaproponowano wiele podejść rozwiązania tego problemu. Istotnym aspektem ze względów praktycznych tych pomysłów jest złożoność obliczeniowa ich implementacji.
		
		Jednym z działań wykonywanych przez ludzkie zdolności poznawcze po zarejestrowaniu obrazu jest wybranie miejsca w obrazie o największym znaczeniu (z ludzkiego punktu widzenia) \cite{s23146450}. Niektóre z rozwiązań tego problemu w informatyce opierały się nie tylko na samych danych pochodzących z obrazu, ale również tworząc modele z uwzględnieniem danych statycznych z badań dotyczących ludzkich umiejętności poznawczych \cite{Borji_Cheng_Hou_Jiang_Li_2019}. Jednak spora część z nich opiera się na przetwarzaniu samych danych pozyskanych z obrazów, bez potrzeby dostarczania danych statystycznych. Aby komputer był w stanie wykonać tego typu operację potrzebny jest szereg złożonych obliczeniowo operacji \cite{detection, Borji_Cheng_Hou_Jiang_Li_2019}. 
		
		W celu zmniejszenia kosztu obliczeniowego tego typu procesu można wykonać preprocesowanie danych wejściowych. Jednym z nich może być wykorzystanie algorytmu wyznaczającego otoczkę wypukłą tak zwanych punktów kluczowych \cite{detection}. W celu znalezienia lokalizacji punktów kluczowych można skorzystać z wyznaczania lokalnej energii piksela za pomocą wzoru (\ref{eq:harris}) na wielopoziomową energię Harrisa \cite{detection, harris}. 
		\begin{align}\label{eq:harris}  
			\textbf{E}_{WPH}\left(p\right) = \sum_{l=0}^{L}\textbf{R}\left(l\right)\left(p\right)	 		
        \end{align}\\[-5 pt]
        Gdzie poszczególne symbole oznaczają:\\[10 pt]
       \begin{tabular}{rcl}
    	$p$ & -- & rozpatrywany piksel obrazu\\
		$\textbf{R}\left(l\right)\left(p\right)$ & -- &  \begin{minipage}[t]{123 mm} energia piksela $p$ wyznaczona ze wzoru Harrisa (\ref{eq:harris_single}) dla $l$ - tego poziomu obrazu \end{minipage} \\[20 pt] 
		$L$ & -- & \begin{minipage}[t]{123 mm}
		liczba poziomów dla których ma być wyznaczona suma energii, przyjęte jako $L=5$ \cite{detection}
		\end{minipage}
		
    	\end{tabular}\\[15 pt]
    	W celu wyznaczenia wielopoziomowej Harrisa dla danego piksela $p$ musimy znać też wzór (\ref{eq:harris_single}) na energię piksela $p$ dla pojedynczego poziomu $l$.
    	\begin{align}\label{eq:harris_single}  
		\textbf{R}\left(l\right)\left(p\right) =
		\left|\textbf{M}\left(l\right)\left(p\right)\right| + k \cdot \mathrm{tr}\left(\textbf{M}\left(l\right)\left(p\right)\right)
        \end{align}
Gdzie poszczególne symbole oznaczają:\\[10 pt]
        \begin{tabular}{rcl}
    	$k$ & -- & współczynnik skalarny\\
    	$\textbf{M}\left(l\right)\left(p\right)$ & -- & 
    	\begin{minipage}[t]{121 mm}
    	macierz autokorelacji dla danego piksela $p$ i poziomu obrazu $l$ wyznaczana ze wzoru \ref{eq:autocorelate} \cite{harrispptx, harris, detection}
    	\end{minipage}\\[20 pt]
    	$\mathrm{tr}\left(\textbf{A}\right)$ & -- & 
    	\begin{minipage}[t]{121 mm}
    	ślad macierzy $\textbf{A}$, czyli suma elementów należących do głównej przekątnej macierzy $\textbf{A}$, obliczany według wzoru \cite{matrixtrace}
    	\end{minipage}
    	\end{tabular}\\[10 pt]
    	\begin{align}\label{eq:autocorelate}  
		\textbf{M}\left(l\right)\left(p\right) =
		\textbf{G}_{\sigma}
		\cdot\left[
		\begin{matrix}
			\textbf{I}_x^2\left(l\right)\left(p\right) & \textbf{I}_x\left(l\right)\left(p\right) \cdot \textbf{I}_y\left(l\right)\left(p\right)\\
			\textbf{I}_y\left(l\right)\left(p\right) \cdot \textbf{I}_x\left(l\right)\left(p\right) & \textbf{I}_y^2\left(l\right)\left(p\right)
		\end{matrix}
		\right]
        \end{align}\\[10 pt]
        Gdzie poszczególne symbole oznaczają:\\[10 pt]
    	\begin{tabular}{rcl}
    	$G_{\sigma}$ & -- & 
    	\begin{minipage}[t]{123 mm}
    	prostokątna macierz o liczbie kolumn 2 stanowiąca okno czasowe Gaussa o odchyleniu standardowym $\sigma$
    	\end{minipage}\\[5 pt]
    	$\textbf{M}\left(l\right)\left(p\right)$ & -- & 
    	\begin{minipage}[t]{123 mm}
    	macierz autokorelacji dla danego piksela $p$ i poziomu obrazu $l$ wyznaczana ze wzoru \ref{eq:autocorelate} \cite{harrispptx, harris, detection}
    	\end{minipage}
    	\end{tabular}\\[10 pt]
		
        \section{Wyznaczanie obwiedni sygnału}
        Sygnał to zapis informacji zmieniającej się w zależności od czasu. Taką informacją może być natężenie/napięcie prądu, natężenie pola elektromagnetycznego, dźwięk utworu muzycznego itd. W przypadku, jeśli zmienną informację da się  reprezentować w sposób liczbowy, możliwe jest graficzne przedstawienie sygnału. Zapisanie informacji w ten sposób może być łatwiejsze do interpretacji przez człowieka. Jednak w przypadku, gdy mamy do czynienia ze skomplikowanym przebiegiem, na
        
        Algorytm wyznaczający otoczkę wypukłą wielokąta prostego może posłużyć znajdowaniu obwiedni sygnału.
        
   \chapter{Implementacja w języku Scala}
   		\section{Pojęcia ogólne}
   		W tej sekcji przedstawione zostaną listingi zawierające kod w języku Scala definiujący pojęcia potrzebne do implementacji poszczególnych algorytmów wyznaczających otoczkę wypukłą.
   		
   		Klasa \texttt{Point} (Listing \ref{lst:point}: linia 4) posłuży jako podstawowa klasa reprezentująca punkt w przestrzeni dwuwymiarowej. W celu łatwiejszego manipulowania danymi. Zostały w jej ramach zaimplementowane 3 metody pomocnicze.
   		
   		Odejmowanie (Listing \ref{lst:point}: linia 7) - zaimplementowane jako odejmowanie od siebie odpowiadających sobie współrzędnych dwu punktów.
   		
   		Dodawanie (Listing \ref{lst:point}: linia 8) - dodawanie do siebie odpowiadających sobie współrzędnych dwu punktów.
   		
   		Dzielenie przez skalar (Listing \ref{lst:point}: linia 9) - zaimplementowane jako dzielenie obu współrzędnych przez wskazaną liczbę. Zaimplementowane dzięki użyciu domniemanemu argumentowi \scala{numeric: Numeric[T]}. W języku Scala tego typu konstrukcja umożliwia korzystanie z metody \scala{/} w tedy i tylko wtedy jeśli zaimportowany zostanie do aktualnego kontekstu obiekt typu \scala{Numeric[T]}. Domyślnie typ ten jest zaimplementowany dla \scala{T} będącym jednym z podstawowych typów liczbowych: \scala{Float}, \scala{Int}, \scala{Double} etc. Więc w domyśle tego typu konstrukcja służy niejako ,,udowodnieniu'', że wskazywany argument jest w istocie liczbą.
   		
   		\lstinputlisting[language=scala, caption={Point.scala\mysrc}, label={lst:point}]{\listingsDirectory/Point.scala}
   		Kolejnym istotnym elementem wykorzystanym w implementacji algorytmów wyznaczających otoczkę wypukłą zbioru punktów na płaszczyźnie są metody wyliczające określone właściwości punktów.
   		
   		Metoda \scala{calculateCentroid(points: Points): Points} (Listing \ref{lst:pointsutils}: linia 11) wyznacza centroid zbioru punktów \scala{points}.
   		
   		Metoda \scala{distanceFromCenterSquared(p: Point): Double} (Listing \ref{lst:pointsutils}: linia 17) wyznacza kwadrat odległości wskazanego punktu od środka układu współrzędnych.
   		
   		Na podstawie wyżej wymienionej metody zostały zaimplementowane trzy metody związane z odległością między punktami.
   		
   		Metoda \scala{distanceSquared(a: Point, b: Point): Double} (Listing \ref{lst:pointsutils}: linia 13) wyznacza kwadrat odległości pomiędzy dwoma punktami.
   		
   		Metoda \scala{distanceFromCenter(p: Point): Double} (Listing \ref{lst:pointsutils}: linia 19) wyznacza odległość od środka układu współrzędnych.
   		
   		Metoda \scala{distance(a: Point, b: Point): Double} (Listing \ref{lst:pointsutils}: linia 15) wyznacza odległość pomiędzy dwoma punktami.

		W ramach obiektu \scala{PointsUtils} (Listing \ref{lst:pointsutils}: linia 5) zostały również zaimplementowane metody służące do wyznaczania wartości umożliwiających sortowanie punktów względem kąta występującego pomiędzy horyzontalną osią układu współrzędnych, a wektorem łączącym początek układu współrzędnych z punktem (kąt $\theta$ zaznaczony na rysunku \ref{fig:axis}).
		
		Metoda \scala{phase(p: Point): Double} (Listing \ref{lst:pointsutils}: linia 21) wyznacza dokładną wartość szukanego kąta. Matematycznie ta funkcja została zapisana we wzorze \ref{eq:theta}.
		
		Metoda \scala{alpha(p: Point): Double} (Listing \ref{lst:pointsutils}: linia 32) wyznacza wartość umożliwiającą porównanie kąta dla wskazanego punktu z innymi punktami. Matematycznie ta funkcja została zapisana we wzorze \ref{eq:alpha1}.
   		
   		\lstinputlisting[language=scala, caption={PointsUtils.scala}, label={lst:pointsutils}]{\listingsDirectory/PointsUtils.scala}
   		\newpage
   		
   		W celu zapewnienia odpowiedniego sortowania punktów został utworzony \scala{trait OrientationOrdering}. Język Scala umożliwia wykorzystywanie zaimplementowanie własnego obiektu \scala{Ordering[T]}, który będzie następnie używany w metodzie \scala{sorted} dla kolekcji \scala{C[T]}.
   		
   	 	Trait \scala{OrientationOrdering} (Listing \ref{lst:order}: linia 6) ma zaimplementowaną metodę \scala{compare(x: Point, y: Point)} (Listing \ref{lst:order}: linia 7) w taki sposób, aby w pierwszej kolejności porównywane były wartości funkcji reprezentujących fazę tych punktów, a następnie porównywane wartości funkcji reprezentujących odległość od środka układu współrzędnych.
   	 	
   	 	W ramach \scala{OrientationOrdering} zostały zaimplementowane dwa obiekty.
   	 	
   	 	Obiekt \scala{Exact} (Listing \ref{lst:order}: linia 19) służy do dokładnego porównania fazy punktów, wyznaczanej ze wzoru \ref{eq:theta} oraz odległości od środka układu współrzędnych. 
   	 	
   	 	Obiekt \scala{Indicator} (Listing \ref{lst:order}: linia 24) służy do porównywania fazy punktów za pomocą funkcji wyznaczanej ze wzoru \ref{eq:alpha1}.
   		
   		\lstinputlisting[language=scala,caption={OrientationOrdering.scala},captionpos=b, label={lst:order}]{\listingsDirectory/OrientationOrdering.scala}
   		\newpage
   		\lstinputlisting[language=scala,caption={ConvexHullAlgorithm.scala},captionpos=b]{\listingsDirectory/ConvexHullAlgorithm.scala}
   		\newpage
   		\lstinputlisting[language=scala,caption={AlgorithmTest.scala},captionpos=b]{\listingsDirectory/AlgorithmTest.scala}
   		\newpage
   		\section{Algorytm Grahama}
   		\lstinputlisting[language=scala,caption={Graham.scala},captionpos=b]{\listingsDirectory/Graham.scala}
   		\newpage
   		\lstinputlisting[language=scala,caption={PointsCycle.scala},captionpos=b]{\listingsDirectory/PointsCycle.scala}
   		\newpage
   		\lstinputlisting[language=scala,caption={GrahamAlgorithmTest.scala},captionpos=b]{\listingsDirectory/GrahamAlgorithmTest.scala}
   		
   		\section{Algorytm Jarvisa}
   		Dzięki wykorzystaniu pattern matchingu dostępnego w Scali, możliwe jest przejrzyste zaimplementowanie algorytmu Jarvisa, który, jak widać jest dużo mniej skomplikowany od algorytmu Grahama. Mniejsze skomplikowanie implementacji wiąże się jednak z dużo większą złożonością obliczeniową.
        \lstinputlisting[language=scala,caption={Jarvis.scala},captionpos=b]{\listingsDirectory/Jarvis.scala}
        \lstinputlisting[language=scala,caption={JarvisAlgorithmTest.scala},captionpos=b]{\listingsDirectory/JarvisAlgorithmTest.scala}
    \chapter{Dynamiczna otoczka wypukła}
    	\section{Algorytm}
    	\section{Implementacja w języku Scala} 
	\chapter{Podsumowanie} 
    \printbibliography[heading=bibintoc]
\end{document}

    
    Z powyższych obliczeń wynika, że we wszystkich przedziałach z zależności
    \ref{eq:xytotheta1}, 
     \ref{eq:xytotheta2}, 
     \ref{eq:xytotheta3} oraz
     \ref{eq:xytotheta4} funkcja $\gamma\left(\theta\right)$ ma dodatnią pochodną, a tym samym jest rosnąca we wszystkich przedziałach, oraz każda z wartości osiąganych w przedziale jest większa od każdej wartości z poprzedniego przedziału, a tym samym funkcja $\alpha\left(p\right)$ spełnia warunki \ref{eq:condition1} i \ref{eq:condition2}.
     
     Postać funkcji $\gamma\left(\theta\right)$ może być opisana wzorem \ref{eq:gammaeq} a wykres jej przebiegu, przedstawiony na rysunku \ref{fig:gamma} wyraźnie potwierdza jej rosnącą monotoniczność we wszystkich przedziałach.
        \begin{align}\label{eq:gammaeq}
        	\gamma\left(
        		\theta
        	\right) = 
        		\left\{
        			\begin{array}{lcl}
        				\frac{
        					\sin\theta
        				}{
							\sin\theta + \cos\theta        			
        				} & 
        					\text{dla} &
        					\theta \in \left\langle
        						0;
        						\frac{
        							\pi
        						}{
									2        						
        						}
        					\right\rangle\\
        				2
        				- \frac{
        					\sin\theta
        				}{
							\sin\theta - \cos\theta        			
        				} & 
        					\text{dla} &
        					\theta \in \left(
        						\frac{
        							\pi
        						}{
									2        						
        						};
        						\pi
        					\right\rangle\\
        				2
        				+ \frac{
        					\sin\theta
        				}{
							\sin\theta + \cos\theta        			
        				} & 
        					\text{dla} &
        					\theta \in \left(
        						\pi;
        						\frac{
        							3\pi
        						}{
									2        						
        						}
        					\right)\\
        				4
        				- \frac{
        					\sin\theta
        				}{
							\sin\theta + \cos\theta        			
        				} & 
        					\text{dla} &
        					\theta \in \left\langle
        						\frac{
        							3\pi
        						}{
									2        						
        						};
        						2\pi
        					\right)\\
        			\end{array}
        		\right.      		
        \end{align}
        \begin{figure}[h!]
        	\begin{center}
    			\begin{tikzpicture}
	
	\draw 
		[help lines, very thin] 
		(0,0) 
		grid [xstep = pi/2]
		(2*pi - 0.001,4- 0.001);
		
	\tzaxes 
		[
			-arcs,
			line width=0.2mm
		]
		(2*pi,4)
		{$\theta$}
		{$\gamma\left(\theta\right)$}
		
	\tzticksx
		(-5pt:5pt)
		{
			0,
			pi*.5/$\frac{\pi}{2}$,
			pi/$\pi$,
			pi*1.5/$\frac{3\pi}{2}$
		}
		[below=5pt]
		
	\tzticksy
		(-5pt:5pt)
		{	
			0,
			1,
			2,
			3
		}
		[left=5pt]
	\tzfn
		[
			black,
			line width=0.25mm
		]
		{sin(\x r)/(sin(\x r) + cos(\x r))}
		[0:.5*pi] 
	\tzfn
		[
			black,
			line width=0.25mm
		]
		{2 - sin(\x r)/(sin(\x r) - cos(\x r))}
		[.5*pi:pi] 
	\tzfn
		[
			black,
			line width=0.25mm
		]
		{2 + sin(\x r)/(sin(\x r) + cos(\x r))}
		[pi:1.5*pi] 
	\tzfn
		[
			black,
			line width=0.25mm
		]
		{4 + sin(\x r)/(cos(\x r) - sin(\x r))}
		[1.5*pi:2*pi] 
\end{tikzpicture}
    			\caption{Wykres funkcji $\gamma\left(\theta\right)$\mysrc}\label{fig:gamma}
    	\end{center}
    	\end{figure}\vspace{-8 mm}\\
       
              
        \subsection{Algorytm Grahama}
        \null
        \begin{figure}[h!]
    			\begin{tikzpicture}[node distance=1.4cm] 
	\node (0,0) (start) {};
	\path[ellipse_block = 
    		reference: start 
    		text: START
	];
	\node[below of = start_block] (input) {};
	\block
		{input}
		{Dane wejściowe:\\
		zbiór punktów 
			$P = \left\{
				p_1,
				p_2,
				...,
				p_n
			\right\} \subset \mathbb{R}^2$, dla każdego z punktów w zbiorze określone są funkcje:\\
			 $x\left(p\right)$ (oznaczająca wartość odciętej punktu $p$)\\
			 $y\left(p\right)$ (oznaczająca wartość rzędnej punktu $p$)\\
		\vspace{.8cm}
		}
	\node[below of = input_south] (centroid) {};
	\block
		{centroid}
		{
		Wyznaczyć punkt $O$, stanowiący centroid zbioru $P$.
		Następnie dla każdego punktu $p$ ze zbioru $P$ wykonać przekształcenia:
		 $x\left(p\right) := x\left(p\right) -  x\left(O\right)$\\
		 $y\left(p\right) := y\left(p\right) -  y\left(O\right)$\\
		 Następnie przypisać współrzędne punktu $O$:\\
		 $x\left(O\right) := 0$\\
		 $y\left(O\right) := 0$\\
		(Tak aby punkt $O$ znajdował się na początku układu współrzędnych.)
		}



	\node[below of = centroid_south] (sort) {};
	\block
		{sort}
		{
			Posortować leksykograficznie wszystkie punkty zbioru $P$ według porządku $\left(\theta\left(p\right), \left|R\right|\left(p\right)\right)$, gdzie $\theta\left(p\right)$ to wartość kąta wektora wiodącego punktu $p$, natomiast $\left|R\right|\left(p\right)$ to długość tego wektora.
		}
	\node[below of = sort_south] (cykl) {};
	\block
		{cykl}
		{
			Z otrzymanego, uporządkowanego zbioru $P$ utworzyć listę pozwalającą na wyznaczenie następnika punktu $p$ w liście - poprzez zastosowanie funkcji $next\left(p\right)$ oraz znalezienie poprzednika punktu $p$ w liście poprzez zastosowanie funkcji $prev\left(p\right)$.\\
	Funkcja $next$ dla ostatniego punktu w uporządkowanym zbiorze będzie wskazywała na pierwszy punkt tego zbioru, natomiast funkcja $prev$ dla pierwszego punktu uporządkowanego zbioru będzie wskazywała na ostatni punkt tego zbioru.
		}
	\node[below of =cykl_south] (punkts) {};
	\block
		{punkts}
		{
			Wyznacz punkt $s$ będący punktem o najmniejszej wartości odciętej ze zbioru punktów o najmniejszej wartości rzędnej ze zbioru $P$.
		}
		
	\node[right of =start, node distance = 8 cm] (q) {};
	\smallblock
		{q}
		{
			$q:=s$
		}
	\node[below of =q] (isequal) {};
	\chooseblock
		{isequal}
		{
			\begin{small}
			$next\left(q\right) \neq s$
			\end{small}
		}
	\node[below of =isequal_south] (inside) {};
	\chooseblock
		{inside}
		{
			$next\left(q\right) \in \Delta\left(
			 	O,
			 	q,
			 	next\left(
			 		next\left(
			 			q
			 		\right)
			 	\right)
			 \right)$
		}
	\node[below of =inside_south] (remove) {};
	\smallblock
		{remove}
		{
			$next\left(q\right) := next\left(
				next\left(
					q
				\right)
			\right)$\\
			$prev\left(next\left(q\right)\right) := q$
			
			
		}
	
	\node[below of =remove_south] (qequalss) {};
	\chooseblock
		{qequalss}
		{	
		\begin{small}
			$q\neq s$
		\end{small}	
		}
	
	\node[below of =qequalss_south] (prev) {};
	\smallblock
		{prev}
		{	
		$q:=prev\left(q\right)$
		}
	
	\node[below of =prev_south] (next) {};
	\smallblock
		{next}
		{	
		$q:=next\left(q\right)$
		}
	
	\node[below of =next_south] (output) {};
	\block
		{output}
		{	
			Wynikiem działania algorytmu jest lista zaczynająca się od punktu $s$ - zawiera ona uporządkowane wierzchołki otoczki wypukłej zbioru punktów $P$.
		}
	
	\node[below of =output_south] (stop) {};
	\path[ellipse_block = 
    		reference: stop
    		text: STOP
	];
	
	\draw
		[-arcs]
		(start_block)--
		(input_block);
		
	\draw
		[-arcs]
		(input_block)--
		(centroid_block);
	
	\draw
		[-arcs]
		(centroid_block)--
		(sort_block);
	
	\draw
		[-arcs]
		(sort_block)--
		(cykl_block);
	
	\draw
		[-arcs]
		(cykl_block)--
		(punkts_block);
	
	
	\draw [-arcs] 
		(punkts_block.east) --++
		(.5,0) |-
		(q_block.west);
	
	\draw [-arcs] 
		(q_block) -- node (centerq) {}
		(isequal_block);
		
	\draw [-arcs] 
		(isequal_block) --
		(inside_block);
	
	\draw [-arcs] 
		(inside_block) --
		(remove_block);
	
	\draw [-arcs] 
		(remove_block) --
		(qequalss_block);
		
	\draw [-arcs] 
		(qequalss_block) --
		(prev_block);
	
	\draw [-arcs] 
		(isequal_block.west) --++
		(-2,0) |-
		(output_block.west);
	
	\draw [-arcs] 
		(inside_block.west) --++
		(-1.07,0) |-
		(next_block.west);
		
	\draw [-arcs] 
		(inside_block.west) --++
		(-1.07,0) |-
		(next_block.west);
	
	
		
	
	
	\draw [-arcs]
		(prev_block.south) --++
		(0,-1) node (prev_below) {};
		
	\draw [-arcs]
		(qequalss_block.west) --++
		(-1.95,0)  |-
		(prev_below.center) --++
		(3.3,0)
		node (prev_right) {};
	
	\draw [-arcs]
		(next_block.east) -|
		(prev_right.center) |-
		(centerq.center);
	
	\draw [-arcs]
		(output_block) --
		(stop_block);
	
	
	
\end{tikzpicture}
    			\caption{Algorytm Grahama}\label{fig:graham}
    	\end{figure}
       \newpage
        \subsection{Algorytm Jarvisa}
        
        
        \section{Otoczka wypukła wielokąta prostego}
        \cite{convexhullsimplepolygon}
        \section{Redukcja zbioru punktów do wielokąta prostego}
    \chapter{Zastosowania} 
%    as first chapter
    Istotnym, jeśli nie najistotniejszym zagadnieniem dotyczącym otoczek wypukłych są ich zastosowania. Już od kilkudziesięciu lat problem ten temat znajduje swoje użycie w wielu dziedzinach kombinatoryki i informatyki. Dla przykładu problem sortowania elementów liczbowych listy można sprowadzić do problemu znalezienia otoczki wypukłej. W związku z tym rozwiązanie tego problemu może pomóc w rozwiązaniu innych problemów w bardziej symboliczny i graficzny sposób.
        \section{Generalizacja kartograficzna}
        W celu jak najbardziej informatywnego i niezłożonego przedstawienia danych geograficznych w sposób graficzny potrzebne jest zastosowanie algorytmu generalizującego informacje. Taka generalizacja w kartografii zaczęła być coraz bardziej potrzebna, kiedy w dobie cyfrowego przetwarzania danych mapy zaczęły być przechowywane cyfrowo. Przed tym generalizacją zajmowali się manualnie ludzie, którzy wyposażeni w ludzkie zmysły wykrywania wzorów i kształtów byli w stanie z łatwością narysować obrys zbioru drzew na płaszczyźnie i opisać go jako teren leśny. W przypadku przechowywania danych geograficznych w postaci cyfrowej zaistniała potrzeba wykonywania tej samej operacji w sposób zalgorytmizowany. \cite{cartography}
        
        Do celu generalizacji można by wykorzystać otoczkę wypukłą w sposób bezpośredni \cite{LAURINI2017139}, jednak taki obrys terenu może zawierać w sobie przestrzenie pozbawione obiektów, co byłoby niepomocne dla ludzkiej interpretacji przedstawianego mu obszaru. 
        
        Należy też zwrócić uwagę, że otoczka wypukła wielokąta prostego zawiera w sobie wszystkie punkty należące do wewnętrznej powierzchni tego wielokąta. \cite{convexhullsimplepolygon} W związku z tym jeśli udowodni się, że dany punkt nie należy do wnętrza otoczki wypukłej będzie to implikowało, że punkt ten nie należy do wnętrza wielokąta prostego. Taka wiedza okazuje się być przydatna jeśli uwzględnimy dodatkowo fakt, że określanie czy punkt leży wewnątrz wielokąta wypukłego jest dużo łatwiejsze niż dla pozostałych wielokątów. \cite{heckbert1994graphics}
        
        Dla przykładu rozpatrując obrys granic Polski (Rysunek \ref{fig:poland}; źródło współrzędnych obrysu: \cite{datahubCountryPolygons}; współrzędne przedstawione zgodnie z projekcją EPSG:2180 \cite{epsg2180} przy użyciu biblioteki GeoTools \cite{geotools}) można zauważyć, że jego kształt jest bardzo skomplikowany. W istocie zawiera on 1336 krawędzi.
        \begin{figure}[h!]
        	\begin{center}
    			\begin{tikzpicture}[scale=5.5] \draw (0.47582,0.85189)--(0.51112,0.85726)--(0.51889,0.86108)--(0.52258,0.86197)--(0.52555,0.86337)--(0.53129,0.86945)--(0.53288,0.87017)--(0.53489,0.86856)--(0.53955,0.86699)--(0.54689,0.86668)--(0.56630,0.86588)--(0.60275,0.86453)--(0.62993,0.86366)--(0.65713,0.86291)--(0.69084,0.86213)--(0.73168,0.86141)--(0.75791,0.86108)--(0.78416,0.86086)--(0.80617,0.86075)--(0.82391,0.86071)--(0.83028,0.86320)--(0.83197,0.86474)--(0.83411,0.86928)--(0.83422,0.86953)--(0.83651,0.87075)--(0.83852,0.87059)--(0.84432,0.86799)--(0.84535,0.86791)--(0.84746,0.86848)--(0.84844,0.86831)--(0.84949,0.86735)--(0.85010,0.86612)--(0.85059,0.86489)--(0.85124,0.86389)--(0.85234,0.86323)--(0.85559,0.86303)--(0.85624,0.86217)--(0.85598,0.86062)--(0.85554,0.85876)--(0.85563,0.85706)--(0.85746,0.85478)--(0.85962,0.85492)--(0.86182,0.85600)--(0.86369,0.85654)--(0.86545,0.85581)--(0.87159,0.85125)--(0.87532,0.84926)--(0.88310,0.84684)--(0.88508,0.84535)--(0.88692,0.84397)--(0.88791,0.84261)--(0.88983,0.83943)--(0.89078,0.83823)--(0.89180,0.83763)--(0.89386,0.83720)--(0.89476,0.83673)--(0.89647,0.83452)--(0.89800,0.83138)--(0.89926,0.82784)--(0.90020,0.82445)--(0.90206,0.81702)--(0.90232,0.81339)--(0.90107,0.81083)--(0.89998,0.80980)--(0.89951,0.80913)--(0.89909,0.80839)--(0.89914,0.80666)--(0.89964,0.80522)--(0.90048,0.80411)--(0.90154,0.80334)--(0.90215,0.80265)--(0.90247,0.80186)--(0.90247,0.80097)--(0.90214,0.80001)--(0.90255,0.79530)--(0.90379,0.79142)--(0.90528,0.78779)--(0.90652,0.78380)--(0.90794,0.77529)--(0.90916,0.77204)--(0.91169,0.76879)--(0.91169,0.76878)--(0.91170,0.76878)--(0.91256,0.75891)--(0.91557,0.74782)--(0.92534,0.72330)--(0.93050,0.71416)--(0.93277,0.70920)--(0.93764,0.69423)--(0.93965,0.68976)--(0.94155,0.68754)--(0.94265,0.68533)--(0.94356,0.68304)--(0.94488,0.68063)--(0.94674,0.67884)--(0.94849,0.67767)--(0.94964,0.67580)--(0.94969,0.67188)--(0.94897,0.66950)--(0.94793,0.66748)--(0.94715,0.66520)--(0.94724,0.66205)--(0.94825,0.65938)--(0.95145,0.65580)--(0.95273,0.65349)--(0.95297,0.65226)--(0.95288,0.65031)--(0.95296,0.64929)--(0.95409,0.64486)--(0.95422,0.64357)--(0.95399,0.64256)--(0.95307,0.64031)--(0.95288,0.63902)--(0.95309,0.63730)--(0.95349,0.63636)--(0.95394,0.63558)--(0.95427,0.63436)--(0.95643,0.61490)--(0.95696,0.61008)--(0.95609,0.60310)--(0.95252,0.59804)--(0.94011,0.58829)--(0.92402,0.58256)--(0.91565,0.57694)--(0.90743,0.56920)--(0.89656,0.55004)--(0.89290,0.54497)--(0.89125,0.54203)--(0.88721,0.53241)--(0.88752,0.53235)--(0.88828,0.53178)--(0.88905,0.53121)--(0.89022,0.52968)--(0.89066,0.52758)--(0.89004,0.52467)--(0.89235,0.52338)--(0.89958,0.52187)--(0.90107,0.52256)--(0.90245,0.52132)--(0.90873,0.51938)--(0.91082,0.51929)--(0.91088,0.51829)--(0.91024,0.51676)--(0.91055,0.51651)--(0.91193,0.51715)--(0.91321,0.51667)--(0.91415,0.51581)--(0.91511,0.51569)--(0.91636,0.51742)--(0.91680,0.51673)--(0.91783,0.51596)--(0.91853,0.51524)--(0.92009,0.51695)--(0.92054,0.51577)--(0.92039,0.51382)--(0.92002,0.51323)--(0.92312,0.50790)--(0.92508,0.50741)--(0.92978,0.50785)--(0.93171,0.50684)--(0.93245,0.50563)--(0.93303,0.50435)--(0.93369,0.50337)--(0.93465,0.50301)--(0.93592,0.50224)--(0.93644,0.50041)--(0.93666,0.49856)--(0.93761,0.49661)--(0.93940,0.49062)--(0.94072,0.48794)--(0.93923,0.48668)--(0.93806,0.48454)--(0.93646,0.47995)--(0.93601,0.47694)--(0.93626,0.47172)--(0.93573,0.46986)--(0.93613,0.46973)--(0.93631,0.46973)--(0.93639,0.46954)--(0.93649,0.46881)--(0.93534,0.46806)--(0.93515,0.46454)--(0.93412,0.46314)--(0.93587,0.46032)--(0.93787,0.45795)--(0.93699,0.45414)--(0.93519,0.45211)--(0.93317,0.45069)--(0.93160,0.44863)--(0.93127,0.44745)--(0.93098,0.44570)--(0.93064,0.44195)--(0.93106,0.44114)--(0.93180,0.44005)--(0.93209,0.43906)--(0.93120,0.43856)--(0.93083,0.43809)--(0.93122,0.43707)--(0.93190,0.43600)--(0.93244,0.43533)--(0.93125,0.43366)--(0.93066,0.43325)--(0.92977,0.43307)--(0.92985,0.43186)--(0.93106,0.43069)--(0.93129,0.42914)--(0.93104,0.42476)--(0.93153,0.42247)--(0.93254,0.42028)--(0.93498,0.41663)--(0.93430,0.41652)--(0.93361,0.41655)--(0.93501,0.41411)--(0.93665,0.41360)--(0.93807,0.41286)--(0.93860,0.41073)--(0.93885,0.40975)--(0.93967,0.40754)--(0.94132,0.40654)--(0.94318,0.40598)--(0.94466,0.40510)--(0.94330,0.40178)--(0.94352,0.40079)--(0.94800,0.39501)--(0.94895,0.39313)--(0.94782,0.39251)--(0.94717,0.39134)--(0.94715,0.38968)--(0.94792,0.38754)--(0.94612,0.38670)--(0.94445,0.38469)--(0.94321,0.38223)--(0.94281,0.38005)--(0.94368,0.37667)--(0.94532,0.37595)--(0.94727,0.37604)--(0.94906,0.37502)--(0.95545,0.36312)--(0.95791,0.36050)--(0.96327,0.35757)--(0.96840,0.35298)--(0.96966,0.35110)--(0.96805,0.35012)--(0.96777,0.34861)--(0.96948,0.34552)--(0.97241,0.34157)--(0.97344,0.33948)--(0.97508,0.33282)--(0.97515,0.33182)--(0.97480,0.33053)--(0.97559,0.32992)--(0.97692,0.32948)--(0.97822,0.32872)--(0.97945,0.32708)--(0.97983,0.32577)--(0.98003,0.32433)--(0.98073,0.32226)--(0.98242,0.31983)--(0.98974,0.31394)--(0.99854,0.30985)--(1.00000,0.30792)--(0.99894,0.30504)--(0.99589,0.30417)--(0.99251,0.30391)--(0.99050,0.30284)--(0.98769,0.30418)--(0.98491,0.30362)--(0.98269,0.30168)--(0.98162,0.29883)--(0.98202,0.29577)--(0.98366,0.29382)--(0.98615,0.29286)--(0.98903,0.29271)--(0.98910,0.29171)--(0.98796,0.28880)--(0.98956,0.28643)--(0.99247,0.28488)--(0.99526,0.28439)--(0.99489,0.28068)--(0.99593,0.27735)--(0.99895,0.27127)--(0.99899,0.27121)--(0.99690,0.26694)--(0.99846,0.25936)--(0.99934,0.25728)--(0.99990,0.25687)--(0.99980,0.25650)--(0.99864,0.25456)--(0.99684,0.25236)--(0.99047,0.24845)--(0.99031,0.24649)--(0.99068,0.24340)--(0.99065,0.24130)--(0.99030,0.23953)--(0.98841,0.23408)--(0.98314,0.23147)--(0.96451,0.23002)--(0.96103,0.22867)--(0.95931,0.22765)--(0.95800,0.22619)--(0.95702,0.22376)--(0.95599,0.21940)--(0.95594,0.21917)--(0.95464,0.21700)--(0.94706,0.20766)--(0.94422,0.20506)--(0.94420,0.20506)--(0.94419,0.20505)--(0.94418,0.20505)--(0.93882,0.20039)--(0.93437,0.19649)--(0.91220,0.16943)--(0.90935,0.16443)--(0.90572,0.16124)--(0.90184,0.15645)--(0.89146,0.13930)--(0.88733,0.13459)--(0.88686,0.13323)--(0.88618,0.12994)--(0.88575,0.12877)--(0.88489,0.12781)--(0.88309,0.12704)--(0.88220,0.12638)--(0.88125,0.12509)--(0.87542,0.11311)--(0.87375,0.11124)--(0.87263,0.11069)--(0.87041,0.11013)--(0.86930,0.10906)--(0.86861,0.10763)--(0.86706,0.10240)--(0.85965,0.09133)--(0.85734,0.08499)--(0.85963,0.07933)--(0.86045,0.07701)--(0.86144,0.07019)--(0.86192,0.06838)--(0.86226,0.06750)--(0.86373,0.06540)--(0.86612,0.06312)--(0.86677,0.06192)--(0.86736,0.05938)--(0.86951,0.04469)--(0.86923,0.04233)--(0.86769,0.03734)--(0.86648,0.03156)--(0.86498,0.02800)--(0.86451,0.02601)--(0.86519,0.02519)--(0.86567,0.02554)--(0.86698,0.02736)--(0.86870,0.02628)--(0.86857,0.02620)--(0.86817,0.02608)--(0.87172,0.02384)--(0.87501,0.01992)--(0.87703,0.01858)--(0.88190,0.01617)--(0.88326,0.01460)--(0.88334,0.01324)--(0.88254,0.01003)--(0.88252,0.00784)--(0.88306,0.00635)--(0.88493,0.00348)--(0.88508,0.00315)--(0.88531,0.00263)--(0.88420,0.00000)--(0.88201,0.00081)--(0.87956,0.00281)--(0.87762,0.00370)--(0.87643,0.00402)--(0.87433,0.00667)--(0.87321,0.00763)--(0.87205,0.00771)--(0.86971,0.00687)--(0.86586,0.00696)--(0.86361,0.00664)--(0.86132,0.00680)--(0.85863,0.00846)--(0.85445,0.01265)--(0.85230,0.01321)--(0.85017,0.01096)--(0.84642,0.01259)--(0.83813,0.01257)--(0.83420,0.01359)--(0.82860,0.01874)--(0.82632,0.01953)--(0.82048,0.01905)--(0.81546,0.02035)--(0.81464,0.02099)--(0.81436,0.02200)--(0.81461,0.02305)--(0.81455,0.02418)--(0.81334,0.02542)--(0.81246,0.02555)--(0.80994,0.02514)--(0.80888,0.02516)--(0.80764,0.02565)--(0.80411,0.02772)--(0.79658,0.02884)--(0.79350,0.03091)--(0.79260,0.03601)--(0.79105,0.04164)--(0.78783,0.04641)--(0.78389,0.04984)--(0.77811,0.05245)--(0.77412,0.05586)--(0.77215,0.05689)--(0.77005,0.05642)--(0.76830,0.05469)--(0.76687,0.05286)--(0.76580,0.05209)--(0.76413,0.05334)--(0.76040,0.05866)--(0.75857,0.06041)--(0.75751,0.06068)--(0.75590,0.06026)--(0.75510,0.06034)--(0.75401,0.06104)--(0.75199,0.06288)--(0.75090,0.06356)--(0.74892,0.06400)--(0.74142,0.06289)--(0.73985,0.06219)--(0.73796,0.06135)--(0.73640,0.06176)--(0.73253,0.06077)--(0.73079,0.06070)--(0.72046,0.06327)--(0.71448,0.06623)--(0.71302,0.06654)--(0.71116,0.06516)--(0.70802,0.06021)--(0.70627,0.05847)--(0.70402,0.05802)--(0.70238,0.05909)--(0.70083,0.06072)--(0.69890,0.06197)--(0.69726,0.06207)--(0.69302,0.06109)--(0.69143,0.06028)--(0.68937,0.05784)--(0.69069,0.05641)--(0.69315,0.05500)--(0.69450,0.05261)--(0.69368,0.05111)--(0.69175,0.05067)--(0.68953,0.05058)--(0.68789,0.05019)--(0.68578,0.04803)--(0.68247,0.04290)--(0.68025,0.04087)--(0.67780,0.03991)--(0.67585,0.04023)--(0.67411,0.04143)--(0.67235,0.04318)--(0.67034,0.04459)--(0.66868,0.04480)--(0.66689,0.04465)--(0.66452,0.04496)--(0.66283,0.04614)--(0.65323,0.05708)--(0.65158,0.05734)--(0.64762,0.05661)--(0.64534,0.05690)--(0.64438,0.05681)--(0.64331,0.05618)--(0.64168,0.05408)--(0.64050,0.05292)--(0.63803,0.05198)--(0.63571,0.05250)--(0.62674,0.05686)--(0.62510,0.05645)--(0.62518,0.05368)--(0.61975,0.05337)--(0.61541,0.05492)--(0.61416,0.05487)--(0.61310,0.05405)--(0.61273,0.05304)--(0.61254,0.05156)--(0.61203,0.04925)--(0.61132,0.04700)--(0.61081,0.04628)--(0.60270,0.04543)--(0.60111,0.04451)--(0.59886,0.04173)--(0.59784,0.04075)--(0.59552,0.04099)--(0.59525,0.04123)--(0.59466,0.04043)--(0.59272,0.03588)--(0.59214,0.03395)--(0.59144,0.03215)--(0.59016,0.03055)--(0.58964,0.02492)--(0.58860,0.02087)--(0.58656,0.01925)--(0.58304,0.02092)--(0.57753,0.02597)--(0.57457,0.02746)--(0.57120,0.02704)--(0.56931,0.02560)--(0.56727,0.02345)--(0.56580,0.02190)--(0.56347,0.02101)--(0.55860,0.02132)--(0.55592,0.02227)--(0.55454,0.02414)--(0.55454,0.02415)--(0.55494,0.02626)--(0.55887,0.03281)--(0.55933,0.03320)--(0.56001,0.03340)--(0.56061,0.03376)--(0.56085,0.03467)--(0.56065,0.03537)--(0.56061,0.03541)--(0.55980,0.03624)--(0.55977,0.03630)--(0.55960,0.03667)--(0.55959,0.03667)--(0.55814,0.03773)--(0.55777,0.03826)--(0.55777,0.03828)--(0.55782,0.03895)--(0.55862,0.04016)--(0.55877,0.04082)--(0.55807,0.04860)--(0.55751,0.05128)--(0.55650,0.05432)--(0.55648,0.05433)--(0.55553,0.05504)--(0.55202,0.05359)--(0.54989,0.05335)--(0.54760,0.05358)--(0.54154,0.05559)--(0.54154,0.05564)--(0.54175,0.05753)--(0.54166,0.06006)--(0.54182,0.06093)--(0.54230,0.06195)--(0.53810,0.06196)--(0.53590,0.06253)--(0.53409,0.06391)--(0.53354,0.06505)--(0.53181,0.07017)--(0.52984,0.07828)--(0.52855,0.08151)--(0.52609,0.08313)--(0.52529,0.08396)--(0.52351,0.08706)--(0.52262,0.08742)--(0.52204,0.08765)--(0.52144,0.08738)--(0.52104,0.08658)--(0.51201,0.07652)--(0.51122,0.07584)--(0.50974,0.07523)--(0.50866,0.07505)--(0.50543,0.07507)--(0.50341,0.07463)--(0.50170,0.07378)--(0.50018,0.07232)--(0.49875,0.07003)--(0.49709,0.06480)--(0.49727,0.06461)--(0.49761,0.06413)--(0.49782,0.06337)--(0.49757,0.06229)--(0.49711,0.06194)--(0.49581,0.06190)--(0.49551,0.06184)--(0.49478,0.05870)--(0.49451,0.05669)--(0.49376,0.05540)--(0.49052,0.05409)--(0.48788,0.05356)--(0.48679,0.05364)--(0.48578,0.05415)--(0.48369,0.05567)--(0.48273,0.05600)--(0.48043,0.05546)--(0.47637,0.05310)--(0.47371,0.05289)--(0.47166,0.05327)--(0.47107,0.05509)--(0.47151,0.06038)--(0.47124,0.06277)--(0.47073,0.06495)--(0.47062,0.06725)--(0.47154,0.06996)--(0.46851,0.07182)--(0.45812,0.07279)--(0.45857,0.07548)--(0.45828,0.07881)--(0.45679,0.08568)--(0.45633,0.08714)--(0.45510,0.08996)--(0.45462,0.09151)--(0.45393,0.09705)--(0.45350,0.09831)--(0.45194,0.09948)--(0.45031,0.09920)--(0.44867,0.09850)--(0.44715,0.09838)--(0.44583,0.09918)--(0.44379,0.10161)--(0.44243,0.10273)--(0.44095,0.10319)--(0.43773,0.10348)--(0.43634,0.10446)--(0.43569,0.10568)--(0.43467,0.11037)--(0.43223,0.11544)--(0.43045,0.12035)--(0.43043,0.12467)--(0.43331,0.12797)--(0.43056,0.12948)--(0.43012,0.13145)--(0.43039,0.13393)--(0.42976,0.13685)--(0.42823,0.13779)--(0.42412,0.13520)--(0.42170,0.13520)--(0.41400,0.13953)--(0.40999,0.14100)--(0.40579,0.14052)--(0.40602,0.14052)--(0.40620,0.14042)--(0.40634,0.14024)--(0.40645,0.13996)--(0.40352,0.13745)--(0.40201,0.13716)--(0.39996,0.13887)--(0.39939,0.13997)--(0.39870,0.14247)--(0.39826,0.14336)--(0.39704,0.14429)--(0.39460,0.14482)--(0.39339,0.14538)--(0.39014,0.14824)--(0.38857,0.14913)--(0.38195,0.15047)--(0.38062,0.15205)--(0.38133,0.15501)--(0.37532,0.15905)--(0.37385,0.15944)--(0.37214,0.15989)--(0.37255,0.15624)--(0.37309,0.15600)--(0.37392,0.15625)--(0.37481,0.15621)--(0.37553,0.15509)--(0.37569,0.15348)--(0.37515,0.15278)--(0.37176,0.15181)--(0.36759,0.15174)--(0.36570,0.15128)--(0.36260,0.14859)--(0.35880,0.14750)--(0.35503,0.14836)--(0.34841,0.15516)--(0.34727,0.15679)--(0.34649,0.15873)--(0.34594,0.16073)--(0.34526,0.16252)--(0.34414,0.16391)--(0.34318,0.16490)--(0.34287,0.16583)--(0.34322,0.16671)--(0.34420,0.16752)--(0.34424,0.16763)--(0.34427,0.16776)--(0.34425,0.16791)--(0.34421,0.16807)--(0.34276,0.16840)--(0.33997,0.16985)--(0.33861,0.17024)--(0.33750,0.17000)--(0.33662,0.16951)--(0.33556,0.16934)--(0.33396,0.17010)--(0.32905,0.17656)--(0.32886,0.17774)--(0.32965,0.17935)--(0.33079,0.17990)--(0.33579,0.18000)--(0.33857,0.18103)--(0.34302,0.18359)--(0.34514,0.18537)--(0.34616,0.18780)--(0.34460,0.19080)--(0.34352,0.19190)--(0.34332,0.19353)--(0.34355,0.19500)--(0.34472,0.19800)--(0.34496,0.19850)--(0.34504,0.19898)--(0.34500,0.19942)--(0.34475,0.19979)--(0.34289,0.20238)--(0.34233,0.20293)--(0.34059,0.20361)--(0.33991,0.20319)--(0.33963,0.20210)--(0.33903,0.20079)--(0.33589,0.19661)--(0.33409,0.19520)--(0.33169,0.19462)--(0.32249,0.19642)--(0.31756,0.19608)--(0.31445,0.19246)--(0.31391,0.19211)--(0.31336,0.19201)--(0.31281,0.19214)--(0.31229,0.19250)--(0.31140,0.19552)--(0.30918,0.19739)--(0.30690,0.19727)--(0.30571,0.19435)--(0.30436,0.19542)--(0.30368,0.19709)--(0.30352,0.19913)--(0.30370,0.20123)--(0.30324,0.20165)--(0.30282,0.20212)--(0.30247,0.20267)--(0.30220,0.20329)--(0.30411,0.20401)--(0.30200,0.20496)--(0.29716,0.20493)--(0.29472,0.20570)--(0.29114,0.20808)--(0.29016,0.20921)--(0.28957,0.21062)--(0.28898,0.21272)--(0.28867,0.21447)--(0.28891,0.21492)--(0.28771,0.21531)--(0.28456,0.21474)--(0.28314,0.21479)--(0.28184,0.21553)--(0.27950,0.21760)--(0.27839,0.21832)--(0.26789,0.22157)--(0.26550,0.22135)--(0.26400,0.22230)--(0.26115,0.22413)--(0.25880,0.22449)--(0.25596,0.22287)--(0.25590,0.22063)--(0.25906,0.21566)--(0.26015,0.21268)--(0.26082,0.20923)--(0.26175,0.20608)--(0.26362,0.20404)--(0.26610,0.20249)--(0.26792,0.20038)--(0.26893,0.19746)--(0.26898,0.19354)--(0.27027,0.19173)--(0.27042,0.18960)--(0.26972,0.18824)--(0.26848,0.18875)--(0.26743,0.19043)--(0.26697,0.19142)--(0.26637,0.19186)--(0.26485,0.19192)--(0.26455,0.19149)--(0.26452,0.18954)--(0.26424,0.18896)--(0.26331,0.18892)--(0.26142,0.18965)--(0.26051,0.18964)--(0.25925,0.18882)--(0.25746,0.18648)--(0.25646,0.18564)--(0.25530,0.18538)--(0.25403,0.18510)--(0.25196,0.18530)--(0.24990,0.18506)--(0.24752,0.18314)--(0.24558,0.18038)--(0.24439,0.17823)--(0.24307,0.17648)--(0.24072,0.17488)--(0.23740,0.17063)--(0.23318,0.17045)--(0.22890,0.17284)--(0.22544,0.17625)--(0.22370,0.17869)--(0.22279,0.18073)--(0.22180,0.18626)--(0.22078,0.18935)--(0.21776,0.19272)--(0.21652,0.19513)--(0.21510,0.19721)--(0.21263,0.20208)--(0.21122,0.20422)--(0.20899,0.20591)--(0.20445,0.20775)--(0.20270,0.21025)--(0.20249,0.21167)--(0.20262,0.21313)--(0.20260,0.21463)--(0.20193,0.21616)--(0.20096,0.21650)--(0.19627,0.21667)--(0.19505,0.21632)--(0.19429,0.21571)--(0.19353,0.21557)--(0.19225,0.21668)--(0.19180,0.21763)--(0.19062,0.22184)--(0.18730,0.22255)--(0.18644,0.22547)--(0.18735,0.22799)--(0.18938,0.22745)--(0.18921,0.22795)--(0.18877,0.22976)--(0.18940,0.22979)--(0.19037,0.23036)--(0.19094,0.23056)--(0.19258,0.23182)--(0.19442,0.23283)--(0.19601,0.23410)--(0.19688,0.23615)--(0.19852,0.23662)--(0.20178,0.23558)--(0.20355,0.23591)--(0.20457,0.23723)--(0.20676,0.23925)--(0.20768,0.24045)--(0.20739,0.24037)--(0.20741,0.24132)--(0.20759,0.24255)--(0.20778,0.24336)--(0.20826,0.24428)--(0.20856,0.24472)--(0.20858,0.24525)--(0.20823,0.24648)--(0.21150,0.24769)--(0.21068,0.25079)--(0.20228,0.26035)--(0.19920,0.26225)--(0.19588,0.26323)--(0.19217,0.26344)--(0.19084,0.26321)--(0.19003,0.26162)--(0.18921,0.25959)--(0.18790,0.25807)--(0.18616,0.25773)--(0.18463,0.25852)--(0.18148,0.26113)--(0.17718,0.26174)--(0.17059,0.25584)--(0.16620,0.25522)--(0.16601,0.25571)--(0.16595,0.25622)--(0.16606,0.25673)--(0.16628,0.25725)--(0.16780,0.25839)--(0.16811,0.26029)--(0.16764,0.26254)--(0.16684,0.26475)--(0.16561,0.26734)--(0.16429,0.26873)--(0.16261,0.26904)--(0.15645,0.26701)--(0.15362,0.26705)--(0.15297,0.26731)--(0.14818,0.27633)--(0.14789,0.27733)--(0.14772,0.27843)--(0.14519,0.27879)--(0.13661,0.27711)--(0.13555,0.27750)--(0.13483,0.27826)--(0.13357,0.28040)--(0.13245,0.28122)--(0.11235,0.28934)--(0.10808,0.28918)--(0.10707,0.28805)--(0.10602,0.28638)--(0.10483,0.28525)--(0.10466,0.28531)--(0.10341,0.28578)--(0.10282,0.28714)--(0.10298,0.28867)--(0.10334,0.29027)--(0.10339,0.29183)--(0.10231,0.29492)--(0.10083,0.29627)--(0.09903,0.29718)--(0.09707,0.29892)--(0.09608,0.30072)--(0.09501,0.30344)--(0.09421,0.30633)--(0.09402,0.30863)--(0.09488,0.31156)--(0.09585,0.31320)--(0.09604,0.31475)--(0.09454,0.31748)--(0.09190,0.31896)--(0.08611,0.31935)--(0.08528,0.32177)--(0.08494,0.32424)--(0.08378,0.32490)--(0.08242,0.32404)--(0.08139,0.32199)--(0.08139,0.32197)--(0.08097,0.32035)--(0.07981,0.32018)--(0.07842,0.32096)--(0.07729,0.32220)--(0.07651,0.32384)--(0.07143,0.32525)--(0.06959,0.32713)--(0.06923,0.32582)--(0.06872,0.32510)--(0.06719,0.32411)--(0.06661,0.32402)--(0.06602,0.32377)--(0.06548,0.32338)--(0.06499,0.32285)--(0.06498,0.32285)--(0.06531,0.32101)--(0.06688,0.31961)--(0.06831,0.31727)--(0.06817,0.31259)--(0.06781,0.31179)--(0.06675,0.31071)--(0.06643,0.31018)--(0.06629,0.30928)--(0.06632,0.30866)--(0.06641,0.30820)--(0.06623,0.30260)--(0.06594,0.30122)--(0.05435,0.30272)--(0.05061,0.30190)--(0.04844,0.30210)--(0.04862,0.30509)--(0.05020,0.30739)--(0.05405,0.31121)--(0.05986,0.32313)--(0.06503,0.33438)--(0.06544,0.33621)--(0.06563,0.33931)--(0.06589,0.34088)--(0.06650,0.34197)--(0.06732,0.34267)--(0.06805,0.34371)--(0.06838,0.34576)--(0.06871,0.34983)--(0.06979,0.35430)--(0.07139,0.35852)--(0.07331,0.36183)--(0.07292,0.36230)--(0.07258,0.36282)--(0.07336,0.36747)--(0.07206,0.37070)--(0.06992,0.37352)--(0.06823,0.37691)--(0.06801,0.37803)--(0.06773,0.38095)--(0.06774,0.38110)--(0.06751,0.38795)--(0.06737,0.39222)--(0.06834,0.39419)--(0.06742,0.39646)--(0.06567,0.39852)--(0.06417,0.39984)--(0.06195,0.40106)--(0.05731,0.40273)--(0.05509,0.40417)--(0.05509,0.40419)--(0.05306,0.40582)--(0.04662,0.40842)--(0.04454,0.41087)--(0.04431,0.41259)--(0.04416,0.41376)--(0.04516,0.41684)--(0.04729,0.41982)--(0.04866,0.42297)--(0.04874,0.42722)--(0.04800,0.43137)--(0.04687,0.43422)--(0.04224,0.43856)--(0.04015,0.44134)--(0.03936,0.44550)--(0.03922,0.44977)--(0.03795,0.45171)--(0.03622,0.45307)--(0.03473,0.45568)--(0.03464,0.45911)--(0.03611,0.46182)--(0.04218,0.46731)--(0.04416,0.46966)--(0.04587,0.47246)--(0.04690,0.47556)--(0.04783,0.48538)--(0.04846,0.48691)--(0.05084,0.49068)--(0.05135,0.49218)--(0.05162,0.49365)--(0.05202,0.49507)--(0.05286,0.49640)--(0.05482,0.49857)--(0.04866,0.50378)--(0.04780,0.50615)--(0.04865,0.51029)--(0.04989,0.51316)--(0.05041,0.51544)--(0.04911,0.51776)--(0.04917,0.51876)--(0.05125,0.52120)--(0.05148,0.52450)--(0.05011,0.52742)--(0.04742,0.52879)--(0.04465,0.52961)--(0.04140,0.53154)--(0.03934,0.53417)--(0.04011,0.53706)--(0.03833,0.54187)--(0.03642,0.54905)--(0.03626,0.55547)--(0.04164,0.55966)--(0.04424,0.56330)--(0.04610,0.56698)--(0.04575,0.56872)--(0.04402,0.57051)--(0.04501,0.57420)--(0.04811,0.57981)--(0.04521,0.58251)--(0.04469,0.58318)--(0.03939,0.58993)--(0.03367,0.59262)--(0.03110,0.59665)--(0.02930,0.59764)--(0.02834,0.59869)--(0.02541,0.60551)--(0.02401,0.60738)--(0.01413,0.61572)--(0.01156,0.61732)--(0.00863,0.62130)--(0.00673,0.62230)--(0.00457,0.62260)--(0.00257,0.62338)--(0.00096,0.62483)--(0.00000,0.62717)--(0.00274,0.62924)--(0.00388,0.63112)--(0.00449,0.63414)--(0.00440,0.63617)--(0.00360,0.64196)--(0.00313,0.64362)--(0.00320,0.64463)--(0.00814,0.64781)--(0.01426,0.65051)--(0.02028,0.65470)--(0.02346,0.65762)--(0.02495,0.66045)--(0.02531,0.66178)--(0.02881,0.67071)--(0.02876,0.67305)--(0.02832,0.67537)--(0.02814,0.67801)--(0.02855,0.68013)--(0.02937,0.68206)--(0.03042,0.68343)--(0.03148,0.68391)--(0.03225,0.68493)--(0.03509,0.68974)--(0.03390,0.69098)--(0.03358,0.69191)--(0.03219,0.69325)--(0.03166,0.69431)--(0.03147,0.69592)--(0.03160,0.69884)--(0.03153,0.70022)--(0.03108,0.70328)--(0.02967,0.71280)--(0.02921,0.72120)--(0.02741,0.72474)--(0.02663,0.72817)--(0.02454,0.73193)--(0.02406,0.73530)--(0.02423,0.73887)--(0.02488,0.74475)--(0.02480,0.74635)--(0.02321,0.75274)--(0.02198,0.75599)--(0.02254,0.75847)--(0.02272,0.76300)--(0.02273,0.76302)--(0.02273,0.76302)--(0.02367,0.76306)--(0.02544,0.76370)--(0.02678,0.76479)--(0.02699,0.76609)--(0.02539,0.76737)--(0.02636,0.76787)--(0.02670,0.76840)--(0.02643,0.76897)--(0.02553,0.76957)--(0.02615,0.77074)--(0.02703,0.76984)--(0.02832,0.76908)--(0.02945,0.76811)--(0.03035,0.76493)--(0.03164,0.76380)--(0.03417,0.76237)--(0.03595,0.76018)--(0.03694,0.75924)--(0.04111,0.75825)--(0.04613,0.75608)--(0.04783,0.75457)--(0.04915,0.75238)--(0.05192,0.74614)--(0.05293,0.74464)--(0.05347,0.74508)--(0.05374,0.74766)--(0.05427,0.74913)--(0.05514,0.75007)--(0.05652,0.75103)--(0.05649,0.75164)--(0.05650,0.75228)--(0.05665,0.75322)--(0.05564,0.75358)--(0.05320,0.75396)--(0.05214,0.75460)--(0.04963,0.76194)--(0.05047,0.76546)--(0.05325,0.77230)--(0.05428,0.77126)--(0.05525,0.77147)--(0.05632,0.77202)--(0.05781,0.77201)--(0.05739,0.77612)--(0.05794,0.77933)--(0.05877,0.78217)--(0.05928,0.78518)--(0.05863,0.78522)--(0.05828,0.78331)--(0.05745,0.78207)--(0.05503,0.77991)--(0.05556,0.77878)--(0.05474,0.77918)--(0.05410,0.77965)--(0.05360,0.78029)--(0.05314,0.78114)--(0.05366,0.78158)--(0.05459,0.78267)--(0.05524,0.78322)--(0.05410,0.78603)--(0.05191,0.78746)--(0.04143,0.78923)--(0.04004,0.79097)--(0.04145,0.79403)--(0.04116,0.79437)--(0.04108,0.79442)--(0.04106,0.79456)--(0.04092,0.79517)--(0.03988,0.79362)--(0.03912,0.79344)--(0.03835,0.79392)--(0.03723,0.79431)--(0.03638,0.79403)--(0.03568,0.79328)--(0.03497,0.79228)--(0.03409,0.79129)--(0.03690,0.79055)--(0.03840,0.78974)--(0.03914,0.78865)--(0.03883,0.78671)--(0.03774,0.78536)--(0.03650,0.78488)--(0.03568,0.78557)--(0.03297,0.78409)--(0.03165,0.78376)--(0.03027,0.78371)--(0.03020,0.78261)--(0.03148,0.78201)--(0.03138,0.78168)--(0.03013,0.78151)--(0.02924,0.78171)--(0.02778,0.78259)--(0.02668,0.78284)--(0.02516,0.78402)--(0.02103,0.79096)--(0.01887,0.79213)--(0.01865,0.79212)--(0.01864,0.79212)--(0.01805,0.79468)--(0.01652,0.79684)--(0.01745,0.79705)--(0.01827,0.79754)--(0.02018,0.80177)--(0.02170,0.80095)--(0.02268,0.80002)--(0.02551,0.79872)--(0.03487,0.79732)--(0.03869,0.79785)--(0.04251,0.79958)--(0.04965,0.80417)--(0.07544,0.81380)--(0.09740,0.81821)--(0.12435,0.82790)--(0.14162,0.83158)--(0.16051,0.83571)--(0.17920,0.84318)--(0.19449,0.84442)--(0.20507,0.84749)--(0.20383,0.84633)--(0.19907,0.84443)--(0.20106,0.84289)--(0.20944,0.84401)--(0.20948,0.84500)--(0.20883,0.84503)--(0.20888,0.84622)--(0.21338,0.84604)--(0.21336,0.84683)--(0.21327,0.84698)--(0.21309,0.84693)--(0.21278,0.84718)--(0.21306,0.84977)--(0.21090,0.84996)--(0.20570,0.84856)--(0.20901,0.85003)--(0.21301,0.85243)--(0.21911,0.85905)--(0.21783,0.85670)--(0.21564,0.85458)--(0.21477,0.85341)--(0.21564,0.85369)--(0.21737,0.85398)--(0.21889,0.85452)--(0.22026,0.85605)--(0.22362,0.85748)--(0.22488,0.85883)--(0.22385,0.86122)--(0.22199,0.86135)--(0.22112,0.86037)--(0.21969,0.85903)--(0.22394,0.86576)--(0.23534,0.87937)--(0.23530,0.87838)--(0.23474,0.87683)--(0.23568,0.87639)--(0.23721,0.87692)--(0.23846,0.87827)--(0.23913,0.88041)--(0.23861,0.88109)--(0.23737,0.88063)--(0.23593,0.87935)--(0.23799,0.88313)--(0.24137,0.88628)--(0.24790,0.89005)--(0.26757,0.89230)--(0.27526,0.89509)--(0.27926,0.89564)--(0.28283,0.89681)--(0.29208,0.90639)--(0.32057,0.91888)--(0.35596,0.92481)--(0.37187,0.92998)--(0.39674,0.93192)--(0.41205,0.93175)--(0.41415,0.93095)--(0.41750,0.92796)--(0.41874,0.92730)--(0.42069,0.92683)--(0.45243,0.90758)--(0.45823,0.90001)--(0.45917,0.89823)--(0.46012,0.89572)--(0.46020,0.89353)--(0.45855,0.89274)--(0.45814,0.89329)--(0.45769,0.89446)--(0.45712,0.89563)--(0.45542,0.89684)--(0.45428,0.89987)--(0.45275,0.90184)--(0.44834,0.90940)--(0.44747,0.90981)--(0.44420,0.91050)--(0.44337,0.91085)--(0.44249,0.91237)--(0.43985,0.91331)--(0.43430,0.91717)--(0.42884,0.91984)--(0.42797,0.92111)--(0.42496,0.92347)--(0.42417,0.92385)--(0.42303,0.92335)--(0.42243,0.92252)--(0.42107,0.91793)--(0.42088,0.91687)--(0.42086,0.91451)--(0.42136,0.91391)--(0.42358,0.91289)--(0.42408,0.91228)--(0.42519,0.91031)--(0.42653,0.90732)--(0.42647,0.90608)--(0.42611,0.90502)--(0.42613,0.90396)--(0.42720,0.90281)--(0.42661,0.89966)--(0.42767,0.89831)--(0.42945,0.89868)--(0.43102,0.90067)--(0.43045,0.89800)--(0.43119,0.89586)--(0.43234,0.89404)--(0.43466,0.88669)--(0.43538,0.88514)--(0.43499,0.88144)--(0.43503,0.87506)--(0.43561,0.86902)--(0.43690,0.86631)--(0.43838,0.86581)--(0.44124,0.86359)--(0.44270,0.86308)--(0.44396,0.86294)--(0.44553,0.86245)--(0.44687,0.86156)--(0.44744,0.86016)--(0.44902,0.85793)--(0.46488,0.85275)--(0.46839,0.85248)--(0.47132,0.85410)--(0.47213,0.85328)--(0.47323,0.85267)--(0.47582,0.85189)--cycle; \end{tikzpicture}
    			\caption{Obrys granic Polski\mysrc}\label{fig:poland}
    	    \end{center}
    	\end{figure}
    	
    	Natomiast otoczka wypukła obrysu granic Polski zawiera tylko 28 krawędzi i została przedstawiona na rysunku \ref{fig:poland_convex_hull}.
    	  \begin{figure}[h!]
        	\begin{center}
    			\begin{tikzpicture}[scale=5.5] \draw (0.39674,0.93192)--(0.41205,0.93175)--(0.83852,0.87059)--(0.84844,0.86831)--(0.88310,0.84684)--(0.88692,0.84397)--(0.89476,0.83673)--(0.89647,0.83452)--(0.89800,0.83138)--(0.89926,0.82784)--(0.94964,0.67580)--(0.95409,0.64486)--(1.00000,0.30792)--(0.99990,0.25687)--(0.99980,0.25650)--(0.88420,0.00000)--(0.58656,0.01925)--(0.55860,0.02132)--(0.55592,0.02227)--(0.47166,0.05327)--(0.23318,0.17045)--(0.22890,0.17284)--(0.04844,0.30210)--(0.00000,0.62717)--(0.01652,0.79684)--(0.02018,0.80177)--(0.32057,0.91888)--(0.37187,0.92998)--cycle; \end{tikzpicture}
    			\caption{Otoczka wypukła obrysu granic Polski\mysrc}\label{fig:poland_convex_hull}
    	    \end{center}
    	\end{figure}
    	
    	W związku z tym, że złożoność obliczeniowa algorytmu sprawdzającego czy punkt należy do wnętrza danego wielokąta jest liniowa \cite{complex}, możliwe jest dużo szybsze wykluczenie punktów znajdujących się poza otoczką. 
		\section{Grafika komputerowa}
		Obiekty wykorzystywane w grafice komputerowej często mogą charakteryzować się skomplikowanymi kształtami. Im bardziej skomplikowany kształt, tym więcej mocy obliczeniowej potrzeba w celu wykonania danej operacji na tym kształcie. W niektórych przypadkach do uproszczenia graficznego danego obiektu używa się otoczki wypukłej jego kształtu. Ze względu na fakt, iż otoczka wypukła wielokąta zawsze będzie miała liczbę wierzchołków mniejszą lub równą liczbie wierzchołków samego wielokąta, powstała otoczka może posłużyć do wykonania mniejszej liczby obliczeń przy wykonywaniu operacji na danym obiekcie.
		\section{Detekcja obiektów}
		Proces analizowania obrazu przez ludzkie organy jest wyjątkowo złożony złożony \cite{BIEDERMAN198529}. W celu imitacji tego procesu w sposób algorytmiczny, zostały przeprowadzone liczne badania naukowe \cite{detection}. Dzięki temu zaproponowano wiele podejść rozwiązania tego problemu. Istotnym aspektem ze względów praktycznych tych pomysłów jest złożoność obliczeniowa ich implementacji.
		
		Jednym z działań wykonywanych przez ludzkie zdolności poznawcze po zarejestrowaniu obrazu jest wybranie miejsca w obrazie o największym znaczeniu (z ludzkiego punktu widzenia) \cite{s23146450}. Niektóre z rozwiązań tego problemu w informatyce opierały się nie tylko na samych danych pochodzących z obrazu, ale również tworząc modele z uwzględnieniem danych statycznych z badań dotyczących ludzkich umiejętności poznawczych \cite{Borji_Cheng_Hou_Jiang_Li_2019}. Jednak spora część z nich opiera się na przetwarzaniu samych danych pozyskanych z obrazów, bez potrzeby dostarczania danych statystycznych. Aby komputer był w stanie wykonać tego typu operację potrzebny jest szereg złożonych obliczeniowo operacji \cite{detection, Borji_Cheng_Hou_Jiang_Li_2019}. 
		
		W celu zmniejszenia kosztu obliczeniowego tego typu procesu można wykonać preprocesowanie danych wejściowych. Jednym z nich może być wykorzystanie algorytmu wyznaczającego otoczkę wypukłą tak zwanych punktów kluczowych \cite{detection}. W celu znalezienia lokalizacji punktów kluczowych można skorzystać z wyznaczania lokalnej energii piksela za pomocą wzoru (\ref{eq:harris}) na wielopoziomową energię Harrisa \cite{detection, harris}. 
		\begin{align}\label{eq:harris}  
			\textbf{E}_{WPH}\left(p\right) = \sum_{l=0}^{L}\textbf{R}\left(l\right)\left(p\right)	 		
        \end{align}\\[-5 pt]
        Gdzie poszczególne symbole oznaczają:\\[10 pt]
       \begin{tabular}{rcl}
    	$p$ & -- & rozpatrywany piksel obrazu\\
		$\textbf{R}\left(l\right)\left(p\right)$ & -- &  \begin{minipage}[t]{123 mm} energia piksela $p$ wyznaczona ze wzoru Harrisa (\ref{eq:harris_single}) dla $l$ - tego poziomu obrazu \end{minipage} \\[20 pt] 
		$L$ & -- & \begin{minipage}[t]{123 mm}
		liczba poziomów dla których ma być wyznaczona suma energii, przyjęte jako $L=5$ \cite{detection}
		\end{minipage}
		
    	\end{tabular}\\[15 pt]
    	W celu wyznaczenia wielopoziomowej Harrisa dla danego piksela $p$ musimy znać też wzór (\ref{eq:harris_single}) na energię piksela $p$ dla pojedynczego poziomu $l$.
    	\begin{align}\label{eq:harris_single}  
		\textbf{R}\left(l\right)\left(p\right) =
		\left|\textbf{M}\left(l\right)\left(p\right)\right| + k \cdot \mathrm{tr}\left(\textbf{M}\left(l\right)\left(p\right)\right)
        \end{align}
Gdzie poszczególne symbole oznaczają:\\[10 pt]
        \begin{tabular}{rcl}
    	$k$ & -- & współczynnik skalarny\\
    	$\textbf{M}\left(l\right)\left(p\right)$ & -- & 
    	\begin{minipage}[t]{121 mm}
    	macierz autokorelacji dla danego piksela $p$ i poziomu obrazu $l$ wyznaczana ze wzoru \ref{eq:autocorelate} \cite{harrispptx, harris, detection}
    	\end{minipage}\\[20 pt]
    	$\mathrm{tr}\left(\textbf{A}\right)$ & -- & 
    	\begin{minipage}[t]{121 mm}
    	ślad macierzy $\textbf{A}$, czyli suma elementów należących do głównej przekątnej macierzy $\textbf{A}$, obliczany według wzoru \cite{matrixtrace}
    	\end{minipage}
    	\end{tabular}\\[10 pt]
    	\begin{align}\label{eq:autocorelate}  
		\textbf{M}\left(l\right)\left(p\right) =
		\textbf{G}_{\sigma}
		\cdot\left[
		\begin{matrix}
			\textbf{I}_x^2\left(l\right)\left(p\right) & \textbf{I}_x\left(l\right)\left(p\right) \cdot \textbf{I}_y\left(l\right)\left(p\right)\\
			\textbf{I}_y\left(l\right)\left(p\right) \cdot \textbf{I}_x\left(l\right)\left(p\right) & \textbf{I}_y^2\left(l\right)\left(p\right)
		\end{matrix}
		\right]
        \end{align}\\[10 pt]
        Gdzie poszczególne symbole oznaczają:\\[10 pt]
    	\begin{tabular}{rcl}
    	$G_{\sigma}$ & -- & 
    	\begin{minipage}[t]{123 mm}
    	prostokątna macierz o liczbie kolumn 2 stanowiąca okno czasowe Gaussa o odchyleniu standardowym $\sigma$
    	\end{minipage}\\[5 pt]
    	$\textbf{M}\left(l\right)\left(p\right)$ & -- & 
    	\begin{minipage}[t]{123 mm}
    	macierz autokorelacji dla danego piksela $p$ i poziomu obrazu $l$ wyznaczana ze wzoru \ref{eq:autocorelate} \cite{harrispptx, harris, detection}
    	\end{minipage}
    	\end{tabular}\\[10 pt]
		
        \section{Wyznaczanie obwiedni sygnału}
        Sygnał to zapis informacji zmieniającej się w zależności od czasu. Taką informacją może być natężenie/napięcie prądu, natężenie pola elektromagnetycznego, dźwięk utworu muzycznego itd. W przypadku, jeśli zmienną informację da się  reprezentować w sposób liczbowy, możliwe jest graficzne przedstawienie sygnału. Zapisanie informacji w ten sposób może być łatwiejsze do interpretacji przez człowieka. Jednak w przypadku, gdy mamy do czynienia ze skomplikowanym przebiegiem, na
        
        Algorytm wyznaczający otoczkę wypukłą wielokąta prostego może posłużyć znajdowaniu obwiedni sygnału.
        
   \chapter{Implementacja w języku Scala}
   		\section{Pojęcia ogólne}
   		W tej sekcji przedstawione zostaną listingi zawierające kod w języku Scala definiujący pojęcia potrzebne do implementacji poszczególnych algorytmów wyznaczających otoczkę wypukłą.
   		
   		Klasa \texttt{Point} (Listing \ref{lst:point}: linia 4) posłuży jako podstawowa klasa reprezentująca punkt w przestrzeni dwuwymiarowej. W celu łatwiejszego manipulowania danymi. Zostały w jej ramach zaimplementowane 3 metody pomocnicze.
   		
   		Odejmowanie (Listing \ref{lst:point}: linia 7) - zaimplementowane jako odejmowanie od siebie odpowiadających sobie współrzędnych dwu punktów.
   		
   		Dodawanie (Listing \ref{lst:point}: linia 8) - dodawanie do siebie odpowiadających sobie współrzędnych dwu punktów.
   		
   		Dzielenie przez skalar (Listing \ref{lst:point}: linia 9) - zaimplementowane jako dzielenie obu współrzędnych przez wskazaną liczbę. Zaimplementowane dzięki użyciu domniemanemu argumentowi \scala{numeric: Numeric[T]}. W języku Scala tego typu konstrukcja umożliwia korzystanie z metody \scala{/} w tedy i tylko wtedy jeśli zaimportowany zostanie do aktualnego kontekstu obiekt typu \scala{Numeric[T]}. Domyślnie typ ten jest zaimplementowany dla \scala{T} będącym jednym z podstawowych typów liczbowych: \scala{Float}, \scala{Int}, \scala{Double} etc. Więc w domyśle tego typu konstrukcja służy niejako ,,udowodnieniu'', że wskazywany argument jest w istocie liczbą.
   		
   		\lstinputlisting[language=scala, caption={Point.scala\mysrc}, label={lst:point}]{\listingsDirectory/Point.scala}
   		Kolejnym istotnym elementem wykorzystanym w implementacji algorytmów wyznaczających otoczkę wypukłą zbioru punktów na płaszczyźnie są metody wyliczające określone właściwości punktów.
   		
   		Metoda \scala{calculateCentroid(points: Points): Points} (Listing \ref{lst:pointsutils}: linia 11) wyznacza centroid zbioru punktów \scala{points}.
   		
   		Metoda \scala{distanceFromCenterSquared(p: Point): Double} (Listing \ref{lst:pointsutils}: linia 17) wyznacza kwadrat odległości wskazanego punktu od środka układu współrzędnych.
   		
   		Na podstawie wyżej wymienionej metody zostały zaimplementowane trzy metody związane z odległością między punktami.
   		
   		Metoda \scala{distanceSquared(a: Point, b: Point): Double} (Listing \ref{lst:pointsutils}: linia 13) wyznacza kwadrat odległości pomiędzy dwoma punktami.
   		
   		Metoda \scala{distanceFromCenter(p: Point): Double} (Listing \ref{lst:pointsutils}: linia 19) wyznacza odległość od środka układu współrzędnych.
   		
   		Metoda \scala{distance(a: Point, b: Point): Double} (Listing \ref{lst:pointsutils}: linia 15) wyznacza odległość pomiędzy dwoma punktami.

		W ramach obiektu \scala{PointsUtils} (Listing \ref{lst:pointsutils}: linia 5) zostały również zaimplementowane metody służące do wyznaczania wartości umożliwiających sortowanie punktów względem kąta występującego pomiędzy horyzontalną osią układu współrzędnych, a wektorem łączącym początek układu współrzędnych z punktem (kąt $\theta$ zaznaczony na rysunku \ref{fig:axis}).
		
		Metoda \scala{phase(p: Point): Double} (Listing \ref{lst:pointsutils}: linia 21) wyznacza dokładną wartość szukanego kąta. Matematycznie ta funkcja została zapisana we wzorze \ref{eq:theta}.
		
		Metoda \scala{alpha(p: Point): Double} (Listing \ref{lst:pointsutils}: linia 32) wyznacza wartość umożliwiającą porównanie kąta dla wskazanego punktu z innymi punktami. Matematycznie ta funkcja została zapisana we wzorze \ref{eq:alpha1}.
   		
   		\lstinputlisting[language=scala, caption={PointsUtils.scala}, label={lst:pointsutils}]{\listingsDirectory/PointsUtils.scala}
   		\newpage
   		
   		W celu zapewnienia odpowiedniego sortowania punktów został utworzony \scala{trait OrientationOrdering}. Język Scala umożliwia wykorzystywanie zaimplementowanie własnego obiektu \scala{Ordering[T]}, który będzie następnie używany w metodzie \scala{sorted} dla kolekcji \scala{C[T]}.
   		
   	 	Trait \scala{OrientationOrdering} (Listing \ref{lst:order}: linia 6) ma zaimplementowaną metodę \scala{compare(x: Point, y: Point)} (Listing \ref{lst:order}: linia 7) w taki sposób, aby w pierwszej kolejności porównywane były wartości funkcji reprezentujących fazę tych punktów, a następnie porównywane wartości funkcji reprezentujących odległość od środka układu współrzędnych.
   	 	
   	 	W ramach \scala{OrientationOrdering} zostały zaimplementowane dwa obiekty.
   	 	
   	 	Obiekt \scala{Exact} (Listing \ref{lst:order}: linia 19) służy do dokładnego porównania fazy punktów, wyznaczanej ze wzoru \ref{eq:theta} oraz odległości od środka układu współrzędnych. 
   	 	
   	 	Obiekt \scala{Indicator} (Listing \ref{lst:order}: linia 24) służy do porównywania fazy punktów za pomocą funkcji wyznaczanej ze wzoru \ref{eq:alpha1}.
   		
   		\lstinputlisting[language=scala,caption={OrientationOrdering.scala},captionpos=b, label={lst:order}]{\listingsDirectory/OrientationOrdering.scala}
   		\newpage
   		\lstinputlisting[language=scala,caption={ConvexHullAlgorithm.scala},captionpos=b]{\listingsDirectory/ConvexHullAlgorithm.scala}
   		\newpage
   		\lstinputlisting[language=scala,caption={AlgorithmTest.scala},captionpos=b]{\listingsDirectory/AlgorithmTest.scala}
   		\newpage
   		\section{Algorytm Grahama}
   		\lstinputlisting[language=scala,caption={Graham.scala},captionpos=b]{\listingsDirectory/Graham.scala}
   		\newpage
   		\lstinputlisting[language=scala,caption={PointsCycle.scala},captionpos=b]{\listingsDirectory/PointsCycle.scala}
   		\newpage
   		\lstinputlisting[language=scala,caption={GrahamAlgorithmTest.scala},captionpos=b]{\listingsDirectory/GrahamAlgorithmTest.scala}
   		
   		\section{Algorytm Jarvisa}
   		Dzięki wykorzystaniu pattern matchingu dostępnego w Scali, możliwe jest przejrzyste zaimplementowanie algorytmu Jarvisa, który, jak widać jest dużo mniej skomplikowany od algorytmu Grahama. Mniejsze skomplikowanie implementacji wiąże się jednak z dużo większą złożonością obliczeniową.
        \lstinputlisting[language=scala,caption={Jarvis.scala},captionpos=b]{\listingsDirectory/Jarvis.scala}
        \lstinputlisting[language=scala,caption={JarvisAlgorithmTest.scala},captionpos=b]{\listingsDirectory/JarvisAlgorithmTest.scala}
    \chapter{Dynamiczna otoczka wypukła}
    	\section{Algorytm}
    	\section{Implementacja w języku Scala} 
	\chapter{Podsumowanie} 
    \printbibliography[heading=bibintoc]
\end{document}

    
    Z powyższych obliczeń wynika, że we wszystkich przedziałach z zależności
    \ref{eq:xytotheta1}, 
     \ref{eq:xytotheta2}, 
     \ref{eq:xytotheta3} oraz
     \ref{eq:xytotheta4} funkcja $\gamma\left(\theta\right)$ ma dodatnią pochodną, a tym samym jest rosnąca we wszystkich przedziałach, oraz każda z wartości osiąganych w przedziale jest większa od każdej wartości z poprzedniego przedziału, a tym samym funkcja $\alpha\left(p\right)$ spełnia warunki \ref{eq:condition1} i \ref{eq:condition2}.
     
     Postać funkcji $\gamma\left(\theta\right)$ może być opisana wzorem \ref{eq:gammaeq} a wykres jej przebiegu, przedstawiony na rysunku \ref{fig:gamma} wyraźnie potwierdza jej rosnącą monotoniczność we wszystkich przedziałach.
        \begin{align}\label{eq:gammaeq}
        	\gamma\left(
        		\theta
        	\right) = 
        		\left\{
        			\begin{array}{lcl}
        				\frac{
        					\sin\theta
        				}{
							\sin\theta + \cos\theta        			
        				} & 
        					\text{dla} &
        					\theta \in \left\langle
        						0;
        						\frac{
        							\pi
        						}{
									2        						
        						}
        					\right\rangle\\
        				2
        				- \frac{
        					\sin\theta
        				}{
							\sin\theta - \cos\theta        			
        				} & 
        					\text{dla} &
        					\theta \in \left(
        						\frac{
        							\pi
        						}{
									2        						
        						};
        						\pi
        					\right\rangle\\
        				2
        				+ \frac{
        					\sin\theta
        				}{
							\sin\theta + \cos\theta        			
        				} & 
        					\text{dla} &
        					\theta \in \left(
        						\pi;
        						\frac{
        							3\pi
        						}{
									2        						
        						}
        					\right)\\
        				4
        				- \frac{
        					\sin\theta
        				}{
							\sin\theta + \cos\theta        			
        				} & 
        					\text{dla} &
        					\theta \in \left\langle
        						\frac{
        							3\pi
        						}{
									2        						
        						};
        						2\pi
        					\right)\\
        			\end{array}
        		\right.      		
        \end{align}
        \begin{figure}[h!]
        	\begin{center}
    			\begin{tikzpicture}
	
	\draw 
		[help lines, very thin] 
		(0,0) 
		grid [xstep = pi/2]
		(2*pi - 0.001,4- 0.001);
		
	\tzaxes 
		[
			-arcs,
			line width=0.2mm
		]
		(2*pi,4)
		{$\theta$}
		{$\gamma\left(\theta\right)$}
		
	\tzticksx
		(-5pt:5pt)
		{
			0,
			pi*.5/$\frac{\pi}{2}$,
			pi/$\pi$,
			pi*1.5/$\frac{3\pi}{2}$
		}
		[below=5pt]
		
	\tzticksy
		(-5pt:5pt)
		{	
			0,
			1,
			2,
			3
		}
		[left=5pt]
	\tzfn
		[
			black,
			line width=0.25mm
		]
		{sin(\x r)/(sin(\x r) + cos(\x r))}
		[0:.5*pi] 
	\tzfn
		[
			black,
			line width=0.25mm
		]
		{2 - sin(\x r)/(sin(\x r) - cos(\x r))}
		[.5*pi:pi] 
	\tzfn
		[
			black,
			line width=0.25mm
		]
		{2 + sin(\x r)/(sin(\x r) + cos(\x r))}
		[pi:1.5*pi] 
	\tzfn
		[
			black,
			line width=0.25mm
		]
		{4 + sin(\x r)/(cos(\x r) - sin(\x r))}
		[1.5*pi:2*pi] 
\end{tikzpicture}
    			\caption{Wykres funkcji $\gamma\left(\theta\right)$\mysrc}\label{fig:gamma}
    	\end{center}
    	\end{figure}\vspace{-8 mm}\\
       
              
        \subsection{Algorytm Grahama}
        \null
        \begin{figure}[h!]
    			\begin{tikzpicture}[node distance=1.4cm] 
	\node (0,0) (start) {};
	\path[ellipse_block = 
    		reference: start 
    		text: START
	];
	\node[below of = start_block] (input) {};
	\block
		{input}
		{Dane wejściowe:\\
		zbiór punktów 
			$P = \left\{
				p_1,
				p_2,
				...,
				p_n
			\right\} \subset \mathbb{R}^2$, dla każdego z punktów w zbiorze określone są funkcje:\\
			 $x\left(p\right)$ (oznaczająca wartość odciętej punktu $p$)\\
			 $y\left(p\right)$ (oznaczająca wartość rzędnej punktu $p$)\\
		\vspace{.8cm}
		}
	\node[below of = input_south] (centroid) {};
	\block
		{centroid}
		{
		Wyznaczyć punkt $O$, stanowiący centroid zbioru $P$.
		Następnie dla każdego punktu $p$ ze zbioru $P$ wykonać przekształcenia:
		 $x\left(p\right) := x\left(p\right) -  x\left(O\right)$\\
		 $y\left(p\right) := y\left(p\right) -  y\left(O\right)$\\
		 Następnie przypisać współrzędne punktu $O$:\\
		 $x\left(O\right) := 0$\\
		 $y\left(O\right) := 0$\\
		(Tak aby punkt $O$ znajdował się na początku układu współrzędnych.)
		}



	\node[below of = centroid_south] (sort) {};
	\block
		{sort}
		{
			Posortować leksykograficznie wszystkie punkty zbioru $P$ według porządku $\left(\theta\left(p\right), \left|R\right|\left(p\right)\right)$, gdzie $\theta\left(p\right)$ to wartość kąta wektora wiodącego punktu $p$, natomiast $\left|R\right|\left(p\right)$ to długość tego wektora.
		}
	\node[below of = sort_south] (cykl) {};
	\block
		{cykl}
		{
			Z otrzymanego, uporządkowanego zbioru $P$ utworzyć listę pozwalającą na wyznaczenie następnika punktu $p$ w liście - poprzez zastosowanie funkcji $next\left(p\right)$ oraz znalezienie poprzednika punktu $p$ w liście poprzez zastosowanie funkcji $prev\left(p\right)$.\\
	Funkcja $next$ dla ostatniego punktu w uporządkowanym zbiorze będzie wskazywała na pierwszy punkt tego zbioru, natomiast funkcja $prev$ dla pierwszego punktu uporządkowanego zbioru będzie wskazywała na ostatni punkt tego zbioru.
		}
	\node[below of =cykl_south] (punkts) {};
	\block
		{punkts}
		{
			Wyznacz punkt $s$ będący punktem o najmniejszej wartości odciętej ze zbioru punktów o najmniejszej wartości rzędnej ze zbioru $P$.
		}
		
	\node[right of =start, node distance = 8 cm] (q) {};
	\smallblock
		{q}
		{
			$q:=s$
		}
	\node[below of =q] (isequal) {};
	\chooseblock
		{isequal}
		{
			\begin{small}
			$next\left(q\right) \neq s$
			\end{small}
		}
	\node[below of =isequal_south] (inside) {};
	\chooseblock
		{inside}
		{
			$next\left(q\right) \in \Delta\left(
			 	O,
			 	q,
			 	next\left(
			 		next\left(
			 			q
			 		\right)
			 	\right)
			 \right)$
		}
	\node[below of =inside_south] (remove) {};
	\smallblock
		{remove}
		{
			$next\left(q\right) := next\left(
				next\left(
					q
				\right)
			\right)$\\
			$prev\left(next\left(q\right)\right) := q$
			
			
		}
	
	\node[below of =remove_south] (qequalss) {};
	\chooseblock
		{qequalss}
		{	
		\begin{small}
			$q\neq s$
		\end{small}	
		}
	
	\node[below of =qequalss_south] (prev) {};
	\smallblock
		{prev}
		{	
		$q:=prev\left(q\right)$
		}
	
	\node[below of =prev_south] (next) {};
	\smallblock
		{next}
		{	
		$q:=next\left(q\right)$
		}
	
	\node[below of =next_south] (output) {};
	\block
		{output}
		{	
			Wynikiem działania algorytmu jest lista zaczynająca się od punktu $s$ - zawiera ona uporządkowane wierzchołki otoczki wypukłej zbioru punktów $P$.
		}
	
	\node[below of =output_south] (stop) {};
	\path[ellipse_block = 
    		reference: stop
    		text: STOP
	];
	
	\draw
		[-arcs]
		(start_block)--
		(input_block);
		
	\draw
		[-arcs]
		(input_block)--
		(centroid_block);
	
	\draw
		[-arcs]
		(centroid_block)--
		(sort_block);
	
	\draw
		[-arcs]
		(sort_block)--
		(cykl_block);
	
	\draw
		[-arcs]
		(cykl_block)--
		(punkts_block);
	
	
	\draw [-arcs] 
		(punkts_block.east) --++
		(.5,0) |-
		(q_block.west);
	
	\draw [-arcs] 
		(q_block) -- node (centerq) {}
		(isequal_block);
		
	\draw [-arcs] 
		(isequal_block) --
		(inside_block);
	
	\draw [-arcs] 
		(inside_block) --
		(remove_block);
	
	\draw [-arcs] 
		(remove_block) --
		(qequalss_block);
		
	\draw [-arcs] 
		(qequalss_block) --
		(prev_block);
	
	\draw [-arcs] 
		(isequal_block.west) --++
		(-2,0) |-
		(output_block.west);
	
	\draw [-arcs] 
		(inside_block.west) --++
		(-1.07,0) |-
		(next_block.west);
		
	\draw [-arcs] 
		(inside_block.west) --++
		(-1.07,0) |-
		(next_block.west);
	
	
		
	
	
	\draw [-arcs]
		(prev_block.south) --++
		(0,-1) node (prev_below) {};
		
	\draw [-arcs]
		(qequalss_block.west) --++
		(-1.95,0)  |-
		(prev_below.center) --++
		(3.3,0)
		node (prev_right) {};
	
	\draw [-arcs]
		(next_block.east) -|
		(prev_right.center) |-
		(centerq.center);
	
	\draw [-arcs]
		(output_block) --
		(stop_block);
	
	
	
\end{tikzpicture}
    			\caption{Algorytm Grahama}\label{fig:graham}
    	\end{figure}
       \newpage
        \subsection{Algorytm Jarvisa}
        
        
        \section{Otoczka wypukła wielokąta prostego}
        \section{Redukcja zbioru punktów do wielokąta prostego}
    \chapter{Zastosowania} 
%    as first chapter
    Istotnym, jeśli nie najistotniejszym zagadnieniem dotyczącym otoczek wypukłych są ich zastosowania. Już od kilkudziesięciu lat problem ten temat znajduje swoje użycie w wielu dziedzinach kombinatoryki i informatyki. Dla przykładu problem sortowania elementów liczbowych listy można sprowadzić do problemu znalezienia otoczki wypukłej. W związku z tym rozwiązanie tego problemu może pomóc w rozwiązaniu innych problemów w bardziej symboliczny i graficzny sposób.
        \section{Generalizacja kartograficzna}
        W celu jak najbardziej informatywnego i niezłożonego przedstawienia danych geograficznych w sposób graficzny potrzebne jest zastosowanie algorytmu generalizującego informacje.
		\section{Grafika komputerowa}
		Obiekty wykorzystywane w grafice komputerowej często mogą charakteryzować się skomplikowanymi kształtami. Im bardziej skomplikowany kształt, tym więcej mocy obliczeniowej potrzeba w celu wykonania danej operacji na tym kształcie. W niektórych przypadkach do uproszczenia graficznego danego obiektu używa się otoczki wypukłej jego kształtu. Ze względu na fakt, iż otoczka wypukła wielokąta zawsze będzie miała liczbę wierzchołków mniejszą lub równą liczbie wierzchołków samego wielokąta, powstała otoczka może posłużyć do wykonania mniejszej liczby obliczeń przy wykonywaniu operacji na danym obiekcie.
		\section{Detekcja obiektów}
        \section{Wyznaczanie obwiedni sygnału}
        Sygnał to zapis informacji zmieniającej się w zależności od czasu. Taką informacją może być natężenie/napięcie prądu, natężenie pola elektromagnetycznego, dźwięk utworu muzycznego itd. W przypadku, jeśli zmienną informację da się  reprezentować w sposób liczbowy, możliwe jest graficzne przedstawienie sygnału. Zapisanie informacji w ten sposób może być łatwiejsze do interpretacji przez człowieka. Jednak w przypadku, gdy mamy do czynienia ze skomplikowanym przebiegiem, na
        
        Algorytm wyznaczający otoczkę wypukłą wielokąta prostego może posłużyć znajdowaniu obwiedni sygnału.
        
   \chapter{Implementacja w języku Scala}
   		\section{Pojęcia ogólne}
   		W tej sekcji przedstawione zostaną listingi zawierające kod w języku Scala definiujący pojęcia potrzebne do implementacji poszczególnych algorytmów wyznaczających otoczkę wypukłą.
   		
   		Klasa \texttt{Point} (Listing \ref{lst:point}: linia 4) posłuży jako podstawowa klasa reprezentująca punkt w przestrzeni dwuwymiarowej. W celu łatwiejszego manipulowania danymi. Zostały w jej ramach zaimplementowane 3 metody pomocnicze.
   		
   		Odejmowanie (Listing \ref{lst:point}: linia 7) - zaimplementowane jako odejmowanie od siebie odpowiadających sobie współrzędnych dwu punktów.
   		
   		Dodawanie (Listing \ref{lst:point}: linia 8) - dodawanie do siebie odpowiadających sobie współrzędnych dwu punktów.
   		
   		Dzielenie przez skalar (Listing \ref{lst:point}: linia 9) - zaimplementowane jako dzielenie obu współrzędnych przez wskazaną liczbę. Zaimplementowane dzięki użyciu domniemanemu argumentowi \scala{numeric: Numeric[T]}. W języku Scala tego typu konstrukcja umożliwia korzystanie z metody \scala{/} w tedy i tylko wtedy jeśli zaimportowany zostanie do aktualnego kontekstu obiekt typu \scala{Numeric[T]}. Domyślnie typ ten jest zaimplementowany dla \scala{T} będącym jednym z podstawowych typów liczbowych: \scala{Float}, \scala{Int}, \scala{Double} etc. Więc w domyśle tego typu konstrukcja służy niejako ,,udowodnieniu'', że wskazywany argument jest w istocie liczbą.
   		
   		\lstinputlisting[language=scala, caption={Point.scala\mysrc}, label={lst:point}]{\listingsDirectory/Point.scala}
   		Kolejnym istotnym elementem wykorzystanym w implementacji algorytmów wyznaczających otoczkę wypukłą zbioru punktów na płaszczyźnie są metody wyliczające określone właściwości punktów.
   		
   		Metoda \scala{calculateCentroid(points: Points): Points} (Listing \ref{lst:pointsutils}: linia 11) wyznacza centroid zbioru punktów \scala{points}.
   		
   		Metoda \scala{distanceFromCenterSquared(p: Point): Double} (Listing \ref{lst:pointsutils}: linia 17) wyznacza kwadrat odległości wskazanego punktu od środka układu współrzędnych.
   		
   		Na podstawie wyżej wymienionej metody zostały zaimplementowane trzy metody związane z odległością między punktami.
   		
   		Metoda \scala{distanceSquared(a: Point, b: Point): Double} (Listing \ref{lst:pointsutils}: linia 13) wyznacza kwadrat odległości pomiędzy dwoma punktami.
   		
   		Metoda \scala{distanceFromCenter(p: Point): Double} (Listing \ref{lst:pointsutils}: linia 19) wyznacza odległość od środka układu współrzędnych.
   		
   		Metoda \scala{distance(a: Point, b: Point): Double} (Listing \ref{lst:pointsutils}: linia 15) wyznacza odległość pomiędzy dwoma punktami.

		W ramach obiektu \scala{PointsUtils} (Listing \ref{lst:pointsutils}: linia 5) zostały również zaimplementowane metody służące do wyznaczania wartości umożliwiających sortowanie punktów względem kąta występującego pomiędzy horyzontalną osią układu współrzędnych, a wektorem łączącym początek układu współrzędnych z punktem (kąt $\theta$ zaznaczony na rysunku \ref{fig:axis}).
		
		Metoda \scala{phase(p: Point): Double} (Listing \ref{lst:pointsutils}: linia 21) wyznacza dokładną wartość szukanego kąta. Matematycznie ta funkcja została zapisana we wzorze \ref{eq:theta}.
		
		Metoda \scala{alpha(p: Point): Double} (Listing \ref{lst:pointsutils}: linia 32) wyznacza wartość umożliwiającą porównanie kąta dla wskazanego punktu z innymi punktami. Matematycznie ta funkcja została zapisana we wzorze \ref{eq:alpha1}.
   		
   		\lstinputlisting[language=scala, caption={PointsUtils.scala}, label={lst:pointsutils}]{\listingsDirectory/PointsUtils.scala}
   		\newpage
   		
   		W celu zapewnienia odpowiedniego sortowania punktów został utworzony \scala{trait OrientationOrdering}. Język Scala umożliwia wykorzystywanie zaimplementowanie własnego obiektu \scala{Ordering[T]}, który będzie następnie używany w metodzie \scala{sorted} dla kolekcji \scala{C[T]}.
   		
   	 	Trait \scala{OrientationOrdering} (Listing \ref{lst:order}: linia 6) ma zaimplementowaną metodę \scala{compare(x: Point, y: Point)} (Listing \ref{lst:order}: linia 7) w taki sposób, aby w pierwszej kolejności porównywane były wartości funkcji reprezentujących fazę tych punktów, a następnie porównywane wartości funkcji reprezentujących odległość od środka układu współrzędnych.
   	 	
   	 	W ramach \scala{OrientationOrdering} zostały zaimplementowane dwa obiekty.
   	 	
   	 	Obiekt \scala{Exact} (Listing \ref{lst:order}: linia 19) służy do dokładnego porównania fazy punktów, wyznaczanej ze wzoru \ref{eq:theta} oraz odległości od środka układu współrzędnych. 
   	 	
   	 	Obiekt \scala{Indicator} (Listing \ref{lst:order}: linia 24) służy do porównywania fazy punktów za pomocą funkcji wyznaczanej ze wzoru \ref{eq:alpha1}.
   		
   		\lstinputlisting[language=scala,caption={OrientationOrdering.scala},captionpos=b, label={lst:order}]{\listingsDirectory/OrientationOrdering.scala}
   		\newpage
   		\lstinputlisting[language=scala,caption={ConvexHullAlgorithm.scala},captionpos=b]{\listingsDirectory/ConvexHullAlgorithm.scala}
   		\newpage
   		\lstinputlisting[language=scala,caption={AlgorithmTest.scala},captionpos=b]{\listingsDirectory/AlgorithmTest.scala}
   		\newpage
   		\section{Algorytm Grahama}
   		\lstinputlisting[language=scala,caption={Graham.scala},captionpos=b]{\listingsDirectory/Graham.scala}
   		\newpage
   		\lstinputlisting[language=scala,caption={PointsCycle.scala},captionpos=b]{\listingsDirectory/PointsCycle.scala}
   		\newpage
   		\lstinputlisting[language=scala,caption={GrahamAlgorithmTest.scala},captionpos=b]{\listingsDirectory/GrahamAlgorithmTest.scala}
   		
   		\section{Algorytm Jarvisa}
   		Dzięki wykorzystaniu pattern matchingu dostępnego w Scali, możliwe jest przejrzyste zaimplementowanie algorytmu Jarvisa, który, jak widać jest dużo mniej skomplikowany od algorytmu Grahama. Mniejsze skomplikowanie implementacji wiąże się jednak z dużo większą złożonością obliczeniową.
        \lstinputlisting[language=scala,caption={Jarvis.scala},captionpos=b]{\listingsDirectory/Jarvis.scala}
        \lstinputlisting[language=scala,caption={JarvisAlgorithmTest.scala},captionpos=b]{\listingsDirectory/JarvisAlgorithmTest.scala}
    \chapter{Dynamiczna otoczka wypukła}
    	\section{Algorytm}
    	\section{Implementacja w języku Scala} 
	\chapter{Podsumowanie} 
      
    \begin{thebibliography}{9}
    \addcontentsline{toc}{chapter}{Bibliografia}
    \bibitem{convexhullsimplepolygon}
    Ronald L. Graham, Frances Yao, Finding the Convex Hull of a Simple Polygon (1981)
    \bibitem{online}
    Avraham A. Melkman, On-line Construction of the Convex Hull of a Simple Polyline (1985)
    \bibitem{cartography}
    Jacqueleen Jourban, Yair Gabay, A Method for Construction of 2D Hull For Generalized Cartographic Representation (2000)
    \bibitem{gpu}
    Min Tang, Jie-yi Zhao, Ruo-feng Tong, Dinesh Manocha, GPU accelerated convex hull computation (2012)
    \bibitem{detection}
    Navjot Singh, Rinki Arya, R.K. Agrawal, A convex hull approach in conjunction with Gaussian mixture model for salient object detection (2016)
\end{thebibliography}
\end{document}
