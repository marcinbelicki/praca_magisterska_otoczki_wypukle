\newcommand*{\includesDirectory}{includes}
\newcommand*{\settingsDirectory}{\includesDirectory/settings}

\documentclass[a4paper,12pt,oneside]{book}
\usepackage{geometry}
\usepackage{indentfirst}

\geometry{
    left=35mm,
    right=25mm,
    top=25mm,
    bottom=25mm
}

\renewcommand{\baselinestretch}{1.5}
\usepackage[T1]{fontenc}
\usepackage[polish]{babel}
\usepackage{lmodern}
\usepackage{fancyhdr}
\pagestyle{fancy}
\fancyhf{}
\renewcommand{\headrulewidth}{0pt}
\fancyhead[R]{\thepage}

\fancypagestyle{plain}{
    \fancyhf{}
    \renewcommand{\headrulewidth}{0pt}
    \fancyhead[R]{\thepage}
}
\usepackage{titletoc}
\usepackage{hyperref}
\setcounter{tocdepth}{1}

\renewcommand\thechapter{Rozdział \Roman{chapter}}
\renewcommand\thesection{\arabic{section}.}
\renewcommand\thesubsection{\arabic{section}.\arabic{subsection}}
\contentsmargin{0.5cm}
\newlength\chapterlength
\settowidth\chapterlength{\hspace{2.5cm}}

\titlecontents{chapter}
    [\chapterlength] %5.3
    {\vspace{0.2cm}}
    {\contentslabel[\thecontentslabel]{\chapterlength}}%\thecontentslabel
    {\hspace*{-\chapterlength}}% unnumbered chapters
    {\titlerule*[1cm]{.}\contentspage}[\vspace{0.2cm}]%

\titlecontents{section}
    [\chapterlength] %5.3
    {\small}
    {\contentslabel[\thecontentslabel]{0.5cm}}
    {}
    {\titlerule*[0.5cm]{.}\contentspage}[]




\usepackage{titlesec}


\titleformat{\chapter}[block]
  {\normalfont\huge}{\thechapter\vspace{-13pt}\\}{0pt}{\LARGE}
\titlespacing*{\chapter}{0pt}{0pt}{0pt}

\titleformat{\section}[block]
  {\normalfont}
  {\makebox[0.5cm][l]{\thesection}}{10pt}{}
\titlespacing*{\section}{0pt}{0pt}{0pt}

\titleformat{\subsection}[block]
  {\normalfont}
  {\hspace{1cm}\makebox[0.5cm][l]{\thesubsection}}{10pt}{}
\titlespacing*{\subsection}{0pt}{0pt}{0pt}

\begin{document}
    \thispagestyle{empty}
    \begin{center}
        \Huge{Otoczki wypukłe}
    \end{center}
    \tableofcontents				

    \newcommand\intro{Wstęp}
    
    \chapter*{\intro}  
    \addcontentsline{toc}{chapter}{\intro}      

    \chapter{Omówienie teoretyczne otoczki wypukłej na płaszczyźnie}
        \section{Otoczka wypukła zbioru punktów}
        \section{Otoczka wypukła wielokąta prostego}
        \section{Redukcja zbioru punktów do wielokąta prostego}  
    \chapter{Zastosowania} 
        \section{Generalizacja kartograficzna}
		\section{Grafika komputerowa}
		\section{Detekcja obiektów}
        \section{Wyznaczanie obwiedni sygnału}
    \chapter{Dynamiczna otoczka wypukła}
    	\section{Algorytm}
    	\section{Implementacja w języku Scala} 
	\chapter{Podsumowanie} 
      
    \begin{thebibliography}{9}
    \addcontentsline{toc}{chapter}{Bibliografia}
    \bibitem{convexhullsimplepolygon}
    Ronald L. Graham, Frances Yao, Finding the Convex Hull of a Simple Polygon (1981)
    \bibitem{online}
    Avraham A. Melkman, On-line Construction of the Convex Hull of a Simple Polyline (1985)
    \bibitem{cartography}
    Jacqueleen Jourban, Yair Gabay, A Method for Construction of 2D Hull For Generalized Cartographic Representation (2000)
    \bibitem{gpu}
    Min Tang, Jie-yi Zhao, Ruo-feng Tong, Dinesh Manocha, GPU accelerated convex hull computation (2012)
    \bibitem{detection}
    Navjot Singh, Rinki Arya, R.K. Agrawal, A convex hull approach in conjunction with Gaussian mixture model for salient object detection (2016)
\end{thebibliography}
\end{document}
