\newcommand*{\includesDirectory}{includes}
\newcommand*{\settingsDirectory}{\includesDirectory/settings}

\documentclass[a4paper,12pt,oneside]{book}
\usepackage{geometry}
\usepackage{indentfirst}

\geometry{
    left=35mm,
    right=25mm,
    top=25mm,
    bottom=25mm
}

\renewcommand{\baselinestretch}{1.5}
\include{\settingsDirectory/language}
\usepackage{fancyhdr}
\pagestyle{fancy}
\fancyhf{}
\renewcommand{\headrulewidth}{0pt}
\fancyhead[R]{\thepage}

\fancypagestyle{plain}{
    \fancyhf{}
    \renewcommand{\headrulewidth}{0pt}
    \fancyhead[R]{\thepage}
}
\usepackage{titletoc}
\usepackage{hyperref}
\setcounter{tocdepth}{1}

\renewcommand\thechapter{Rozdział \Roman{chapter}}
\renewcommand\thesection{\arabic{section}.}
\renewcommand\thesubsection{\arabic{section}.\arabic{subsection}}
\contentsmargin{0.5cm}
\titlecontents{chapter}
    [2.2cm] %5.3
    {\vspace{0.2cm}}
    {\contentslabel[\thecontentslabel]{2.2cm}}%\thecontentslabel
    {\hspace*{-2.2cm}}% unnumbered chapters
    {\titlerule*[1cm]{.}\contentspage}[\vspace{0.2cm}]%

\titlecontents{section}
    [2.2cm] %5.3
    {}
    {\contentslabel[\thecontentslabel]{0.5cm}}
    {}
    {\titlerule*[0.5cm]{.}\contentspage}[]



\titlecontents{subsection}
    [2.8cm] %5.3
    {}
    {\contentslabel[\thecontentslabel]{0.7cm}}
    {}
    {\titlerule*[0.5cm]{.}\contentspage}[]



\usepackage{titlesec}


\titleformat{\chapter}[block]
  {\normalfont\huge}{\thechapter\vspace{-13pt}\\}{0pt}{\LARGE}
\titlespacing*{\chapter}{0pt}{0pt}{0pt}

\titleformat{\section}[block]
  {\normalfont}
  {\makebox[0.5cm][l]{\thesection}}{10pt}{}
\titlespacing*{\section}{0pt}{0pt}{0pt}

\titleformat{\subsection}[block]
  {\normalfont}
  {\hspace{1cm}\makebox[0.5cm][l]{\thesubsection}}{10pt}{}
\titlespacing*{\subsection}{0pt}{0pt}{0pt}

\begin{document}
    \thispagestyle{empty}
    \begin{center}
        \Huge{Otoczki wypukłe}
    \end{center}
    \tableofcontents	

  
    \chapterwithout{Wstęp}        

    \chapter{Omówienie teoretyczne otoczki wypukłej na płaszczyźnie}
        \section{Otoczka wypukła zbioru punktów}
        \section{Otoczka wypukła wielokąta prostego}
        \section{Redukcja zbioru punktów do wielokąta prostego}
    \chapter{Zastosowania} 
    Istotnym, jeśli nie najistotniejszym zagadnieniem dotyczącym otoczek wypukłych są ich zastosowania. Już od kilkudziesięciu lat problem ten temat znajduje swoje użycie we wielu dziedzinach kombinatoryki i informatyki. Dla przykładu problem sortowania elementów liczbowych listy można sprowadzić do problemu znalezienia otoczki wypukłej. W związku z czym rozwiązanie tego problemu może pomóc w rozwiązaniu innych problemów w bardziej symboliczny i graficzny sposób.
        \section{Generalizacja kartograficzna}
        W celu jak najbardziej informatywnego i niezłożonego przedstawienia danych geograficznych w sposób graficzny potrzebne jest zastosowanie algorytmu generalizującego informacje.
		\section{Grafika komputerowa}
		Obiekty wykorzystywane w grafice komputerowej często mogą charakteryzować się skomplikowanymi kształtami. Im bardziej skomplikowany kształt tym więcej mocy obliczeniowej potrzebne będzie w celu wykonania danej operacji na tym kształcie. W niektórych przypadkach do uproszczenia graficznego danego obiektu używa się otoczki wypukłej jego kształtu. Ze względu na fakt, iż otoczka wypukła wielokąta zawsze będzie miała liczbę wierzchołków mniejszą lub równą liczbie wierzchołków samego wielokąta, powstała otoczka może posłużyć do wykonania mniejszej liczby obliczeń przy wykonywaniu operacji na danym obiekcie.
		\section{Detekcja obiektów}
        \section{Wyznaczanie obwiedni sygnału}
        
    \chapter{Dynamiczna otoczka wypukła}
    	\section{Algorytm}
    	\section{Implementacja w języku Scala} 
	\chapter{Podsumowanie} 
      
    \begin{thebibliography}{9}
    \addcontentsline{toc}{chapter}{Bibliografia}
    \bibitem{convexhullsimplepolygon}
    \href{https://mathweb.ucsd.edu/~ronspubs/83_09_convex_hull.pdf}
    {Ronald L. Graham, Frances Yao, Finding the Convex Hull of a Simple Polygon (1981)}
    \bibitem{online}
    \href{https://www.ime.usp.br/~walterfm/cursos/mac0331/2006/melkman.pdf}
    {Avraham A. Melkman, On-line Construction of the Convex Hull of a Simple Polyline (1985)}
    \bibitem{cartography}
    \href{https://www.isprs.org/proceedings/xxxiii/congress/part4/417_XXXIII-part4.pdf}
    {Jacqueleen Jourban, Yair Gabay, A Method for Construction of 2D Hull For Generalized Cartographic Representation (2000)}
    \bibitem{gpu}
    \href{https://www.sciencedirect.com/science/article/pii/S0097849312000544}
    {Min Tang, Jie-yi Zhao, Ruo-feng Tong, Dinesh Manocha, GPU accelerated convex hull computation (2012)}
    \bibitem{detection}
     \href{https://www.sciencedirect.com/science/article/pii/S1051200416300318}
     {Navjot Singh, Rinki Arya, R.K. Agrawal, A convex hull approach in conjunction with Gaussian mixture model for salient object detection (2016)}
    \bibitem{dynamic}
    \href{https://www.sciencedirect.com/science/article/pii/S1568494619306775}
    {Fan Cheng, Qiangqiang Zhang, Ye Tian, Xingyi Zhang, Maximizing receiver operating characteristics convex hull via dynamic reference point-based multi-objective evolutionary algorithm (2019)}
\end{thebibliography}
\end{document}
